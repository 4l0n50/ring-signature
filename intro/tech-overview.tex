% !TEX root = ../main-ring-signature.tex

Most ring signature constructions have followed the following approach. Given a ring of users defined by the set of their public keys and a message: a) sign the message, b) prove in zero-knowledge knowledge of a signature which can be verified using some committed/randomized verification key, and then c) prove in zero-knowledge that this verification key belongs to the set of public keys in the ring.  The most expensive part is c) and is sometimes called a \emph{set-membership proof}.

We observe that in the case of ring signatures \emph{all the verification keys forming the ring are honestly generated}. 
Indeed, it only makes sense to guarantee unforgeability when all the members of the ring are honest (otherwise the adversary knows at least one secret key) and thus the set-membership proof might assume that all verification keys were honestly generated.
It turns out that all the schemes we are aware of, in particular Chandran et al.'s, obviate this property, meaning that their set-membership proofs are for adversarially chosen verification keys.
We ask the following natural question.
\begin{displayquote}
Can we construct more efficient set membership proofs (without random oracles or non-falsifiable assumptions) when verification keys are sampled from a known distribution?
\end{displayquote}
We answer this question in the affirmative and we construct a $\Theta(\sqrt[3]{n})$ set membership proof when the verification keys are honestly sampled. In contrast, Chandran et al.'s proof is of size $\Theta(\sqrt{n})$ but it makes no assumption on the verification keys distribution.

Our main technical tool are two structure preserving --- i.e.~compatible with Groth-Sahai proofs --- hash functions The first funtion, $h$, is \emph{second-preimage resistant} under a slightly different notion of collision. That is, given $A$ in the domain of $h$, is hard to find $A,A'$ such that $h(A')=h(A)$ {\bf whenever $A$ is randomly sampled from some known distribution} and whenever $A'$ is not a permutation of $A$.
Our function $h$ takes as input $A$, a list of verification keys (in fact, a fragment of each verification key), and returns a constant size digest which is simply $h(A)= \sum_{i=1}^m a_i$. Thereby, whenever $h$ is applied to a set of honestly sampled verification keys, $A$ is indeed random and hence is infeasible to compute a second preimage. 

Our second function $g$ is collision-resistant in the traditional sense an uses $A$ as its key and returns $g_A(x_1,\ldots,x_m) = \sum_{i=1}^m a_ix_i$. Both functions can be used together with the following nice property, given $A,A'$ such that $h(A)=h(A')$, then $g_A(x)=g_{A'}(x')$ implies that $x$ is a permutation of $x'$.

\subsubsection{High level description.}
In our ring signature, the verification key will contain $a,vk$ and $a\cdot vk$, where $vk$ the verification key of a signature scheme. We want to show that some commitment $c$ opens to $vk$ and $vk\in\{vk_1,\ldots,vk_n\}$. To do so, we arrange the $n$ elements of the ring into $n^{2/3}$ blocks of size $m=\sqrt[3]{n}$. We use the following notation: for $\{s_1,\ldots, s_n\}$ define $s_{i,j}:=s_{(i-1)m+j}$, where  $1\leq i \leq n^{2/3},1\leq  j \leq m$. Assume that $vk = vk_{\mu,\nu}$.

We divide $\{a_{1},\ldots, a_{n}\}$ into $A_1 := (a_{1,1},\ldots,a_{1,m}),\ldots, A_{n^{2/3}} := (a_{n^{2/3},1},\ldots,\allowbreak a_{n^{2/3},m})$ and we define $H\allowbreak:=\{h(A_1),\allowbreak\ldots,\allowbreak h(A_{n^{2/3}})\}.$  
We use Chandran et al.'s set-membership proof of size $\Theta(\sqrt{n})$ to prove knowledge of some $h(A_\mu)\in H$. Since $|H|=n^{2/3}$, this proof is of size $\Theta(\sqrt[3]{n})$. Then we prove knowledge of $A'$, a preimage of $h(A_\mu)$, which using Groth-Sahai proofs requires commitments to the $\sqrt[3]{n}$ elements in $A'$ plus a $\Theta(1)$ proof that $h(A')=h(A_\mu)$. Hence, this part of the proof adds up to $\Theta(\sqrt[3]{n})$ group elements.
%Since $h$ is invariant under permutations of the input, we can sort the commitments of the $\sqrt[3]{n}$ elements in $A'$ so that the first commitment is $c$. We conclude that $A'=A_\mu$, unless we break second-preimage resistance of $h$, and thus $a$, the opening of $c$, belongs to $\{a_1,\ldots,a_n\}$.

We give another set-membership proof that
$g\in\{
g_{A_1}(vk_{1,1},\ldots,vk_{1,m}),
\ldots,\allowbreak
g_{A_{n^{2/3}}}(
	vk_{n^{2/3},1},
	\ldots,
	vk_{n^{2/3},m})\}$ such that $g$ is also at the $\mu$-th position in the set (this is straightforward to do in Chandran et al.'s set-membership proof).
We commit to $x_1=vk_{\mu,\nu},\ldots,x_m=vk_{\mu,m}$ and we prove using Groth-Sahai proofs that $g = g_{A'}(x_1,\ldots,x_m)$. Again, this part of the proof adds $\Theta(\sqrt[3]{n})$ group elements.

Finally, since $h(A')=h(A_\mu)$ and $g_{A'}(x_1,\ldots,x_m) = g = g_{A_{\mu}}(vk_{\mu,1},\ldots,x_{\mu,m})$ we conclude that $x_1,\ldots,x_m$ is a permutation of $vk_{\mu,1},\ldots,vk_{\mu,m}$. I suffices to give a third set membership proof that $c$ opens to some element in $\{x_1,\ldots,x_m\}$ (but we can omit this proof setting $x_1 =vk_{\mu,\nu}$ and a clever use of the ``permutation friendly'' properties of $h$ and $g$).