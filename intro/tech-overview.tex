Consider a symmetric bilinear group $gk:=(\GG,\GG_T,e,\mathcal{P},q)$ of prime order $q$, where $\mathcal{P}$ is a generator of $\GG$. Define $[x]=x\mathcal{P}$ for any $x\in\mathbb{Z}_q$.  We will write all group operations using additive notation.

Our main technical tool is a structure preserving -- i.e.~compatible with Groth-Sahai proofs -- hash function with \emph{always second-preimage resistance} (aSec in the terminology of Rogaway and Shrimpton \cite{FSE:RogShr04})
\begin{align*}
h : Q_m &\to \GG^2\\
      A &\mapsto h(A) := \sum_{[\vecb{a}]\in A} [\vecb{a}],
\end{align*}
 where
$$
Q_m:= \{A\subset\GG^2: |A|=m \text{ and }\forall [\vecb{a}]\in A, e([a_2],[1])=e([a_1],[a_1])\}.
$$
That is, for a randomly chosen $A\in Q_m$, it is computationally infeasible to find a different $A'\in Q_m$ such that and $h(A)=h(A')$. On Section \ref{sec:hash} we show that the hardness of finding second preimages for $h$ is a direct consequence of the permutation pairing assumption.

We consider also a family of collision resistant hash functions parametrized by $[\matr{A}]\in\GG^{2\times m}$
\begin{align*}
g_{[\matr{A}]} : \GG^m &\to \GG^2\\
           [\vecb{x}] &\mapsto g_{[\matr{A}]}([\vecb{x}]) = [\matr{A}\vecb{x}]
\end{align*}
Given $[\matr{A}],[\matr{A}']\in\GG^{2\times m}$ define $A,A'$ as the sets whose elements are the columns of, respectively, $[\matr{A}]$ and $[\matr{A}']$. Then $g_{[\matr{A}]}([\vecb{x}]) = g_{[\matr{A}']}([\vecb{x}'])$ and $A=A'$ implies that there is a permutation matrix $\matr{P}$ such that $[\vecb{x}]=\matr{P}[\vecb{x}']$ unless $[\vecb{x}],\matr{P}[\vecb{x}']$ is a collision for $g_{[\matr{A}]}$.
 
 Finding collosions for $g_{[\matr{A}]}$ is as hard as finding a non-zero element in the kernel of $[\matr{A}]$, since
$$
g_{[\matr{A}]}([\vecb{x}]) = g_{[\matr{A}]}([\vecb{x}']) \Longleftrightarrow [\matr{A}(\vecb{x}-\vecb{x}')]=[0],
$$
which is in general a hard problem. Indeed,
Morillo et al.~\cite{AC:MorRafVil16} formally defined this computational (or search) problem as the kernel matrix Diffie-Hellman assumption (KerMDH) and proved the hardness of the KerMDH assumption in generic bilinear groups for many distributions of $[\matr{A}]$. For some specific distributions (e.g.~the uniform distribution) it can be proven harder than the decisional linear assumption. If $A$ is randomly chosen from $Q_m$, then finding an element on the kernel of $[\matr{A}]$ was proven hard in generic bilinear groups by Groth and Lu \cite{AC:GroLu07} (although using a different terminology).
The KerMDH assumption has found many applications such as constructing constant size QA-NIZK proofs of membership in the linear span of a matrix \cite{EC:LPJY14,EC:KilWee15}.

\subsubsection{High level description.}
Our scheme builds on top of the ring signature of Chandran et al.~and improves the underlying $\Theta(\sqrt{n})$ proof that the Groth-Sahai commitment of a Boneh-Boyen signature verification key $[vk]\in\GG$ belongs to the ring of verification keys $R=\{[vk_1],\ldots,[vk_n]\}$. In the rest of this section we simply refer to this proof as a ``set-membership proof'' and we remark that it might be applied to any set of group elements (not only of verification keys).

Our proof consists of two set-mebership proofs in sets of size $\Theta(n^{2/3})$ -- i.e.~each proof is of size $\Theta(\sqrt[3]{n})$ --  plus $\Theta(\sqrt[3]{n})$ Groth-Sahai proofs and Groth-Sahai commitments.
We enlarge the verification key by including $[\vecb{a}]\gets Q_1$ and $sk[\vecb{a}]=\vecb{a}[vk]$, where $[vk]$ is the verification key of a Boneh-Boyen signature and $sk=vk$ is the secret key (recall that $vk$ is the discrete logarithm of $[vk]$). In spite of this differences, our proof also shows $[vk]$ is in $\{[vk_1],\ldots,[vk_n]\}$.

For a sequence $\{s\}_{1\leq i \leq n}$ define $s_{\mu,\nu}:=s_{(\mu-1)m+\nu}$, where $m:=\sqrt[3]{n}$ and $1\leq\mu\leq n^{2/3},1\leq \nu\leq m$.  The prover and the verifier split the verification keys into $[\vecb{\kappa}_1],\ldots, [\vecb{\kappa}_{n^{2/3}}]$, $A_1,\ldots, A_{n^{2/3}}$, and $[\matr{A}_1],\ldots, [\matr{A}_{n^{2/3}}]$ such that $[\vecb{\kappa}_i]:= ([vk_{i,1}],\ldots,[vk_{i,m}])^\top\in\GG^m, A_i:=\{[\vecb{a}_{i,1}],\ldots,[\vecb{a}_{i,m}]\}\in Q_m$ and $[\matr{A}_i]=[\vecb{a}_{i,1}\cat\cdots\cat\vecb{a}_{i,m}]\in\GG^{2\times m}$. They also define the sets
\begin{align*}
&H:=\{h(A_1),\ldots,\allowbreak h(A_{n^{2/3}})\}\\
&G:=\{g_{[\matr{A}_1]}([\vecb{\kappa}_1]),\allowbreak\ldots,g_{[\matr{A}_{n^{2/3}}]}([\vecb{\kappa}_{n^{2/3}}])\},
\end{align*}
where $g_{[\matr{A}_i]}([\vecb{\kappa}_i])$ is computed as $\sum_{i=1}^m \vecb{a}_{i,j}[vk_{i,j}]$.

The prover starts computing Groth-Sahai commitments to $[vk]=[vk_\alpha]=[vk_{\mu,\nu}]$ and to $h(A_\mu)$, for some $1\leq \alpha \leq n$, and computes the first set-membership proof showing that $h(A_\mu)\in H$.
The prover commits to each element of $A'=A_\mu$ such that the first commited element is $[\vecb{a}_\alpha]$ and the rest columns are the elements columns of $[\matr{A}_\mu]$. Denote by $[\matr{A}']$ the matrix whose columns are the elements of $A'$ in the order defined before.  The prover shows using Groth-Sahai proofs that $A'\in Q_m$ and that $h(A')=h(A_\mu)$. From this part of the proof we know that, with all but negligible probability, $A'=A_\mu$.

Next, the prover computes Groth-Sahai commitments to each element of the vector $[\vecb{\kappa}']$, whose first element is $[vk_\alpha]$ and the rest are the other verification keys in $[\vecb{\kappa}_\mu]$, and commits also to $g_{[\matr{A_{\mu'}}]}([\vecb{\kappa}_{\mu'}])$, where $\mu'=\mu$. The second set-membership proof shows that $g_{[\matr{A_{\mu'}}]}([\vecb{\kappa}_{\mu'}])\in G$ and that $\mu'=\mu$. Finally, the prover gives a Groth-Sahai proof that $g_{[\matr{A}']}([\vecb{\kappa}'])=g_{[\matr{A_{\mu'}}]}([\vecb{\kappa}_{\mu'}])$.

Sine $A'=A_\mu$, we conclude that $[\vecb{\kappa}']$ is a permutation of $[\vecb{\kappa}_\mu]$ and thus $[\kappa'_1]=[vk_\alpha]$ is in the ring of verification keys.


