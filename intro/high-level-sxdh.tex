% !TEX root = ../main-ring-signature.tex


%We propose an alternative way of obtaining a $\sqrt{n}$ ring signature. We then combine both techniques, Chandran et al. and ours, and obtain a $\sqrt[3]{n}$ signature.
%
%
%Following Chandran et al.'s approach, if we want to obtain a $m:=\sqrt[3]{n}$ proof it is natural to arrange the verification keys in $m$ matrices of size $m\times m$ (a 3d array) as depcited bellow
%$$
%\begin{pmatrix}
%[vk_{1,1,1}] & \cdots & [vk_{1,1,m}]\\
%\vdots       & \ddots & \vdots      \\
%[vk_{1,m,1}] & \cdots & [vk_{1,m,m}]
%\end{pmatrix},
%\ldots,
%\begin{pmatrix}
%[vk_{m,1,1}] & \cdots & [vk_{m,1,m}]\\
%\vdots       & \ddots & \vdots      \\
%[vk_{m,m,1}] & \cdots & [vk_{m,m,m}]
%\end{pmatrix},
%$$
%where $vk_{i,j,k}:=vk_{(i-1)m^2+(j-1)m+k}$ for $i,j,k\in[m]$.
%
%The naive approach of selecting one of these matrices and then applying Chandran et al.'s approach will end up with a proof of size $n^{2/3}$. We follow the approach of Gonzalez et al.~\cite{AC:GonHevRaf15} which aggreates the $m$ matrices into a single one selecting $\vecb{a}_1,\ldots,\vecb{a}_n$ from some distribution such that the corresponding kernel problem is hard (using the terminology of Morillo et al.~\cite{AC:MorRafVil16}). That is, compute
%$$
%[\matr{V}] := \sum_{i=1}^{m} \vecb{a}_i
%\begin{pmatrix}
%[vk_{i,1,1}] & \cdots & [vk_{i,1,m}]\\
%\vdots       & \ddots & \vdots      \\
%[vk_{i,m,1}] & \cdots & [vk_{i,m,m}]
%\end{pmatrix}
%$$


\subsubsection{High Level Description.} %Our scheme builds on top of the ring signature of Chandran et al.~and improves the underlying $\Theta(\sqrt{n})$ proof that the opening of a Groth-Sahai commitment is a Boneh-Boyen signature verification key $vk$ and belongs to the ring of verification keys $R=\{vk_1\ldots,vk_n\}$. In the rest of this section we simply refer to this proof as a ``set-membership proof'' and we remark that it might be applied to any set of (vectors of) group elements (not only of verification keys).
%
Our proof consists of two set-membership proofs in sets of size $n^{2/3}$ --- i.e.~each proof is of size $\Theta(\sqrt[3]{n})$ ---  plus $\Theta(\sqrt[3]{n})$ Groth-Sahai proofs and Groth-Sahai commitments.
Let $(x,vk)$ be Boneh-Boyen secret/verification keys (note that $[x]_2=vk$).
We enlarge user's verification keys the by including $g$'s local key $([\vecb{a}]_1, \pi)\gets \KGen_{\mathsf{local}}(gk,k_0,\beta)$, where $\beta=0$, plus $[\vecb{c}]_2,[\vecb{d}]_1$ and $[\vecb{\psi}]_2,[\vecb{\omega}]_1$ defined below. 
%Recall that $[\vecb{a}]_1$, where $\vecb{a}\gets\mathcal{Q}_1^\beta$, is a commitment to $\beta$, as well as $[\vecb{b}]_2=\Com_{[\matr{V}]_2}(\beta;\rho)$.
We compute $[\vecb{c}]_2=\Com_{[\matr{W}]_2}(x;s)$, $[\vecb{d}]_1 = x[\vecb{a}]_1+t[\vecb{u}_2]_2=\Com_{[\matr{U}]_1}(y)$, for $y:=\beta x$, and $[\vecb{\psi}]_2,[\vecb{\omega}]_1$ is a Groth-Sahai proof that $\beta x = y$ (see App. \ref{sec:GSproofs-g} for more details). Thereby, verification key of the $i$ th user is of the form $\vecb{vk}_i:=([x_i]_2,[\vecb{a}_i]_1,[\vecb{b}_i]_2,[\vecb{c}_i]_2,[\vecb{d}_i]_1,[\vecb{\psi}_i]_2,[\vecb{\omega}_i]_1,\pi_i)$.

Note that from commitments $[\vecb{a}_1]_1,\ldots,[\vecb{a}_n]_1$ one can derive a commitment to $h(\matr{A})=h((\vecb{a}_1\cat\cdots\cat\vecb{a}_n))$ by simply computing $\sum_{i=0}^n [\vecb{a}_i]_1$. Similarly one can also compute $[\vecb{g}]_1 = \sum_{i=1}^m[\vecb{d}_i]_1,[\vecb{\psi}]_2 = \sum_{i=1}^m [\vecb{\psi}_i]_2$, and $[\vecb{\omega}_i]_1 = \sum_{i=1}^m [\vecb{\omega}_i]_1$ such that
\begin{align*}
[\vecb{g}]_1[\vecb{w}_1^\top]_2 &= \sum_{i=1}^m [\vecb{d}_i]_1[\vecb{w}_1^\top]_2
= \sum_{i=1}^m [\vecb{a}_i][\vecb{c}_i^\top] - [\vecb{u}_2]_1\left(\sum_{i=1}^m[\vecb{\psi}_i]_2\right)^\top  -\left(\sum_{i=1}^m[\vecb{\omega}_i]_1\right)[\vecb{w}_2^\top]_2\\
&= [\matr{A}]_1[\matr{C}^\top]_2 - [\vecb{u}_2]_1[\vecb{\psi}]_2 - [\vecb{\omega}]_1[\vecb{w}_2^\top]_2
= g_{\matr{A}}([\matr{C}]_2,[\vecb{\psi}]_2,[\vecb{\omega}]_1),
\end{align*}
where the first step follows from the fact that $[\vecb{d}_i]_1$ satisfies equation (\ref{eq:ver-betax=y}) and $\matr{C}=(\vecb{c}_1\cat\cdots\cat\vecb{c}_m)$. %Note also that $\pi_1,\ldots,\pi_n$ is a proof that $\matr{A}\in\mathcal{Q}_n$.

%In spite of all these differences with Chandran et al.'s proof, we also show that the opening of a Groth-Sahai commitment is a Boneh-Boyen verification key $[x_i]_2$ and belongs to $\{[x_1]_2,\ldots,[x_n]_2\}$.
Our first step is to arrange the verification keys in $n^{2/3}$ blocks of size $m=\sqrt[3]{n}$. To do so we use the following notation: for a sequence $\{s\}_{1\leq i \leq n}$ define $s_{\mu,\nu}:=s_{(\mu-1)m+\nu}$, where  $1\leq\mu\leq n^{2/3},1\leq \nu\leq m$. The prover and the verifier arrange some elements of the verification keys into $[\matr{C}_1]_2,\ldots, [\matr{C}_{n^{2/3}}]_2$, and $[\matr{A}_1]_1,\ldots, [\matr{A}_{n^{2/3}}]_1$ such that $[\matr{C}_i]_2:= [\vecb{c}_{i,1}\cat\cdots\cat\vecb{c}_{i,m}]_2\in\GG_2^{2\times m}$,  and $[\matr{A}_i]_1=[\vecb{a}_{i,1}\cat\cdots\cat\vecb{a}_{i,m}]_1\in\GG^{2\times m}_1$. They also define the sets
\begin{align*}
&H:=\{[\vecb{h}_1]_1,\ldots,\allowbreak [\vecb{h}_{n^{2/3}}]_1\},
&G:=\{[\vecb{g}_1]_1,\ldots,[\vecb{g}_{n^{2/3}}]_1\}, 
\end{align*}
where $[\vecb{h}_i]_1 = \sum_{j=1}^m [\vecb{a}_{i,j}]_1$ and $[\vecb{g}_i]_1 =  \sum_{j=1}^m [\vecb{d}_{i,j}]_1$.

Assume that prover's secret key is $x_\alpha = x_{\mu,\nu}$.
The prover starts computing $[\vecb{h}']_1$,
%a Groth-Sahai commitment $[\vecb{c}_{\mu,\nu}]_2$ to its secret key $x_\alpha=x_{\mu,\nu}$, and 
a re-randomization of $[\vecb{h}_\mu]_1$, and computes the first set-membership proof showing that $[\vecb{h}']_1\in H$.
The prover computes a matrix $[\matr{A}']_1$ such that the first column is a re-randomization of $[\vecb{a}_{\mu,\nu}]_1$, and the columns that follow are re-randomizations of the other columns of $[\matr{A}_\mu]_1$ (preserving the order). The prover also derives a proof that $\matr{A}'\in\mathcal{Q}_m$ re-randomizing proofs $\pi_{\mu,1},\ldots,\pi_{\mu,m}$ and shows that $h(\matr{A}_\mu)=h(\matr{A}')$ with a Groth-Sahai proof of the satisfiability of equation equation (\ref{eq:coll-h}). Then, the prover computes $[\vecb{g}']_1$, a re-randomization of $[\vecb{g}_{\mu'}]$ where $\mu'=\mu$, and computes a  second set-membership proof showing that $[\vecb{g}']\in G$ and that $\mu'=\mu$.

Let $[\vecb{\psi}_\mu]_2 := \sum_{i=1}^m[\vecb{\psi}_{\mu,i}]_2$ and $[\vecb{\omega}_\mu]_1 := \sum_{i=1}^m[\vecb{\omega}_{\mu,i}]_1$. Note that $g_{\matr{A}_\mu}([\matr{C}_\mu]_2,\allowbreak[\vecb{\psi}_\mu]_2,[\vecb{\omega}_\mu]_1) = [\vecb{g}_\mu]_1[\vecb{w}_1^\top]_2$. Let $[\matr{C}']_2$ such that its first column is a re-ran\-do\-mi\-za\-tion of $[\vecb{c}_{\mu,\nu}]$ and the following columns are re-ran\-do\-mi\-za\-tions of $[\vecb{c}_{\mu,1}]_1,\ldots,\allowbreak[\vecb{c}_{\mu,\nu-1}]_1,\allowbreak[\vecb{c}_{\mu,\nu+1}]_1,\ldots,[\vecb{c}_{\mu,m}]$. Let $[\vecb{\psi}']_2,[\vecb{\omega}']_1$ as computed in equation (\ref{eq:g-rerand-proofs}).  It holds that $g_{\matr{A}'}([\matr{C}']_2,[\vecb{\psi}']_2,[\vecb{\omega}']_1) = [\vecb{g}']_1[\vecb{w}_1^\top]_2$.

%For $i\in[m]$, consider $[\vecb{c}'_i]_2,[\vecb{d}'_i]_1$ re-randomizations of, respectively, $[\vecb{c}_{\mu,i}]_1,[\vecb{d}_{\mu,i}]_2$ and $\theta'_i = ([\vecb{\psi}'_i]_2,[\vecb{\omega}']_1)$ the corresponding re-randomized proof. Clearly, $[\vecb{d}']$
%Consider proofs $\theta_{\mu,\nu},\theta_{\mu,1},\ldots,\allowbreak\theta_{\mu,\nu-1},\theta_{\mu,\nu+1},\ldots,\theta_{\mu,m}$. Recall that $\theta_{i,j} = \allowbreak([\vecb{\psi}_{i,j}]_2,\allowbreak[\vecb{\omega}_{i,j}]_1)$ is such that $g_{\vecb{a}_{i,j}}([x_{i,j}]_2,[\vecb{\psi}_{i,j}]_2,[\vecb{\omega}_{i,j}]_1)=[\vecb{d}_{i,j}]_1[\vecb{w}_1^\top]_2$. 

%Next, the prover computes $[\matr{C}']_2,[\matr{D}']_2\in\GG_2^{2\times m}$, whose first columns are, respectively, re-randomizations of $[\vecb{c}_{\mu,\nu}],[\vecb{d}_{\mu,\nu}]$ and the following columns are re-ran\-do\-mi\-za\-tions of $[\vecb{c}_{\mu,1}]_1,\ldots,[\vecb{c}_{\mu,\nu-1}]_1,\allowbreak[\vecb{c}_{\mu,\nu+1}]_1,\ldots,[\vecb{c}_{\mu,m}]$ and $[\vecb{d}_{\mu,1}]_1,\ldots,[\vecb{d}_{\mu,\nu-1}]_1,\allowbreak[\vecb{d}_{\mu,\nu+1}]_1,\allowbreak\ldots,\allowbreak[\vecb{d}_{\mu,m}]$ (preserving the order). It re-randomizes proofs $\theta_{\mu,\nu},\theta_{\mu,1},\ldots,\allowbreak\theta_{\mu,\nu-1},\theta_{\mu,\nu+1},\ldots,\theta_{\mu,m}$, showing that

We will show that one can extract some $[\vecb{\varphi}]_2$ such that $g_{\matr{A}_\mu}([\matr{C}_\mu]_2,[\vecb{\varphi}]_2,[\vecb{\omega}_\mu]_1) =g_{\matr{A}'}([\matr{C}']_2,[\vecb{\psi}']_2,[\vecb{\omega}']_1)$. Furhtermore,
since $\matr{A}'\sim_{\matr{U}}\matr{A}_\mu$, Lemma \ref{lemma:g-crp-sxdh} implies that $\matr{C}'\sim_\matr{W}\matr{C}$, which in turn implies that $\vecb{x}'$ is a permutation of $\vecb{x}_\mu$. Therefore, $[\vecb{c}_{\mu,\nu}]_1$, opens to $x_{\mu,\nu}=x_\alpha$ which is in the ring of secret keys.