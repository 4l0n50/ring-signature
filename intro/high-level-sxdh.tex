% !TEX root = ../main-ring-signature.tex


%We propose an alternative way of obtaining a $\sqrt{n}$ ring signature. We then combine both techniques, Chandran et al. and ours, and obtain a $\sqrt[3]{n}$ signature.
%
%
%Following Chandran et al.'s approach, if we want to obtain a $m:=\sqrt[3]{n}$ proof it is natural to arrange the verification keys in $m$ matrices of size $m\times m$ (a 3d array) as depcited bellow
%$$
%\begin{pmatrix}
%[vk_{1,1,1}] & \cdots & [vk_{1,1,m}]\\
%\vdots       & \ddots & \vdots      \\
%[vk_{1,m,1}] & \cdots & [vk_{1,m,m}]
%\end{pmatrix},
%\ldots,
%\begin{pmatrix}
%[vk_{m,1,1}] & \cdots & [vk_{m,1,m}]\\
%\vdots       & \ddots & \vdots      \\
%[vk_{m,m,1}] & \cdots & [vk_{m,m,m}]
%\end{pmatrix},
%$$
%where $vk_{i,j,k}:=vk_{(i-1)m^2+(j-1)m+k}$ for $i,j,k\in[m]$.
%
%The naive approach of selecting one of these matrices and then applying Chandran et al.'s approach will end up with a proof of size $n^{2/3}$. We follow the approach of Gonzalez et al.~\cite{AC:GonHevRaf15} which aggreates the $m$ matrices into a single one selecting $\vecb{a}_1,\ldots,\vecb{a}_n$ from some distribution such that the corresponding kernel problem is hard (using the terminology of Morillo et al.~\cite{AC:MorRafVil16}). That is, compute
%$$
%[\matr{V}] := \sum_{i=1}^{m} \vecb{a}_i
%\begin{pmatrix}
%[vk_{i,1,1}] & \cdots & [vk_{i,1,m}]\\
%\vdots       & \ddots & \vdots      \\
%[vk_{i,m,1}] & \cdots & [vk_{i,m,m}]
%\end{pmatrix}
%$$


\subsubsection{High Level Description.} Our scheme builds on top of the ring signature of Chandran et al.~and improves the underlying $\Theta(\sqrt{n})$ proof that the opening of a Groth-Sahai commitment is a Boneh-Boyen signature verification key $[vk]_2\in\GG_2$ and belongs to the ring of verification keys $R=\{[vk_1]_2,\ldots,[vk_n]_2\}$. In the rest of this section we simply refer to this proof as a ``set-membership proof'' and we remark that it might be applied to any set of group elements (not only of verification keys).

Our proof consists of two set-membership proofs in sets of size $n^{2/3}$ --- i.e.~each proof is of size $\Theta(\sqrt[3]{n})$ ---  plus $\Theta(\sqrt[3]{n})$ Groth-Sahai proofs and Groth-Sahai commitments.
We enlarge user's verification keys the by including $g$'s local key $([\vecb{a}]_1,[\vecb{b}]_2,[\vecb{c}]_2,[\vecb{d}]_2, \pi, \theta)\gets \KGen_{\mathsf{local}}(gk,k_0,\beta,sk)$, where $\beta=0$ and $sk$ is the Boneh-Boyen secret key. Recall that $[\vecb{a}]_1$, where $\vecb{a}\gets\mathcal{Q}_1^\beta$ (i.e.~a commitment to $\beta$), and $[\vecb{b}]_2=\Com_{[\matr{V}]_2}(\beta;\rho)$, $[\vecb{c}]_2=\Com_{[\vecb{v}]_2}(sk;\rho)$, $[\vecb{d}]_2 = \Com_{[\matr{V}]_2}(\beta sk;t)$, and $\pi$ and $\theta$ are the proofs from equations (\ref{eq:Qm-memb-proofs}) and (\ref{eq:wi-proofs}), respectively. Thereby, verification key of the $i$ th user is of the form $\vecb{vk}_i:=([vk_i]_2,[\vecb{a}_i]_1,[\vecb{b}_i]_2,[\vecb{c}_i]_2,[\vecb{d}_i]_2,\pi,\theta)$. Note that from commitments $[\vecb{a}_1]_1,\ldots,[\vecb{a}_n]_1$ one can derive a commtiment to $h(\matr{A})=h((\vecb{a}_1\cat\cdots\cat\vecb{a}_n))$ by somply computing $\sum_{i=0}^n [\vecb{a}_i]_1$. Similarly one can derive commitments to $g_{\matr{A}}(\vecb{sk}) = \sum_{i=0}^m\beta_i sk_i$ from $[\vecb{d}_1]_2,\ldots,[\vecb{d}_n]_2$ (recall the $sk$ is the discrete logarithm of $[vk]_2$, i.e.~$sk=vk$). Note also that $\pi_1,\ldots,\pi_n$ is a proof that $\matr{A}\in\mathcal{Q}_n$.

In spite of all these differences with Chandran et al.'s proof, we also show that the opening of a Groth-Sahai commitment is a Boneh-Boyen verification key $[vk_i]_2$ and belongs to $\{[vk_1]_2,\ldots,[vk_n]_2\}$.

Our first step is to arrange the verification keys in $n^{2/3}$ blocks of size $m=\sqrt[3]{n}$. To do so we use the following notation: for a sequence $\{s\}_{1\leq i \leq n}$ define $s_{\mu,\nu}:=s_{(\mu-1)m+\nu}$, where  $1\leq\mu\leq n^{2/3},1\leq \nu\leq m$.  The prover and the verifier arrange some elements of the verification keys into $\vecb{x}_1,\ldots, \vecb{x}_{n^{2/3}}$, and $[\matr{A}_1]_1,\ldots, [\matr{A}_{n^{2/3}}]_1$ such that $\vecb{x}_i:= (sk_{i,1},\ldots,sk_{i,m})^\top\in\Z_q^m$,  and $[\matr{A}_i]_1=[\vecb{a}_{i,1}\cat\cdots\cat\vecb{a}_{i,m}]_1\in\GG^{2\times m}_1$. They also define the sets
\begin{align*}
&H:=\{[\vecb{h}_1]_1,\ldots,\allowbreak [\vecb{h}_{n^{2/3}}]_1\},\text{ where }[\vecb{h}_i]_1 = \Com_{[\matr{U}]_1}(h(\matr{A}_i)) = \sum_{j=1}^m [\vecb{a}_{i,j}]_1 \\
&G:=\{[\vecb{g}_1]_2,\ldots,[\vecb{g}_{n^{2/3}}]_2\},\text{ where } [\vecb{g}_i] = \Com_{[\matr{V}]_2}(g_{\matr{A}_i}(\vecb{x}_i)) = \sum_{j=1}^m [\vecb{d}_{i,j}]_2
\end{align*}

The prover starts computing a Groth-Sahai commitment $[\vecb{c}_{\mu,\nu}]_2$ to its secret key $sk_\alpha=sk_{\mu,\nu}$, and a re-randomization of $[\vecb{h}_\mu]_1$, for some $1\leq \alpha \leq n$, and computes the first set-membership proof showing that $[\vecb{h}_\mu]_1\in H$.
The prover computes a matrix $\matr{A}'$ such that the first column is a re-randomization of $[\vecb{a}_{\mu,\nu}]_1$, and the ones that follow are re-randomizations of the other columns of $[\matr{A}_\mu]_1$ (preserving the order). The prover also derives a proof that $\matr{A}'\in\mathcal{Q}_m$ re-randomizing proofs $\pi_{\mu,1},\ldots,\pi_{\mu,m}$ and shows that $h(\matr{A}_\mu)=h(\matr{A}')$ with a Groth-Sahai proof of the satisfiability of equation (\ref{eq:coll-h}).

Next, the prover computes $[\matr{C}']_2,[\matr{D}']_2\in\GG_2^{2\times m}$, whose first columns are, respectively, re-randomizations of $[\vecb{c}_{\mu,\nu}],[\vecb{d}_{\mu,\nu}]$ and the following are re-ran\-do\-mi\-za\-tions of $[\vecb{c}_{\mu,1}]_1,\ldots,[\vecb{c}_{\mu,\nu-1}]_1,\allowbreak[\vecb{c}_{\mu,\nu+1}]_1,\ldots,[\vecb{c}_{\mu,m}]$ and $[\vecb{d}_{\mu,1}]_1,\ldots,[\vecb{d}_{\mu,\nu-1}]_1,\allowbreak[\vecb{d}_{\mu,\nu+1}]_1,\allowbreak\ldots,[\vecb{d}_{\mu,m}]$ (preserving the order). It re-randomizes proofs $\theta_{\mu,\nu},\theta_{\mu,1},\ldots,\allowbreak\theta_{\mu,\nu-1},\theta_{\mu,\nu+1},\ldots,\theta_{\mu,m}$, showing that $[\matr{C}']_2$ opens to some $\vecb{x}$ such that $[\vecb{d}_i']$ opens to $g_{\vecb{a}'_i}(x'_i)$ and this $\sum_{i=0}^m[\vecb{d}'_i]_2$ opens to $g_{\matr{A}'}(\vecb{x})$.

and also computes a re-randomization of $[\vecb{g}_{\mu'}]$, where $\mu'=\mu$. The second set-membership proof shows that $[\vecb{g}_{\mu'}]\in G$ and that $\mu'=\mu$. Finally, the prover shows that $g_{\matr{A}'}(\vecb{x}')=g_{\matr{A_{\mu'}}}(\vecb{x}_{\mu'})$ with a proof of (\ref{eq:coll-g}).

Since $\matr{A}'\equiv_{\matr{U}}\matr{A}_\mu$, Lemma \ref{lemma:hg} implies that $\vecb{x}'$ is a permutation of $\vecb{x}_\mu$. Therefore, $[\vecb{c}_{\mu,\nu}]_1$, opens to $x_{\mu,\nu}=sk_\alpha$ which is in the ring of secret keys.