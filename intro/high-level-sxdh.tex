% !TEX root = ../main-ring-signature.tex


%We propose an alternative way of obtaining a $\sqrt{n}$ ring signature. We then combine both techniques, Chandran et al. and ours, and obtain a $\sqrt[3]{n}$ signature.
%
%
%Following Chandran et al.'s approach, if we want to obtain a $m:=\sqrt[3]{n}$ proof it is natural to arrange the verification keys in $m$ matrices of size $m\times m$ (a 3d array) as depcited bellow
%$$
%\begin{pmatrix}
%[vk_{1,1,1}] & \cdots & [vk_{1,1,m}]\\
%\vdots       & \ddots & \vdots      \\
%[vk_{1,m,1}] & \cdots & [vk_{1,m,m}]
%\end{pmatrix},
%\ldots,
%\begin{pmatrix}
%[vk_{m,1,1}] & \cdots & [vk_{m,1,m}]\\
%\vdots       & \ddots & \vdots      \\
%[vk_{m,m,1}] & \cdots & [vk_{m,m,m}]
%\end{pmatrix},
%$$
%where $vk_{i,j,k}:=vk_{(i-1)m^2+(j-1)m+k}$ for $i,j,k\in[m]$.
%
%The naive approach of selecting one of these matrices and then applying Chandran et al.'s approach will end up with a proof of size $n^{2/3}$. We follow the approach of Gonzalez et al.~\cite{AC:GonHevRaf15} which aggreates the $m$ matrices into a single one selecting $\vecb{a}_1,\ldots,\vecb{a}_n$ from some distribution such that the corresponding kernel problem is hard (using the terminology of Morillo et al.~\cite{AC:MorRafVil16}). That is, compute
%$$
%[\matr{V}] := \sum_{i=1}^{m} \vecb{a}_i
%\begin{pmatrix}
%[vk_{i,1,1}] & \cdots & [vk_{i,1,m}]\\
%\vdots       & \ddots & \vdots      \\
%[vk_{i,m,1}] & \cdots & [vk_{i,m,m}]
%\end{pmatrix}
%$$

Our scheme builds on top of the ring signature of Chandran et al.~and improves the underlying $\Theta(\sqrt{n})$ proof that the opening of a Groth-Sahai commitment is a Boneh-Boyen signature verification key $[vk]_2\in\GG_2$ and belongs to the ring of verification keys $R=\{[vk_1]_2,\ldots,[vk_n]_2\}$. In the rest of this section we simply refer to this proof as a ``set-membership proof'' and we remark that it might be applied to any set of group elements (not only of verification keys).

Our proof consists of two set-membership proofs in sets of size $n^{2/3}$ --- i.e.~each proof is of size $\Theta(\sqrt[3]{n})$ ---  plus $\Theta(\sqrt[3]{n})$ Groth-Sahai proofs and Groth-Sahai commitments.
We enlarge user's verification keys the by including $[\vecb{a}]_1$, where $\vecb{a}\gets\mathcal{Q}_1^\beta$ (i.e.~a commitment to $\beta$), and $[\vecb{b}]_2=\Com_{[\matr{V}]_2}(\beta;\rho)$, $[\vecb{c}]_2=\Com_{[\vecb{v}]_2}(sk;\rho)$ and $[\vecb{d}]_2 = \Com_{[\matr{V}]_2}(\beta sk;t)$, where $sk$ is the secret key of a Boneh-Boyen signature and $\beta=0$. We additionally compute proofs from equations (\ref{eq:Qm-memb-proofs}) and (\ref{eq:wi-proofs}) and we denote them by $\pi$ and $\theta$ respectively. Thereby, verification key of the $i$ th user is of the form $\vecb{vk}_i:=([vk_i]_2,[\vecb{a}_i]_1,[\vecb{b}_i]_2,[\vecb{c}_i]_2,[\vecb{d}_i]_2,\pi,\theta)$. Note that from commitments $[\vecb{a}_1]_1,\ldots,[\vecb{a}_n]_1$ one can derive a commtiment to $h(\matr{A})=h((\vecb{a}_1\cat\cdots\cat\vecb{a}_n))$ using the homomorphic properties of Groth-Sahai commitments. Similarly one can derive commitments to $[h_{\matr{A}}(\vecb{sk})]_2 = [\sum_{i=0}^m\beta_i sk_i]$ from $[\vecb{d}_1]_2,\ldots,[\vecb{d}_n]_2$. Note also that $\pi_1,\ldots,\pi_n$ is a proof that $\matr{A}\in\mathcal{Q}_n$.

In spite of all the differences with Chandran et al.'s proof, we also show that the opening of a Groth-Sahai commitment is a Boneh-Boyen verification key $[vk_i]_2$ and belongs to $\{[vk_1]_2,\ldots,[vk_n]_2\}$.

Our first step is to arrange the verification keys in $n^{2/3}$ blocks of size $m=\sqrt[3]{n}$. To do so we use the following notation: for a sequence $\{s\}_{1\leq i \leq n}$ define $s_{\mu,\nu}:=s_{(\mu-1)m+\nu}$, where  $1\leq\mu\leq n^{2/3},1\leq \nu\leq m$.  The prover and the verifier arrange the elements of the verification keys into $[\vecb{\kappa}_1]_2,\ldots, [\vecb{\kappa}_{n^{2/3}}]_2$, and $[\matr{A}_1]_1,\ldots, [\matr{A}_{n^{2/3}}]_1$ such that $[\vecb{\kappa}_i]_2:= ([vk_{i,1}]_2,\ldots,[vk_{i,m}]_2)^\top\in\GG^m_2$,  and $[\matr{A}_i]_1=[\vecb{a}_{i,1}\cat\cdots\cat\vecb{a}_{i,m}]_1\in\GG^{2\times m}_1$. They also define the sets
\begin{align*}
&H:=\{[h(\matr{A}_1)]_1,\ldots,\allowbreak [h(\matr{A}_{n^{2/3}})]_1\},\\
&G:=\{[\vecb{g}_1]_2 = \Com(g_{\matr{A}_1}(\vecb{\kappa}_1)),\ldots,[\vecb{g}_{n^{2/3}}]_2 = \Com(g_{\matr{A}_{n^{2/3}}}(\vecb{\kappa}_{n^{2/3}}))\},
\end{align*}
where $[\vecb{g}_i]_2$ is computed as (recall the $sk$ is the discrete logarithm of $[vk]_2$, i.e.~$sk=vk$)
$$\sum_{j=1}^m [\vecb{d}_{i,j}]=\allowbreak\Com_{[\matr{V}]_2}\left(\sum_{j=1}^m\beta_isk_i;\sum_{j=1}^m t_i\right)=\allowbreak \Com_{[\matr{V}]_2}(g_{\matr{A}_i}(\vecb{\kappa}_i);\tilde{t}).$$

The prover starts computing Groth-Sahai commitments to $sk_\alpha=sk_{\mu,\nu}$, and to $[h(\matr{A}_\mu)]_1$ using homomorphic properties of Groth-Sahai commitments, for some $1\leq \alpha \leq n$, and computes the first set-membership proof showing that $[h(\matr{A}_\mu)]_1\in H$.
The prover compute a matrix $\matr{A}'$ such that the first committed element is $[\vecb{a}_\alpha]_1+\gamma_\alpha[\vecb{u}_2]_1$, $\gamma_\alpha\gets\Z_q$ (i.e.~ a re-randomization of $[\vecb{a}_\alpha]_1$), and the ones that follow are re-randomizations of the other columns of $[\matr{A}_\mu]_1$ (preserving the order), with the expetion of the las column which is computed as $[\vecb{a}'_m]_1 = [h(\matr{A}_\mu)]_1-\sum_{i=0}^{m-1}[\vecb{a}'_i]_1$. The prover also derives a proof that $\matr{A}'\in\mathcal{Q}_m$ from proofs $\pi_{\mu,1},\ldots,\pi_{\mu,m}$.
Clearly, $h(\matr{A}_\mu)=h(\matr{A}')$ and hence, there exits a permutation matrix $\matr{P}$ and $\vecb{\gamma}\in\Z_q^m$ such that $\matr{A}' = \matr{A}_\mu\matr{P}+\vecb{u}_2\vecb{\gamma}^\top$.

Next, the prover computes Groth-Sahai commitments to each element of the vector $[\vecb{\kappa}']_2$, whose first element is $[vk_\alpha]_2$ and the rest are the other verification keys in $[\vecb{\kappa}_\mu]_2$ (preserving the order), and commits also to $g_{[\matr{A_{\mu'}}]_1}([\vecb{\kappa}_{\mu'}]_2)$, where $\mu'=\mu$. The second set-membership proof shows that $g_{[\matr{A_{\mu'}}]_1}([\vecb{\kappa}_{\mu'}]_2)\in G$ and that $\mu'=\mu$. Finally, the prover gives a Groth-Sahai proof that $g_{[\matr{A}']_1}([\vecb{\kappa}']_2)=g_{[\matr{A_{\mu'}}]_1}([\vecb{\kappa}_{\mu'}]_2)$.

Since $A'=A_\mu$, Lemma \ref{lemma:hg} implies that $[\vecb{\kappa}']_1$ is a permutation of $[\vecb{\kappa}_\mu]_1$ and thus $[\kappa'_1]_2=[vk_\alpha]_2$ is in the ring of verification keys.