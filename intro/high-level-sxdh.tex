% !TEX root = ../main-ring-signature.tex


%We propose an alternative way of obtaining a $\sqrt{n}$ ring signature. We then combine both techniques, Chandran et al. and ours, and obtain a $\sqrt[3]{n}$ signature.
%
%
%Following Chandran et al.'s approach, if we want to obtain a $m:=\sqrt[3]{n}$ proof it is natural to arrange the verification keys in $m$ matrices of size $m\times m$ (a 3d array) as depcited bellow
%$$
%\begin{pmatrix}
%[vk_{1,1,1}] & \cdots & [vk_{1,1,m}]\\
%\vdots       & \ddots & \vdots      \\
%[vk_{1,m,1}] & \cdots & [vk_{1,m,m}]
%\end{pmatrix},
%\ldots,
%\begin{pmatrix}
%[vk_{m,1,1}] & \cdots & [vk_{m,1,m}]\\
%\vdots       & \ddots & \vdots      \\
%[vk_{m,m,1}] & \cdots & [vk_{m,m,m}]
%\end{pmatrix},
%$$
%where $vk_{i,j,k}:=vk_{(i-1)m^2+(j-1)m+k}$ for $i,j,k\in[m]$.
%
%The naive approach of selecting one of these matrices and then applying Chandran et al.'s approach will end up with a proof of size $n^{2/3}$. We follow the approach of Gonzalez et al.~\cite{AC:GonHevRaf15} which aggreates the $m$ matrices into a single one selecting $\vecb{a}_1,\ldots,\vecb{a}_n$ from some distribution such that the corresponding kernel problem is hard (using the terminology of Morillo et al.~\cite{AC:MorRafVil16}). That is, compute
%$$
%[\matr{V}] := \sum_{i=1}^{m} \vecb{a}_i
%\begin{pmatrix}
%[vk_{i,1,1}] & \cdots & [vk_{i,1,m}]\\
%\vdots       & \ddots & \vdots      \\
%[vk_{i,m,1}] & \cdots & [vk_{i,m,m}]
%\end{pmatrix}
%$$


%Our scheme builds on top of the ring signature of Chandran et al.~and improves the underlying $\Theta(\sqrt{n})$ proof that the opening of a Groth-Sahai commitment is a Boneh-Boyen signature verification key $vk$ and belongs to the ring of verification keys $R=\{vk_1\ldots,vk_n\}$. In the rest of this section we simply refer to this proof as a ``set-membership proof'' and we remark that it might be applied to any set of (vectors of) group elements (not only of verification keys).
%
Our construction follow the high-level description depicted in section \ref{sec:tech-overview} with the only difference that we do not use the verification key of the Boneh-Boyen signature, but a commitment to the secret key. The only reason is efficiency since in this way we use Groth-Sahai proofs for integer equations instead of equations involving group elements. 
Each signature consists of two set-membership proofs in sets of size $n^{2/3}$ --- i.e.~each proof is of size $\Theta(\sqrt[3]{n})$ ---  plus $\Theta(\sqrt[3]{n})$ Groth-Sahai proofs and Groth-Sahai commitments.

%Let $(x,vk)$ be Boneh-Boyen secret/verification keys (note that $[x]_2=vk$) and let $[\matr{U}]_1,[\matr{V}]_2,[\matr{W}]_1$ Groth-Sahai commitment keys.
%We consider extended verification keys which includes $[\vecb{a}]_1 = \Com_{[\matr{U}]_1}(\beta;r)$, where $\beta=0$ and $r\gets\Z_q$, $\pi$ a Groth-Sahai proof that $\beta(\beta-1)=0$ (see App.~\ref{sec:GSproofs-h}), 
%plus $[\vecb{c}]_2,[\vecb{d}]_1$ and $[\vecb{\psi}]_2,[\vecb{\omega}]_1$ defined as follows.
%Recall that $[\vecb{a}]_1$, where $\vecb{a}\gets\mathcal{Q}_1^\beta$, is a commitment to $\beta$, as well as $[\vecb{b}]_2=\Com_{[\matr{V}]_2}(\beta;\rho)$.
%We compute $[\vecb{c}]_2=\Com_{[\matr{W}]_2}(x;s)$, $[\vecb{d}]_1 = x[\vecb{a}]_1+t[\vecb{u}_2]_1=\Com_{[\matr{U}]_1}(y)$, for $y:=\beta x$, and $[\vecb{\psi}]_2,[\vecb{\omega}]_1$ is a Groth-Sahai proof that $\beta x = y$ (see App. \ref{sec:GSproofs-g} for more details). Thereby,  the extended verification key of each user is of the form $\widetilde{vk}:=([x]_2,[\vecb{a}]_1,[\vecb{c}]_2,[\vecb{d}]_1,\pi,[\vecb{\psi}]_2,[\vecb{\omega}]_1)$.

For $[\matr{A}]_1\in\GG_1^{2\times m}$ , a matrix whose colum $[\vecb{a}_i]$ is a Groth-Sahai commitment to $\beta_i$, and $\vecb{x}\in\Z_q^m$ we define $h([\matr{A}]_1) := \sum_{i=1}^m \beta_i$ and $g_\matr{A}(\vecb{x}) = \sum_{i=1}^m \beta_i x_i$. Unlike the PPA-based construction, we do not prove collision resistance of $h$ or $g$ (they are not) but instead these functions are only used as an intuitive link with the simpler PPA-based construction.

The high level description of our ring signature in the SXDH setting was already given in section \ref{sec:tech-overview}. It was left to show how to re-randomize of proofs for the equations $\beta\in\bits$ and $y = \beta x$,  which can be done with the techniques of \cite{C:BCCKLS09} and for completeness we give in App.~\ref{sec:GSproofs-h} and App.~\ref{sec:GSproofs-g}, respectively. It was also left to show how to derive a proof that $g_{\matr{A}'}(\vecb{x}') = g_{\matr{A}_\mu}(\vecb{x}_\mu)$ and we do it in the following section.

\subsection{NIZK proof that $g_{\matr{A}'}(\vecb{x}') = g_{\matr{A}}(\vecb{x})$}
Let $[\matr{U}]_1$ and $[\matr{W}]_2$ Groth-Sahai commitment keys. Consider $[\vecb{a}_i]_1 = \Com(\beta_i;r_i)$, $[\vecb{c}_i]_2 = \Com_{[\matr{W}]_2}(x_i;s)$, and $[\vecb{d}_i] = \Com_{[\matr{U}]_1}(y_i;t)$, where $y_i=\beta_ix_i$, $\beta\in\bits$, $r,s,t\in\Z_q$, and $1\leq i\leq m$. Consider also $[\vecb{g}]_1$, a re-randomization of $\sum_{i=1}^m [\vecb{d}_i] = \Com(g_\matr{A}(\vecb{x}))$, and $[\matr{A}']_1$ and $[\matr{C}']_2$ re-randomizations of permutations of $[\matr{A}]_1 := ([\vecb{a}_1]|\cdots|[\vecb{a}_m])$ and $[\matr{C}]_2 := ([\vecb{c}_1]_2|\cdots|[\vecb{c}_m]_2)$, respectively. We want to construct a proof that $g_{\matr{A}'}(\vecb{x}') = g_\matr{A}(\vecb{x})$, or equivalently $\sum_{i=1}^m \beta'_ix'_i = \sum_{i=1}^m \beta_i x_i$, only from the extended verification keys and the random coins used in the re-randomizations.

Apart from $[\vecb{a}_i]_1,[\vecb{c}_i]_2,[\vecb{d}_i]_1$, the extended verification key contains Groth-Sahai proofs $[\vecb{\psi}_i]_2,[\vecb{\omega}_i]_1$ for the equation $\beta_ix_i = y_i$ defined in App.~\ref{sec:GSproofs-g}. Each of these proofs satisfy the verification equation
$$
[\vecb{a}_i]_1[\vecb{c}_i^\top]_2 - [\vecb{d}_i]_1[\vecb{w}_1^\top]_2 = [\vecb{u}_2]_1[\vecb{\psi}^\top_i]_2 + [\vecb{\omega}_i]_1[\vecb{w}_2^\top]_2.
$$

$[\matr{A}']_1$, $[\matr{C}']_2$ and $[\vecb{g}]_1$ are computed as $[\matr{A}']_1 = [\matr{A}]_1\matr{P}+[\vecb{u}_2]_1\vecb{\delta}_a^\top$, $[\matr{C}']_2 = [\matr{C}]_2\matr{P}+[\vecb{w}_2]_2\vecb{\delta}_c^\top$, and $[\vecb{g}]_1 = \sum_{i=1}^m [\vecb{d}_i]_1 + [\vecb{u}_2]_1\delta_g$, where $\matr{P}$ is a permutation matrix and $\vecb{\delta}_a,\vecb{\delta}_c\in\Z_q^m$ and $\delta_g\in\Z_q$.
The right side of the verification equation for equation $\sum_{i=1}^m\beta'_i x'_i - y = 0$, where $y=\sum_{i=1}^n \beta_ix_i$ is the opening of $[\vecb{d}']_1$ and $\vecb{\beta}',\vecb{x}'$ are the openings of $[\matr{A}']_1$ and $[\matr{C}']_2$ respectively, is equal to
\begin{align*}
&[\matr{A}']_1[{\matr{C}'}^\top]_2 - [\vecb{d}']_1[\vecb{w}_1^\top]_2\\
&= [\matr{A}]_1\matr{P}\matr{P}^\top[\matr{C}^\top]_2 + [\matr{A}]_1\matr{P}\vecb{\delta}_c[\vecb{w}_2^\top]_2 + [\vecb{u}_2]_1\vecb{\delta}_a^\top[\matr{C}'^\top]_2-[\vecb{d}']_1[\vecb{w}_2^\top]_2\\
&=\sum_{i=1}^m ([\vecb{a}_i]_1[\vecb{c}_i^\top]_2-[\vecb{d}_i]_1[\vecb{w}_1^\top]) +[\matr{A}]_1\matr{P}\vecb{\delta}_c[\vecb{w}_2^\top]_2 + [\vecb{u}_2]_1(\vecb{\delta}_a^\top[\matr{C}'^\top]_2-\delta_g[\vecb{w}_1^\top]_2)\\
&= [\vecb{u}_2]_1\left(\sum_{i=1}^m[\vecb{\psi}_i]_1+[\matr{C}']_2\vecb{\delta}_a-\delta_g[\vecb{w}_1]_2\right)^\top + \left(\sum_{i=1}^m[\vecb{\omega}_i]_1+[\matr{A}]_1\matr{P}\vecb{\delta}_c\right)[\vecb{w}_2^\top]_2.
\end{align*}

The last equation indicates that the proof must be the factors of $[\vecb{u}_2]_1$ and $[\vecb{w}_2^\top]_2$ plus randomization terms. That is, for $\xi\gets\Z_q$
\begin{align}
&[\vecb{\psi}']_2 = \sum_{i=1}^m[\vecb{\psi}_i]_1+[\matr{C}']_2\vecb{\delta}_a-\delta_g[\vecb{w}_1]_2 + \xi[\vecb{w}_2]_2\nonumber\\
&
[\vecb{\omega}']_1 = \sum_{i=1}^m[\vecb{\omega}_i]_1+[\matr{A}]_1\matr{P}\vecb{\delta}_c - \xi[\vecb{u}_2]_1.\label{eq:rerand-proofs-g}
\end{align}

Assuming $[\vecb{d}']_1$ is correctly computed, the proof is sound because it satisfy the Groth-Sahai verification equation. Furthermore, the proof is uniformly distributed conditioned on satisfying the verification equation and thus follows exactly the same distribution as a fresh Groth-Sahai proof.