% !TEX root = ../main-ring-signature.tex

\begin{table}[h]
\begin{center}
\begin{minipage}{\textwidth}
\begin{center}
%\begin{scriptsize}
\begin{tabular}{l|l|l}
%\hline
                                           & Chandran et al.~\cite{ICALP:ChaGroSah07} & This work \\
\hline%\hline
\rule{0pt}{2.5ex}CRS size  $\GG_1/\GG_2$              & 4/4                                      & $4/4$       \\
\rule{0pt}{2.5ex}Verification key size $\GG_1/\GG_2$    & $1/0$                                       & $2/5$       \\
\rule{0pt}{2.5ex}Signature size      $\GG_1/\GG_2$      & $12\sqrt{n}+10/15\sqrt{n}+8$                        & $24\sqrt[3]{n} + 36/34\sqrt[3]{n} + 24$\\
\rule{0pt}{2.5ex}Signature generation time & $37\sqrt{n}+23$                        & $80\sqrt[3]{n}+71$\\
\rule{0pt}{2.5ex}Verification time         & $2n + 60\sqrt{n}+38$                & $8n^{2/3} + 162\sqrt[3]{n} + 118$\\
%\hline 
\end{tabular}
%\end{scriptsize}
\end{center}
\caption{Comparison of Chandran et al.'s ring signature and ours for a ring of size $n$. `Signature generation time' is measured in number of exponentiations, `Verification time' is measured in number of pairings, and all other rows are measured in number of group elements. A quick calculation shows that our signature outperforms Chandran et al.'s (roughly) for $n>180$.\label{table:eff}}
\end{minipage}
\end{center}
\end{table}
