Ring signatures, introduced by Rivest, Shamir and Tauman, \cite{AC:RivShaTau01}, allow to anonymously sign a message on behalf of a \emph{ring} of users $P_1,\ldots,P_n$, only if the signer belongs to the ring. Although there are other cryptographic schemes that provides similar guarantees (e.g.~group signatures \cite{EC:ChaVan91}), ring signatures are not coordinated: each user generates secret/public keys on his own -- i.e.~no central authorities -- and might sign on behalf of a ring without the approval or assistance of the other members.

While there exist even logarithmic size solutions \cite{EC:GroKoh15,EC:LLNW16}, most of them rely on the {random oracle model} (ROM). The ROM idealizes the behavior of hash functions and proofs of security in the ROM model are considered only heuristic arguments, since there are protocols secure in the ROM but insecure using any real hash functions \cite{STOC:CanGolHal98}. Without random oracles all the constructions have signatures of size linear in the size of the ring, being the the sole exception the $\Theta(\sqrt{n})$ ring signature of Chandran et al.~\cite{ICALP:ChaGroSah07}. 
We remark that no asymptotic improvements to Chandran et al.'s construction have been made since their introduction (only improvements in the constants by R\`afols \cite{TCC:Rafols15} and by Gonz\'alez et al.~\ref{sec:bits-applications}). Although some previous works claim to construct signatures of constant \cite{ACISP:BosDasRan15} or logarithmic \cite{IET:GriSusPla16} size, they are either in a weaker security model or we can identify a flaw in the construction (see Section \ref{sec:rs-flawed}). 

In this work we present the first ring signature whose signature size is asymptotically smaller than Chandran et al.'s. Specifically, our ring signature is of size $\Theta(\sqrt[3]{n})$. Interestingly, the security of our construction relies on a security assumption -- the {permutation pairing assumption} -- introduced by Groth and Lu \cite{AC:GroLu07} in an unrelated setting: proofs of correctness of a shuffle. While the assumption is ``non-standard'', in the sense that is not a ``DDH like'' assumption, it is a falsifiable assumption and it was proven to be generically hard by Groth and Lu. For simplicity, we work on symmetric groups ($\GG_1=\GG_2$), but our techniques should be easily extended to asymmetric groups if a natural translation to asymmetric groups of the Groth and Lu's assumption is given.

