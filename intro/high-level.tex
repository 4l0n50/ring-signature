% !TEX root = ../main-ring-signature.tex


%We propose an alternative way of obtaining a $\sqrt{n}$ ring signature. We then combine both techniques, Chandran et al. and ours, and obtain a $\sqrt[3]{n}$ signature.
%
%
%Following Chandran et al.'s approach, if we want to obtain a $m:=\sqrt[3]{n}$ proof it is natural to arrange the verification keys in $m$ matrices of size $m\times m$ (a 3d array) as depcited bellow
%$$
%\begin{pmatrix}
%[vk_{1,1,1}] & \cdots & [vk_{1,1,m}]\\
%\vdots       & \ddots & \vdots      \\
%[vk_{1,m,1}] & \cdots & [vk_{1,m,m}]
%\end{pmatrix},
%\ldots,
%\begin{pmatrix}
%[vk_{m,1,1}] & \cdots & [vk_{m,1,m}]\\
%\vdots       & \ddots & \vdots      \\
%[vk_{m,m,1}] & \cdots & [vk_{m,m,m}]
%\end{pmatrix},
%$$
%where $vk_{i,j,k}:=vk_{(i-1)m^2+(j-1)m+k}$ for $i,j,k\in[m]$.
%
%The naive approach of selecting one of these matrices and then applying Chandran et al.'s approach will end up with a proof of size $n^{2/3}$. We follow the approach of Gonzalez et al.~\cite{AC:GonHevRaf15} which aggreates the $m$ matrices into a single one selecting $\vecb{a}_1,\ldots,\vecb{a}_n$ from some distribution such that the corresponding kernel problem is hard (using the terminology of Morillo et al.~\cite{AC:MorRafVil16}). That is, compute
%$$
%[\matr{V}] := \sum_{i=1}^{m} \vecb{a}_i
%\begin{pmatrix}
%[vk_{i,1,1}] & \cdots & [vk_{i,1,m}]\\
%\vdots       & \ddots & \vdots      \\
%[vk_{i,m,1}] & \cdots & [vk_{i,m,m}]
%\end{pmatrix}
%$$

Our scheme builds on top of the ring signature of Chandran et al.~and improves the underlying $\Theta(\sqrt{n})$ proof that the opening of a Groth-Sahai commitment is a Boneh-Boyen signature verification key $[vk]_2\in\GG_2$ and belongs to the ring of verification keys $R=\{[vk_1]_2,\ldots,[vk_n]_2\}$. In the rest of this section we simply refer to this proof as a ``set-membership proof'' and we remark that it might be applied to any set of group elements (not only of verification keys).

Our proof consists of two set-membership proofs in sets of size $n^{2/3}$ --- i.e.~each proof is of size $\Theta(\sqrt[3]{n})$ ---  plus $\Theta(\sqrt[3]{n})$ Groth-Sahai proofs and Groth-Sahai commitments.
We enlarge user's verification keys the by including $[\vecb{a}]_1,[\vecb{a}]_2$, where $\vecb{a}\gets\mathcal{Q}$, and $[\vecb{a}vk]_2=\vecb{a}[vk]_2$, where $[vk]_2$ is the verification key of a Boneh-Boyen signature. Thereby, verification key of the $i$ th user is of the form $\vecb{vk}_i:=([vk_i]_2,[\vecb{a}]_1,[\vecb{a}]_2,[\vecb{a}_ivk_i]_2)$. In spite of these differences with Chandran et al.'s verification key, our proof also shows that the opening of a Groth-Sahai commitment is a Boneh-Boyen verification key $[vk_i]_2$ and belongs to $\{[vk_1]_2,\ldots,[vk_n]_2\}$.

Our first step is to arrange the verification keys in $n^{2/3}$ blocks of size $m=\sqrt[3]{n}$. To do so we coin the following notation: for a sequence $\{s\}_{1\leq i \leq n}$ define $s_{\mu,\nu}:=s_{(\mu-1)m+\nu}$, where  $1\leq\mu\leq n^{2/3},1\leq \nu\leq m$.  The prover and the verifier arrange the elements of the verification keys into $[\vecb{\kappa}_1]_2,\ldots, [\vecb{\kappa}_{n^{2/3}}]_2$, $A_1,\ldots, A_{n^{2/3}}$, and $[\matr{A}_1]_1,\ldots, [\matr{A}_{n^{2/3}}]_1$ such that $[\vecb{\kappa}_i]_2:= ([vk_{i,1}]_2,\ldots,[vk_{i,m}]_2)^\top\in\GG^m_2, A_i:=\{[\vecb{a}_{i,1}]_1,\ldots,[\vecb{a}_{i,m}]_1\}\in Q_m$ and $[\matr{A}_i]_1=[\vecb{a}_{i,1}\cat\cdots\cat\vecb{a}_{i,m}]_1\in\GG^{2\times m}_1$. They also define the sets
\begin{align*}
&H:=\{h(A_1),\ldots,\allowbreak h(A_{n^{2/3}})\},\quad G:=\{g_{[\matr{A}_1]_1}([\vecb{\kappa}_1]_2),\allowbreak\ldots,g_{[\matr{A}_{n^{2/3}}]_1}([\vecb{\kappa}_{n^{2/3}}]_2)\},
\end{align*}
where $g_{[\matr{A}_i]_1}([\vecb{\kappa}_i]_2)$ is computed as $[\matr{A}_i\vecb{\kappa}_i]_2=\sum_{j=1}^m [\vecb{a}_{i,j}vk_{i,j}]_2$.

The prover starts computing Groth-Sahai commitments to $[vk_\alpha]_2=[vk_{\mu,\nu}]_2$ and to $h(A_\mu)$, for some $1\leq \alpha \leq n$, and computes the first set-membership proof showing that $h(A_\mu)\in H$.
The prover commits to each element of $A'=A_\mu$ such that the first committed element is $[\vecb{a}_\alpha]_1$ and the ones that follow are the other columns of $[\matr{A}_\mu]_1$ (preserving the order). Denote by $[\matr{A}']_1$ the matrix whose columns are the elements of $A'$ in the order defined before.  The prover shows using Groth-Sahai proofs that $A'\in Q_m$ and that $h(A')=h(A_\mu)$. From this part of the proof we know that, with all but negligible probability, $A'=A_\mu$.

Next, the prover computes Groth-Sahai commitments to each element of the vector $[\vecb{\kappa}']_2$, whose first element is $[vk_\alpha]_2$ and the rest are the other verification keys in $[\vecb{\kappa}_\mu]_2$ (preserving the order), and commits also to $g_{[\matr{A_{\mu'}}]_1}([\vecb{\kappa}_{\mu'}]_2)$, where $\mu'=\mu$. The second set-membership proof shows that $g_{[\matr{A_{\mu'}}]_1}([\vecb{\kappa}_{\mu'}]_2)\in G$ and that $\mu'=\mu$. Finally, the prover gives a Groth-Sahai proof that $g_{[\matr{A}']_1}([\vecb{\kappa}']_2)=g_{[\matr{A_{\mu'}}]_1}([\vecb{\kappa}_{\mu'}]_2)$.

Since $A'=A_\mu$, Lemma \ref{lemma:hg} implies that $[\vecb{\kappa}']_1$ is a permutation of $[\vecb{\kappa}_\mu]_1$ and thus $[\kappa'_1]_2=[vk_\alpha]_2$ is in the ring of verification keys.