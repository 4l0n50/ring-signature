%We propose an alternative way of obtaining a $\sqrt{n}$ ring signature. We then combine both techniques, Chandran et al. and ours, and obtain a $\sqrt[3]{n}$ signature.
%
%
%Following Chandran et al.'s approach, if we want to obtain a $m:=\sqrt[3]{n}$ proof it is natural to arrange the verification keys in $m$ matrices of size $m\times m$ (a 3d array) as depcited bellow
%$$
%\begin{pmatrix}
%[vk_{1,1,1}] & \cdots & [vk_{1,1,m}]\\
%\vdots       & \ddots & \vdots      \\
%[vk_{1,m,1}] & \cdots & [vk_{1,m,m}]
%\end{pmatrix},
%\ldots,
%\begin{pmatrix}
%[vk_{m,1,1}] & \cdots & [vk_{m,1,m}]\\
%\vdots       & \ddots & \vdots      \\
%[vk_{m,m,1}] & \cdots & [vk_{m,m,m}]
%\end{pmatrix},
%$$
%where $vk_{i,j,k}:=vk_{(i-1)m^2+(j-1)m+k}$ for $i,j,k\in[m]$.
%
%The naive approach of selecting one of this matrices and then applying Chandran et al.'s approach will end up with a proof of size $n^{2/3}$. We follow the approach of Gonzalez et al.~\cite{AC:GonHevRaf15} which aggreates the $m$ matrices into a single one selecting $\vecb{a}_1,\ldots,\vecb{a}_n$ from some distribution such that the corresponding kernel problem is hard (using the terminology of Morillo et al.~\cite{AC:MorRafVil16}). That is, compute
%$$
%[\matr{V}] := \sum_{i=1}^{m} \vecb{a}_i
%\begin{pmatrix}
%[vk_{i,1,1}] & \cdots & [vk_{i,1,m}]\\
%\vdots       & \ddots & \vdots      \\
%[vk_{i,m,1}] & \cdots & [vk_{i,m,m}]
%\end{pmatrix}
%$$

In our scheme the secret/verification keys of party $P$ is $(sk,\vecb{vk})$, where $\vecb{vk}=([vk],[\vecb{a}],\vecb{a}[vk])$, $(sk,[vk])$ are secret/verification keys of the Boneh-Boyen signature scheme, and $\vecb{a}\in\Z_q^2$ is chosen independently for each key from some distribution $\mathcal{Q}$ to be specified later. Suppose that $\vecb{vk}$ is the $\alpha$ th element in the ring $R=\{\vecb{vk}_{1},\ldots,\vecb{vk}_{n}\}$ and let $1\leq i_\alpha,j_\alpha,k_\alpha \leq m$ such that $\vecb{vk}_\alpha = \vecb{vk}_{i_\alpha,j_\alpha,k_\alpha}$, where $\vecb{vk}_{i,j,k}=\vecb{vk}_{(i-1)m^2+(j-1)m+k}$ for $1\leq  i,j,k\leq m$ and $m:=\sqrt[3]{n}$. Below we describe a $\Theta(\sqrt[3]{n})$ set-mebership-proof in $R$.

Consider the sets
\begin{align*}
&S:=\{[\vecb{s}_1],\ldots,[\vecb{s}_{n^{2/3}}]\}:=\left\{\sum_{i=1}^{m}[\vecb{a}_{i,1,1}],\ldots,\sum_{i=1}^{m}[\vecb{a}_{i,m,m}]\right\}\text{ and }\\
&S':=\{[\vecb{s}'_1],\ldots,[\vecb{s}'_{n^{2/3}}]\}:=\left\{\sum_{i=1}^{m}\vecb{a}_{i,1,1}[vk_{i,1,1}],\ldots,\sum_{i=1}^{m}\vecb{a}_{i,m,m}[vk_{i,m,m}]\right\}.
\end{align*}

The prover commits to $[\vecb{x}]=[\vecb{s}_\mu]$ and $[\vecb{y}]=[\vecb{s}'_{\mu'}]$, for $\mu=\mu'=(j_\alpha-1)m+k_\alpha$, and shows, using (twice) the set-membership proof of Chandran et al., that $[\vecb{x}]\in S$ and that $[\vecb{y}]\in S'$.
The prover also needs to assure that $\mu=\mu'$, which can be done reutilizing the commitment to $\mu$ (in fact to its representation as two unitary vectors of size $m$) used in the proof that $[\vecb{x}]\in S$ and in the proof that $[\vecb{y}]\in S'$. Since both sets are of size $n^{2/3}$, the two set membership proofs are of size $\Theta(\sqrt[3]{n})$.
 
Now that the prover has committed to elements $[\vecb{x}]=\sum_{i=1}^{m}[\vecb{a}_{i,j_\alpha,k_\alpha}]$ and $[\vecb{y}]=\sum_{i=1}^{m}\vecb{a}_{i,j_\alpha,k_\alpha}[vk_{i,j_\alpha,k_\alpha}]$, it additionally commits to $[\kappa_1]:=[vk_{1,j_\alpha,k_\alpha}],\allowbreak\ldots,\allowbreak[\kappa_m]:=[vk_{m,j_\alpha,k_\alpha}]$ and $[\vecb{z}_1]:=[\vecb{a}_{1,j_\alpha,k_\alpha}],\ldots,[\vecb{z}_m]:=[\vecb{a}_{m,j_\alpha,k_\alpha}]$. The prover now gives a proof that
\begin{equation}
\sum_{i=1}^{m}[\vecb{z}_i][\kappa_i]=[\vecb{y}][1]. \label{eq:verif1}
\end{equation}

Assume for a while that $\vecb{z}_1,\ldots,\vecb{z}_m$ is a permutation of $\vecb{a}_{1,j_\alpha,k_\alpha},\ldots,\vecb{a}_{m,j_\alpha,k_\alpha}$, that is $\vecb{z}_i=\vecb{a}_{\pi(i),j_\alpha,k_\alpha}$, $1\leq i\leq m$, for some permutation $\pi\in S_m$. Therefore, equation (\ref{eq:verif1}) implies that
\begin{align*}
\sum_{i=1}^{m}[\vecb{z}_i][\kappa_i]&=\sum_{i=1}^{m}[\vecb{a}_{\pi(i),j_\alpha,k_\alpha}][\kappa_i]=\sum_{i=1}^{m}[\vecb{a}_{i,j_\alpha,k_\alpha}][\kappa_{\pi^{-1}(i)}]\\
&=\sum_{i=1}^{m}[\vecb{a}_{i,j_\alpha,k_\alpha}][vk_{i,j_\alpha,k_\alpha}].
\end{align*}
Then $\kappa_1,\ldots,\kappa_m$ is a permutation of $vk_{1,j_\alpha,k_\alpha},\ldots,vk_{m,j_\alpha,k_\alpha}$ (the same defined by $\vecb{z}_1,\ldots,\vecb{z}_m$), unless $(\kappa_{\pi^{-1}(1)}-{vk_{1,j_\alpha,k_\alpha}),\ldots,\kappa_{\pi^{-1}(m)}-vk_{m,j_\alpha,k_\alpha})})^\top$ is in the kernel of $\matr{A}$. However, this is in general a hard problem and in fact corresponds a to kernel matrix Diffie-Hellman assumption, in the terminology of Morillo et al.~\cite{AC:MorRafVil16}.
For the distribution $\mathcal{Q}$ (defined later), Groth and Lu showed the hardness of this problem in the generic group model \cite{AC:GroLu07}.

Finally, the prover  shows, using Chandran et al.'s set-membership proof, that $[vk_\alpha]\in\{[\kappa_1],\ldots,[\kappa_m]\}$ which implies that  $[vk_\alpha]=[vk_{i_\alpha,j_\alpha,k_\alpha}]$.

It is only left the to show that $\vecb{z}_1,\ldots,\vecb{z}_m$ is a permutation of $\vecb{a}_{1,j_\alpha,k_\alpha},\ldots,\allowbreak\vecb{a}_{m,j_\alpha,k_\alpha}$. To do so we will use the following assumption introduced by Groth and Lu \cite{AC:GroLu07}.
\begin{definition}[Permutation Pairing Assumption]\label{def:ppa}
Let $\mathcal{Q}_{m}=\underbrace{\mathcal{Q}\cat\ldots\cat\mathcal{Q}}_{m\text{ times}}$, where concatenation of  distributions is defined in the natural way and 
$$\mathcal{Q}: \vecb{a}=\pmatri{x\\x^2},\quad x\gets\Z_q.$$
We say that the $m$-permutation pairing assumption holds relative to $\G_s$ if for any adversary $\advA$
$$
\Pr\left[
\begin{array}{l}
gk\gets\G_s(1^\lambda);\matr{A}\gets\mathcal{Q}_{m};[\matr{Z}]\gets\advA(gk,[\matr{A}]):\\
\mathrm{(i)} \sum_{i=1}^{m}[\vecb{z}_i]=\sum_{i=1}^{m}[\vecb{a}_i], \mathrm{(ii)}\ \forall 1\leq i\leq m\ [z_{2,i}][1]=[z_{1,i}][z_{1,i}],\\
\text{ and }\matr{Z}\text{ is not a permutation of the columns of }\matr{A}
\end{array}
\right],
$$
where $[\matr{Z}]=[(\vecb{z}_1,\ldots,\vecb{z}_m)], [\matr{A}]=[(\vecb{a}_1,\ldots,\vecb{a}_m)]\in\GG^{2\times m}$,
is negligible in $\lambda$.
\end{definition}

If the prover additionally proves that equations (i) and (ii) from definition \ref{def:ppa} are satisfied for $\matr{A}:=(\vecb{a}_{1,j_\alpha,k_\alpha},\ldots,\vecb{a}_{m,j_\alpha,k_\alpha})$, which can be done with $\Theta(m)$ group elements using Groth-Sahai proofs, the assumption is guaranteeing that the columns of $\matr{Z}$ are a permutation of the columns of $\matr{A}$.

