% !TEX root = ../main-ring-signature.tex

\subsubsection{Extending our technique.}
A natural question is if this technique can be applied once again. That is, to compute a $\Theta(\sqrt[4]{n})$  proof, compute commitments to an element from $H=\{h(A_1),\ldots,h(A_{n^{3/4}})\}$
%and
%$G=\allowbreak\{
%	g_{[\matr{A}_1]}
%		([
%			\vecb{\kappa}_1]),
%	\ldots,\allowbreak
%	g_{[\matr{A}_{n^3/4}]}
%		([
%			\vecb{\kappa}_{n^{3/4}}
%])\}$,
and then prove that it belongs to $H$ using our set membership proof of size $\Theta(\sqrt[3]{n})$. Since $|H|=n^{3/4}$, the proof will be of size $\Theta(\sqrt[3]{n^{3/4}})=\Theta(\sqrt[4]{n})$. However, this is not possible since the $\Theta(\sqrt[3]{n})$ proof is not a set-membership proof for arbitrary sets and in fact it doesn't work for elements in the image of $h$. Going a step forward, the proof works for elements of the form $(a\cP_s,a^2\cP_s)$, where $a$ is an integer and $\cP_s$ is a generator of one of the base groups of a bilinear group. On the other hand, the image of $h$ contains elements of the form $\sum_{i=1}^m (a_i\cP_s,a_i^2\cP_s)$ for which we do not know how to construct a function like $h$ and thus, we do not how to construct a set membership proof of size $\Theta(\sqrt[3]{n})$.


\subsubsection{Erasures.}
In the security proof we need to embed a random preimage $A=\{\vecb{a}_1,\ldots,\allowbreak \vecb{a}_{q_{\mathsf{gen}}}\}$ of $h$ in the verification keys, where $q_{\mathsf{gen}}$ is the total number of verification keys. On the other hand, the adversary may adaptively corrupt parties obtaining all the random coins used to generate the verification key. That is, we need to reveal $\log_{\cP_S} \vecb{a}_i$ (the discrete logs of $\vecb{a}_i$) to the adversary, which is incompatible with the permutation pairing assumption and thus with the security of $h$. Since is not clear how to obliviously sample $(a_i\cP,a^2_i\cP)$ and we can only guess the set of corrupted parties with negligible probability, we are forced to use erasures: after sampling $a_i$ and computing $\vecb{a}_i$, the key generation algorithm erases $a_i$ and $a_i^2$.

%\subsubsection{Getting rid of the Permutation Pairing Assumption.} Gonzalez et al.~\cite{ACNS:GonRaf16} modify Groth and Lu's proof of correctness of a shuffle \cite{AC:GroLu07} to get rid of the permutation pairing assumption. They showed that the statement ``I know two vectors of group elements $(\vecb{a}_1,\ldots,\vecb{a}_m),(\vecb{a}'_1,\ldots,\vecb{a}'_m)$ which are equal up to a permutation'' can be showed with $m$ set membership proofs that $\vecb{a}'_1,\dots\vecb{a}'_m\in\{\vecb{a},\ldots,\vecb{a}_m\}$ and  a proof that$\sum_{i=1}^m \vecb{a}'_i=\sum_{i=1}^m \vecb{a}_i$.  Gonzalez et al.~construct a $\Theta(m)$ proof of the first statement under standard assumptions (DLin in symmetric groups), while the second statement can be proved using standard techniques.

%If we use Gonzalez et al.'s techniques we would have to show that for all $\vecb{a}'\in A',$ $\vecb{a}'\in A_ \mu$. However, we can't do this since $A_\mu$ is unknown to the verifier. Instead, it seems that we are using stronger properties of the permutation pairing assumption. We use $h(A_\mu)$ as a constant-size computationally binding commitment of the set $A_\mu$, i.e.~invariant under permutations of the input, which is also structure preserving, i.e.~$A_\mu\subset\GG^2_1$ and $h(A_\mu)\in\GG_1$. It is an interesting open problem to construct $h$ from standard assumptions (e.g.~DDH, DLin).

\subsubsection{Relation to \cite{AC:GonHevRaf15}.}
Our construction is similar to the set membership proof of Gonzalez et al.~{\cite[Appendix D.2]{AC:GonHevRaf15}. However, the proof system from \cite{AC:GonHevRaf15} does not suffice for constructing a ring signature because there, the CRS is fixed to a specific set and thus the resulting ring signature will be fixed to a specific ring. 


