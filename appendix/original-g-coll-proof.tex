\begin{lemma}
For any adversary $\advA$ there exists an adversary $\advB$ against SXDH such that for any $\tilde{x}\in\Z_q^m$ and any $\vecb{\beta}\in\bits^m$  the probability that $\advA(k)$, $k\gets\KGen(gk,\tilde{\vecb{x}},\vecb{\beta})$, outputs a collision $[\vecb{x}]_2,[\vecb{x}']_2$ for $h_\matr{A}$  is less than $2m \adv_{\mathrm{SXDH}}(\advB)$. key.
\end{lemma}
\begin{proof}
The proof follows from the indistinguishability of the following games.
\begin{description}
\item[$\sfGame_0(\advA)$:] This game honestly runs the collision resistance experiment for $h_{\matr{A}}$ and outputs 1 if $\vecb{x}\neq\vecb{x}'$ and $g_\matr{A}(\vecb{x})=g_{\matr{A}}(\vecb{x}')$.
\item[$\sfGame_1(\advA)$:] This games picks a random $i\gets[m]$ and aborts if the adversary requests the random coins for generating $k_i$ or outputs $[x_i]_2=[x'_i]$.
\item[$\sfGame_2(\advA)$:] This game is exactly as $\sfGame_1$ but $[\matr{U}]_1$ and $[\matr{V}]_2$ are sampled from the perfectly hiding distribution.
\item[$\sfGame_3(\advA)$:] This game is exactly as $\sfGame_2$ but $\matr{A}$ is sampled from $\mathcal{Q}_m^{\vecb{\beta}_i}$, where $\vecb{\beta}_i\in\bits^m$ contains a single 1 at position $i$. Consequently, $[\matr{B}]_2,[\matr{C}]_2,[\matr{D}]_2$ and corresponding proofs are computed using $\vecb{\beta}_i$.
\item[$\sfGame_4(\advA)$:] This game is exactly as $\sfGame_3$ but $[\matr{U}]_1$ and $[\matr{V}]_2$ are sampled from the perfectly binding distribution.
\end{description}
The probability we want to bound is $\Pr[\sfGame_0(\advA)=1]$ and it holds that $\Pr[\sfGame_0(\advA)=1]\leq m\Pr[\sfGame_1(\advA)=1]$ since $i$ is information theoretically hidden to $\advA$, while there is at least one index $i^*\in[m]$ such that $x_{i^*} \neq x_{i^*}'$.

It also holds that $\Pr[\sfGame_1(\advA)=1]-\Pr[\sfGame_2(\advA)=1]\leq\adv_\mathrm{SXDH}(\advB)$, for some adversary $\advB$, since the only change in the games is the Groth-Sahai commitment key which is changed from perfectly binding to perfectly hiding. However, here the argument is slightly more subtle. An adversary attempting to tell apart perfectly binding from perfectly hiding Groth-Sahai commitments keys (or equivalently, an adversary against SXDH) can't efficiently simulate $\sfGame_1(\advA)$ and $\sfGame_2(\advA)$ as they require the computation of $g$ while the discrete logarithm of the commitment keys is unknown. Let $\vecb{x}_1,\vecb{x}'_1$ and $\vecb{x}_2,\vecb{x}'_2$ the purported collisions output by $\advA$ in, respectively, $\sfGame_1$ and $\sfGame_2$. We note that $\Pr[\sfGame_1(\advA)=1] \leq \Pr[x_{1,i}\neq x'_{1,i}]$ and $\Pr[\sfGame_2(\advA)=1] = \Pr[x_{2,i}\neq x'_{2,i}]$, since in $\sfGame_2$ it holds that $g_{\matr{A}}(\vecb{x}) = \vecb{k}^\top\matr{A}\vecb{x} = 0 = g_\matr{A}(\vecb{x}')$ for any $\vecb{x},\vecb{x}'$. Therefore, $\Pr[\sfGame_1(\advA)=1]-\Pr[\sfGame_2(\advA)=1]\leq \Pr[x_{1,i}\neq x'_{1,i}]-\Pr[x_{2,i}\neq x'_{2,i}]$. Instead of simulating the games, the adversary $\advB$, which receives as challenge Groth-Sahai commitment keys, runs $\sfGame_1$ replacing the commitment keys by it challenges and outputs $1$ if $x_i\neq x'_i$ and $0$ otherwise regardless of whether $g_\matr{A}(\vecb{x})=g_{\matr{A}}(\vecb{x}')$ or not.

It also holds that $\Pr[\sfGame_2(\advA)=1]-\Pr[\sfGame_3(\advA)=1]=0$, since commitment keys $[\matr{U}]_1,[\matr{V}]_2$ are perfectly hiding in both games, and hence, matrices $[\matr{A}]_1,[\matr{B}]_2,[\matr{C}]_2,[\matr{D}]_2$ follow exactly the same distribution in both games.
%Note that, by the self-reducibility of DDH, we can change many ciphertexts at once without increasing  the security loss.\footnote{For completeness, assume that you want to switch from encryptions of $m_1,\ldots,m_\ell\in\Z_q$ to encryptions of $m'_1,\ldots,m'_\ell\in\Z_q$. Construct an adversary that asks to its left or right oracle for encryptions of $0$ or $1$, receives $[\vecb{c}]_2$ as challenge, and returns $[\vecb{c}_i] := m_i[\vecb{c}_i] + m'_i[\smallpmatrix{0\\1}-\vecb{c}_i]_2+\delta_i[\vecb{v}]_2$, $\delta_i\gets\Z_q$. Clearly, when $[\vecb{c}]_2$ encrypts $1$, then the adversary returns encryptions to $m_1,\ldots,m_\ell$, and when it encrypts $0$ returns encryptions to $m'_1,\ldots,m'_\ell$.} Note that, when changing the cyperthexts, we need to simulate proofs as in (\ref{eq:Qm-sim-proofs}) and (\ref{eq:sim-proofs}).
Similarly as before, $\Pr[\sfGame_4(\advA)=1]-\Pr[\sfGame_3(\advA)=1]\leq\adv_\mathrm{SXDH}(\advB)$.

Finally, $\Pr[\sfGame_4(\advA)=1]=0$ since in this case $x_i \neq x'_i$ and $g_\matr{A}(\vecb{x}) = x_i$.
\end{proof}