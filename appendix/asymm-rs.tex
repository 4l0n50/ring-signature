In the following let $n:=|R|, m:=\sqrt[3]{n}$, and for $1\leq \alpha\leq n$ define $1\leq \mu \leq n^{2/3}$ and $1\leq \nu\leq m$ such that $\alpha=(\mu-1)m+\nu$. For a sequence $\{s\}_{1\leq i\leq n}$ we define $s_{\mu,\nu}:=s_{(\mu-1)m+\nu}$. Consider $\mathsf{OT}=(\mathsf{OT}.\KG,\mathsf{OT}.\mathsf{Sign},\allowbreak\mathsf{OT}.\mathsf{Ver})$ a one-time signature scheme.

\begin{description}
\item[$\mathsf{CRSGen}(gk)$:] Pick a perfectly hiding CRS for the Groth-Sahai proof system $\crs_\GS$. Note that $\crs_\GS$ can be also used for the $\Theta(\sqrt{n})$ set-membership of Chandran et al. The CRS is $\rho:=(gk,\crs_\GS).$

\item[$\KG(\rho)$:] Pick $\vecb{a}\gets\mathcal{Q}$ and $(sk,[vk]_2)\gets\mathsf{BB}.\KG(gk)$, compute $[\vecb{a}]_1,[b]_2=[a_1]_2$ and then erase $\vecb{a}$. The secret key is $sk$ and the verification key is $\vecb{vk}:=([vk]_2,[\vecb{a}]_1,[b]_2,\vecb{a}[vk]_2)$.

\item[$\mathsf{Sign}_{\rho,sk}(m,R)$:] Let $\alpha=(\mu-1)m+\nu$ the index of the signer with respect to $R$.
\begin{enumerate}
\item Compute $(sk_\mathsf{ot},vk_\mathsf{ot})\gets\mathsf{OT}.\KG(gk)$ and $\sigma_\mathsf{ot}\gets\allowbreak\mathsf{OT}.\allowbreak\mathsf{Sign}_{sk_\mathsf{ot}}(m,R)$.

\item Compute $[\vecb{c}]_2:=\GS.\Com_{ck}([vk_\alpha]_2;\vecb{r})$, $\vecb{r}\gets\Z_q^3$, $[\sigma]_1\gets\mathsf{BB}.\mathsf{Sign}_{sk}(vk_\mathsf{ot})$, $[\vecb{d}]_1:=\GS.\Com_{ck}([\sigma]_1;\vecb{s})$, $\vecb{s}\gets\Z_q^3$, and a GS proof $\pi_\mathsf{BB}$ that $\mathsf{BB}.\mathsf{Ver}_{[vk]_2}(\allowbreak[\sigma]_1,vk_\mathsf{ot})=1$.

\item For $1\leq i \leq n^{2/3}$, let $[\vecb{\kappa}_i]_2=([vk_{i,1}]_2,\ldots,[vk_{i,m}]_2)^\top$, $A_i=\{[\vecb{a}_{i,1}]_1,\allowbreak\ldots,\allowbreak[\vecb{a}_{i,m}]_1\}$,  $[\matr{A}_i]_1:=[\vecb{a}_{i,1}\cat\cdots\cat\vecb{a}_{i,m}]_1$, $B=\{[b_{i,1}]_2,\ldots,[b_{i,m}]_2\}$, and $[\vecb{b}_i]_2:=[(b_{i,1},\ldots,b_{i,m})]_2$ . Define the sets
$H=\{h(A_1,B_1),\allowbreak\ldots,\allowbreak h(A_{n^{2/3}},B_{n^{2/3}})\}$ and
$G=\{
	g_{[\matr{A}_1]_1}([\vecb{\kappa}_1]_2)
	\allowbreak\ldots,\allowbreak
	g_{[\matr{A}_{n^{2/3}}]_1}([\vecb{\kappa}_{n^{2/3}}]_2)\}$.

\item Let $[\vecb{x}]_1:=h(A_\mu,B_\mu)$ and $[\vecb{y}]_2=g_{[\matr{A}_\mu]_1}([\vecb{\kappa}_\mu]_2)$. Compute GS commitments to $[\vecb{x}]_1$ and $[\vecb{y}]_2$ and compute proofs $\pi_G$ and $\pi_H$ that they belong to $G$ and $H$, respectively. It is also proven that they appear in the same positions reusing the commitments to $b_1,\ldots,b_{m}$ and $b'_1,\ldots,b'_{m}$, used in the set-membership proof of Chandran et al., which define $[\vecb{x}]_1$'s and $[\vecb{y}]_2$'s position in $H$ and $G$ respectively.

\item Let
$$[\vecb{\kappa'}]_2:=([vk_\alpha]_2,[vk_{\mu,1}]_2,\ldots,[vk_{\alpha-1}]_2,[vk_{\alpha+1}]_2,\ldots,[vk_{\mu,m}]_2)^\top\in\GG_2^m$$
$$[\matr{A}']_1:=[\vecb{a}_\alpha \cat \vecb{a}_{\mu,1} \cat \cdots \cat \vecb{a}_{\alpha-1}\cat \vecb{a}_{\alpha+1}\cat\cdots\cat\vecb{a}_{\mu,m}]_1\in\GG_1^{2\times m}$$
and
$$
[\vecb{b}']_2 = [(b_\alpha, b_{\mu,1},\ldots,b_{\alpha-1},b_{\alpha+1},\ldots,b_{\mu,m})]_2
$$
Compute GS commitments to all but the first element of $[\vecb{\kappa}']_2$ (note that $[\vecb{c}]_2$ is a commtiment to the first element of $[\vecb{\kappa}']_2$) and to each element of $[\matr{A}']_1$ and $[\vecb{b}']_2$. Compute also a GS proof $\pi_g$ that $g_{[\matr{A}']_1}([\vecb{\kappa}']_2)=[\vecb{y}]_2$, a GS proof $\pi_{h}$ that $h(A',B')=[\vecb{x}]_1$, were $A'$ is the set of $[\matr{A}']_1$'s columns, and a GS proof $\pi_{Q_m}$ that $(A',B')\in Q_m$.

\item Return the signature $\grkb{\sigma}:=(vk_\mathsf{ot},\sigma_\mathsf{ot},[\vecb{c}]_2,[\vecb{d}]_1,\pi_{\mathsf{BB}},\pi_G,\pi_H, \pi_g,\pi_h,\pi_{Q_m})$. (GS proofs include commitments to variables).
\end{enumerate}

\item[$\mathsf{Verify}_{\rho,R}(m,\grkb{\sigma})$:] Verify the validity of the one-time signature and of all the proofs. Return 0 if any of these checks fails and 1 otherwise.
\end{description}

\begin{theorem}
The scheme presented in this section is a ring signature scheme
with perfect correctness, perfect anonymity and computational unforgeability under the
$q_\mathsf{gen}\mbox{-}\mathrm{PPA}$, the $\mathcal{Q}_{q_\mathsf{gen}}^\top\mbox{-}\akermdh$ assumption, the $\mathrm{SXDH}$ assumption, and the assumption
that the one-time signature and the Boneh-Boyen signature are unforgeable.
Concretely, for any PPT adversary $\advA$ against the unforgeability of the scheme, there exist adversaries $\advB_1,\advB_2,\advB_3,\advB_4,\advB_5$ such that
\begin{align*}
\adv(\advA)\leq &\adv_{\mathrm{SXDH}}(\advB_1)+\adv_{q_\mathsf{gen}\mbox{-}\mathrm{aPPA}}(\advB_2)+\adv_{\mathcal{Q}^\top_{q_\mathsf{gen}}\mbox{-}\akermdh}(\advB_3)+\\
&q_\mathsf{gen}(q_\mathsf{sig}\adv_{\mathsf{OT}}(\advB_4)+\adv_{\mathsf{BB}}(\advB_5)),
\end{align*}
where $q_\mathsf{gen}$ and $q_\mathsf{sign}$ are, respectively, upper bounds for the number of queries that $\advA$ makes to its $\mathsf{VKGen}$ and $\mathsf{Sign}$ oracles.
\end{theorem}
The proof is just a syntactic transformation of the proof in the symmetric case and we omit it.
%\begin{proof}
%Perfect correctness follows directly from the definitions. Perfect anonymity follows from the fact that the perfectly hiding Groth-Sahai CRS defines perfectly hiding commitments and perfect witness-indistinguishable proofs, information theoretically hiding any information about $\vecb{vk}$.
%
%We say that an unforgeability adversary is ``eager'' if  makes all its queries to the $\mathsf{VKGen}$ oracle at the beginning. Note that any non-eager adversary $\advA'$ can be perfectly simulated  by an eager adversary that makes ${q_\mathsf{gen}}$ queries to $\mathsf{VKGen}$ and answers $\advA'$ queries to $\mathsf{VKGen}$ ``on demand''.
%
%W.l.o.g.~we assume that $\advA$ is an eager adversary. Computational unforgeability follows from the indistinguishability of the following games
%\begin{itemize}
%\item[$\sfGame_0$:] This is the real unforgeability experiment. $\sfGame_0$ returns 1 if the adversary $\advA$ produces a valid forgery and 0 if not.
%\item[$\sfGame_1$:] This is game exactly as $\sfGame_0$ with the following differences: 
%    \begin{itemize}
%    \item The Groth-Sahai CRS is sampled together with its discrete logarithms from the perfectly binding distribution.
%    \item At the beginning, variables $\mathsf{err}_2$ and $\mathsf{err}_3$ are initialized to $0$, and a random index $i^*$ is chosen from $\{1,\ldots, q_\mathsf{gen}\}$.
%    \item On a query to $\mathsf{Corrupt}$ with argument $i$, if $i=i^*$ set $\mathsf{err_3}\gets 1$ and proceed as in $\sfGame_0$.
%    \item Let $(m,R,\sigma)$ the purported forgery output by $\advA$. If $[vk]$, the opening of commitment $[\vecb{c}]$ from $\sigma$, is not equal to $[vk_{i^*}]$,  set $\mathsf{err}_3\gets 1$. If $[vk]\notin R$, then set $\mathsf{err}_2=1$.
%    \end{itemize}
%\item[$\sfGame_2$:] This is game exactly as $\sfGame_1$ except that, if $\mathsf{err}_2$ is set to 1, $\sfGame_2$ aborts.
%\item[$\sfGame_3$:] This is game exactly as $\sfGame_2$ except that, if $\mathsf{err}_3$ is set to 1, $\sfGame_3$ aborts. 
%\end{itemize}
%Since in $\sfGame_1$ variables $\err_2$ and $\err_3$ are just dummy variables, the only difference with $\sfGame_0$ comes from the Groth-Sahai CRS distribution. It follows that there is an adversary $\advB_{1}$ against DLin such that $|\Pr[\sfGame_0=1]-\Pr[\sfGame_1=1]|\leq \adv_{\mathrm{DLin}}(\advB_{1})$.
%
%\begin{lemma} There exist adversaries $\advB_2$ and $\advB_3$ against the ${q_\mathsf{gen}}$-permutation pairing assumption and against the $\mathcal{Q}^\top_{{q_\mathsf{gen}}}\mbox{-}\kermdh$ assumption, respectively, such that
%$$
%|\Pr[\sfGame_2=1]-\Pr[\sfGame_1=1]|\leq \adv_{{q_\mathsf{gen}}\mbox{-}\mathrm{PPA}}(\advB_2)+\adv_{\mathcal{Q}^\top_{{q_\mathsf{gen}}}\mbox{-}\kermdh}(\advB_3).
%$$
%\end{lemma}
%\begin{proof}
%Note that
%\begin{align*}
%\Pr[\sfGame_1=1]
% = &\Pr[\sfGame_1=1|\err_2=0]\Pr[\err_2=0]+\\
%&\Pr[\sfGame_1=1|\err_2=1]\Pr[\err_2=1]\\
% \leq& \Pr[\sfGame_2=1] + \Pr[\sfGame_1=1|\err_2=0]\\
%\Longrightarrow  &|\Pr[\sfGame_2=1]-\Pr[\sfGame_1=1]|\leq \Pr[\sfGame_1=1|\err_2=1].
%\end{align*}
%We proceed to bound this last probability constructing two adversaries against collision resistance of $g$ and preimage resitance of $h$. Let $1\leq \mu\leq n^{2/3}$ the index defined in $\pi_G$ and $\pi_S$.
%
%Consider an adversary $\advA_h$ that finds a second preimage of $h$ when $\mathcal{M}=Q_{q_\mathsf{gen}}$. $\advA_h$ receives as challenge $B\in Q_{q_\mathsf{gen}}$ and honestly simulates $\sfGame_1$ with the following exception. On the $i$ th query of $\advA$ to $\mathsf{VKGen}$ picks $(sk,[vk])\gets\mathsf{BB}.\KG(1^\lambda)$ and sets $(sk_i,\vecb{vk}_i):=(sk,([vk],[\vecb{b}_i],sk[\vecb{b}_i]))$, where $[\vecb{b}_i]$ is the $i$ th element of $B$. When $\advA$ outputs $\GS.\Com_{ck_\GS}([\vecb{a}'_1]),\ldots,\allowbreak\GS.\Com_{ck_\GS}([\vecb{a}'_m])$, $\advA_h$ extracts $A'=\{[\vecb{a}'_1],\ldots,[\vecb{a}'_m]\}$ and returns $A'\cup \bar{A}_\mu$, where $\bar{A}_\mu:= B\setminus A_\mu$.
%
%Consider another adversary $\advA_g$ against the collision resitance of $g$ when $\mathcal{M}=\GG^{q_\mathsf{gen}}$. $\advB$ receives as challenge $[\matr{B}]\in\GG^{2\times {q_\mathsf{gen}}}$ and honestly simulates $\sfGame_1$ embedding $[\matr{B}]$ in the user keys in the same way as $\advA_h$. When $\advA$ outputs $[\vecb{c}],\allowbreak\GS.\Com_{ck_\GS}([\kappa'_2]),\ldots,\allowbreak\GS.\Com_{ck_\GS}([\kappa'_m])$, $\advA_g$ extracts $[vk],[\kappa'_2],\ldots,[\kappa'_m]$. W.l.o.g.~\allowbreak assume that $[\matr{B}]=[\matr{A}_\mu\cat\bar{\matr{A}}_\mu]$, where $[\bar{\matr{A}}_\mu]$ is some matrix whose rows are the elements of $\bar{A}_\mu$. $\advA_g$ attempts to extract a permutation matrix $\matr{P}$ such that  $[\matr{A}']=[\matr{A}_\mu]\matr{P}$. If there is no such permutation matrix, then $\advA_g$ aborts. Else,  $\advA_g$ returns
%$\pmatri{
%	{[\vecb{\kappa}_\mu]}\\
%	{[\vecb{0}]}},
%\pmatri{
%	{\matr{P}[\vecb{\kappa}']}\\
%	{[\vecb{0}]}}
%\in\GG^{q_\mathsf{gen}}$,
%where $[\kappa'_1]$ is the oppening of $[\vecb{c}]$.
%
%Perfect soundness of proof $\pi_g$  (recall that the Groth-Sahai CRS is perfectly binding)  implies that
%\begin{align*}
%&g_{[\matr{A}']}([\vecb{\kappa}'])=[\vecb{y}]. %\label{eq-rs-1}
%\end{align*}
%Perfect soundness of proof $\pi_g$ and $\pi_{Q_m}$ implies that
%\begin{align*}
%&h(A')=[\vecb{x}]\text{ and } A'\in Q_m.%\label{eq-rs-2}
%\end{align*}
%Given perfect soundnes of proofs $\pi_G,\pi_H$, it holds that that
%\begin{align*}
%&g_{[\matr{A}']}([\vecb{\kappa}'])=g_{[\matr{A}_\mu]}([\vecb{\kappa}_\mu]),%\label{eq-rs-3}
%\\
%&h(A')=h(A_\mu), %\label{eq-rs-4}.
%\end{align*}
%By Lemma \ref{lemma:hg} we get that either $A'\neq {A}_\mu$ is a second preimage for $h(A_\mu)$, thus $A'\cup\bar{A}_\mu\neq B$ and $\advA_h$ is successful, or there exists a permutation matrix $\matr{P}$, which is the one that $\advA_g$ searches, such that $g_{[\matr{A}_\mu]}(\matr{P}[\vecb{\kappa}'])=g_{[\matr{A}_\mu]}([\vecb{\kappa}_\mu])$. $\mathsf{err}_2=1$ implies that $[vk]=[\kappa'_1]\neq[\kappa_{\mu,i}]$, for all $1\leq i\leq m$, and thus $\matr{P}[\vecb{\kappa}']\neq [\vecb{\kappa}_\mu]$ and, since $[\matr{B}]=[\matr{A}_\mu\cat\bar{\matr{A}}_\mu]$,
%$$
%g_{[\matr{A}_\mu]}(\matr{P}[\vecb{\kappa}'])=
%g_{[\matr{B}]}\pmatri{
%	{\matr{P}[\vecb{\kappa}']}\\
%	{[\vecb{0}]}}
%=
%g_{[\matr{A}_\mu]}([\vecb{\kappa}_\mu])
%=
%g_{[\matr{B}]}\pmatri{
%	{[\vecb{\kappa}_\mu]}\\
%	{[\vecb{0}]}}
%$$ 
%and $\advA_g$ is successful.
%
%As stated in Section \ref{sec:hash}, from $\advA_h$ we can construct an adversary $\advB_2$ that breaks the $q_\mathsf{gen}$-PPA assumption and from $\advA_g$ we can construct an adversary $\advB_3$ that breaks the $\mathcal{Q}^\top_m\mbox{-}\kermdh$ assumption, with the same advantages. We conclude that 
%\begin{align*}
%\Pr[\sfGame_1=1|\err_2=1] \leq & \adv_{{q_\mathsf{gen}}\mbox{-}\mathrm{PPA}}(\advB_2)+\adv_{\mathcal{Q}^\top_{q_\mathsf{gen}}\mbox{-}\kermdh}(\advB_3)
%\end{align*}
%\end{proof}
%
%\begin{lemma}
%$$
%\Pr[\sfGame_3=1]\geq \frac{1}{{q_\mathsf{gen}}}\Pr[\sfGame_2=1].
%$$
%\end{lemma}
%\begin{proof}
%%\footnote{The analysis used in this proof uses similar techniques to Boneh and Franklin's analysis of identity based encryption \cite{C:BonFra01}, which in turn is inspired on Coron's analysis of full domain hash \cite{C:Coron00}.}
%It holds that
%\begin{align*}
%\Pr[\sfGame_3=1] &= \Pr[\sfGame_3=1|\mathsf{err}_3=0]\Pr[\mathsf{err}_3=0]\\
%&=\Pr[\sfGame_2=1|\mathsf{err}_3=0]\Pr[\mathsf{err}_3=0]\\
%&=\Pr[\mathsf{err}_3=0|\sfGame_2=1]\Pr[\sfGame_2=1].
%\end{align*}
%The probability that $\mathsf{err}_3=0$ given $\sfGame_2=1$ is the probability that the ${q_\mathsf{cor}}$ calls to $\mathsf{Corrupt}$ do not abort and that $[vk]=[vk_{i^*}]$. Since $\advA$ is an eager adversary, at the $i$ th call to $\mathsf{Corrupt}$ the index $i^*$ is uniformly distributed over the ${q_\mathsf{gen}}-i+1$ indices of uncorrupted users. Similarly, when $\advA$ outputs its purported forgery, the probability that $[vk]=[vk_{i^*}]$ is $1/({q_\mathsf{gen}}-{q_\mathsf{cor}})$, since $[vk]\in R$ (or otherwise $\sfGame_2$ would have aborted). Therefore
%$$
%\Pr[\mathsf{err}_2=1|\sfGame_2=1]=\frac{{q_\mathsf{gen}}-1}{{q_\mathsf{gen}}}\frac{{q_\mathsf{gen}}-2}{{q_\mathsf{gen}}-1}\ldots\frac{{q_\mathsf{gen}}-{q_\mathsf{cor}}}{{q_\mathsf{gen}}-{q_\mathsf{cor}}+1}\frac{1}{{q_\mathsf{gen}}-{q_\mathsf{cor}}}=\frac{1}{{q_\mathsf{gen}}}.
%$$ 
%\end{proof}
%
%\begin{lemma}  There exist adversaries $\advB_4$ and $\advB_5$ against the unforgeability of the one-time signature scheme and the weak unforgeability of the Boneh-Boyen signature scheme such that
%$$
%\Pr[\sfGame_3=1]\leq q_\mathsf{sig}\adv_{\mathsf{OT}}(\advB_4)+\adv_{\mathsf{BB}}(\advB_5)
%$$
%\end{lemma}
%\begin{proof}
%We construct adversaries $\advB_4$ and $\advB_5$ as follows.
%
%$\advB_4$ receives $vk_\mathsf{ot}^\dag$ and simulates $\sfGame_3$ honestly but with the following differences. It chooses a random $j^*\in\{1,\ldots, q_\mathsf{sig}\}$ and answer the $j^*$ th query to $\mathsf{Sign}(i,m^\dag,R^\dag)$ honestly but computing $\sigma_\mathsf{ot}^\dag$ querying on $(m^\dag,R^\dag)$ its oracle and setting $vk_\mathsf{ot}^\dag$ as the corresponding one-timeverification key. Finally, when $\advA$ outputs its purported forgery $(m,R,(\sigma_\mathsf{ot},vk_\mathsf{ot},\ldots))$, $\advB_4$ outputs the corresponding one-time signature.
%
%$\advB_5$ receives $[vk]$ and simulates $\sfGame_3$ honestly but with the following differences. Let $i:=0$. $\advB_5$ computes $(sk_\mathsf{ot}^i,vk_\mathsf{ot}^i)\gets\mathsf{OT}.\mathsf{KeyGen}(gk)$, for each $1\leq i\leq q_\mathsf{sig}$ and queries its signing oracle on $(vk_{\mathsf{ot}}^1,\ldots,vk_\mathsf{ot}^{q_{\mathsf{sig}}})$ obtaining $[\sigma_1],\ldots,[\sigma_{q_\mathsf{sig}}]$. On the $i^*$ th query of $\advA$ to the key generation algorithm, $\advB_5$ picks $\vecb{a}\gets\mathcal{Q}$ and outputs $\vecb{vk}:=([vk],[\vecb{a}],\vecb{a}[vk])$. When $\advA$ queries the signing oracle on input $(i^*,m,R)$, $\advB_5$ computes an honest signature but replaces $vk_\mathsf{ot}$ with $vk_\mathsf{ot}^i$ and $[\sigma]$ with $[\sigma_i]$, and then adds 1 to $i$. Finally, when $\advA$ outputs its purported forgery $(m,R,(\sigma_\mathsf{ot},vk_\mathsf{ot},[\vecb{c}],[\vecb{d}],\ldots))$, it extracts $[\sigma]$ from $[\vecb{d}]$ as its forgery for $vk_\mathsf{ot}$.
%
%Let $E$ be the event where $vk_\mathsf{ot}$, from the purported forgery of $\advA$, has been previously output by $\Sign$. We have that
%$$
%\Pr[\sfGame_3=1]\leq \Pr[\sfGame_3=1|E]+\Pr[\sfGame_3=1|\neg E].
%$$
%Since  $(m,R)$ has never been signed by a one-time signature and that, conditioned on $E$, the probability of $vk_\mathsf{ot}=vk_\mathsf{ot}^\dag$ is $1/q_\mathsf{sig}$, then
%\begin{align*}
%q_{\mathsf{sig}}\adv_\mathsf{OT}(\advB_4)\geq  \Pr[\sfGame_3=1|E]
%\end{align*}
%Finally, if $\neg E$ holds, then $[\sigma]$ is a forgery for $vk_\mathsf{ot}$ and thus
%$$
%\adv_{\mathsf{BB}}(\advB_5)\geq \Pr[\sfGame_3=1|\neg E]$$
%\end{proof}
%\end{proof}
