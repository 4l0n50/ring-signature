% !TEX root = ../main-ring-signature.tex

%\subsection{Erasures}
\label{sect:erasures}
In the security proof of our PPA-based ring signature we need to embed a random preimage $A=\{\vecb{a}_1,\ldots,\allowbreak \vecb{a}_{q_{\mathsf{gen}}}\}$ of $h$ in the verification keys, where $q_{\mathsf{gen}}$ is the total number of verification keys. On the other hand, the adversary may adaptively corrupt parties obtaining all the random coins used to generate the verification key. That is, we need to reveal $\log_{\cP_S} \vecb{a}_i$ (the discrete logs of $\vecb{a}_i$) to the adversary, which is incompatible with the permutation pairing assumption and thus with the security of $h$. Since is not clear how to obliviously sample $(a_i\cP,a^2_i\cP)$ and we can only guess the set of corrupted parties with negligible probability, we are forced to use erasures: after sampling $a_i$ and computing $\vecb{a}_i$, the key generation algorithm erases $a_i$ and $a_i^2$.

Erasures were considered by Bender et al.~\cite{TCC:BenKatMor06} but only with respect to anony\-mity. Our signature achieves the stronger notion of perfect anonymity of Chandran et al.~\cite{ICALP:ChaGroSah07}, meaning that, information theoretically, there is nothing in the signature that binds a signer to a signature. Since the random coins of our key generation algorithm are completely determined by the public key, in the information theoretic setting it is irrelevant if parties erase or not part of their random coins.

On the other hand, erasures in the unforgeability experiment seems to have not received much attention. In fact, the definition of Bender et al.~doesn't prevent erasures, since, after corrupting a party, the adversary receives only the secret key but not the random coins used by the key generation algorithm.  Chandran et al.'s unforgeability definition explicitly includes the random coins in the adversary view, preventing any erasure. However, this is not discussed and further, erasures are not even mentioned in their work.

We would also like to point out that other schemes may also require erasures. This is the case Malavolta et al.'s CRS-less ring signature. In order to get rid of the CRS, which is a pair of Groth-Sahai commitment keys, each party appends its own Groth-Sahai commitment keys to its public key. The signer and the verifier combines all the commitment keys by simply adding them and, as long as at least one verification key was honestly generated, the combined commitment keys are correctly distributed.

Nevertheless, when proving unforgeability one needs to move from perfectly hiding commitment keys to perfectly binding commitment keys. This implies that the reduction must change itself the verification keys of all the users to perfectly binding ones. In the perfectly hiding setting commitment keys are chosen as $\vecb{u}_1 = \lambda\vecb{u}_2$ and $\vecb{u}_2 = (a\mathcal{P},\mathcal{P})^\top$, for a random $\lambda$ and random $a$.  On the other hand, perfectly binding commitments keys have the only difference that $\vecb{u}_1 = (\mathcal{P},0)^\top+\lambda\vecb{u}_2$.
An adversary that dynamically corrupts parties will eventually gets access to $\lambda$ and $a$, which clearly allows him to detect any change on the commitment keys. We believe is an interesting open question if this problem can be fixed.


