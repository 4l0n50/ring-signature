% !TEX root = ../main-ring-signature.tex

We restate Theorem \ref{theo:security} and prove it.
\begin{theorem}
The scheme presented in Section \ref{sec:our-construction} is a ring signature scheme
with perfect correctness, perfect anonymity and computational unforgeability under the
$q_\mathsf{gen}$-permutation pairing assumption, the $\mathcal{Q}_{q_\mathsf{gen}}^\top\mbox{-}\skermdh$ assumption, the $\mathrm{SXDH}$ assumption, and the assumption
that the one-time signature and the Boneh-Boyen signature are unforgeable.
Concretely, for any PPT adversary $\advA$ against the unforgeability of the scheme, there exist adversaries $\advB_1,\advB_2,\advB_3,\advB_4,\advB_5$ such that
\begin{align*}
\adv(\advA)\leq &\adv_{\mathrm{SXDH}}(\advB_1)+\adv_{q_\mathsf{gen}\mbox{-}\mathrm{PPA}}(\advB_2)+\adv_{\mathcal{Q}^\top_{q_\mathsf{gen}}\mbox{-}\skermdh}(\advB_3)+\\
&q_\mathsf{gen}(q_\mathsf{sig}\adv_{\mathsf{OT}}(\advB_4)+\adv_{\mathsf{BB}}(\advB_5)),
\end{align*}
where $q_\mathsf{gen}$ and $q_\mathsf{sign}$ are, respectively, upper bounds for the number of queries that $\advA$ makes to its $\mathsf{VKGen}$ and $\mathsf{Sign}$ oracles.
\end{theorem}

\begin{proof}
Perfect correctness follows directly from the definitions. Perfect anonymity follows from the fact that the perfectly hiding Groth-Sahai CRS defines perfectly hiding commitments and perfect witness-indistinguishable proofs, information theoretically hiding any information about $\vecb{vk}$.

We say that an unforgeability adversary is ``eager'' if  makes all its queries to the $\mathsf{VKGen}$ oracle at the beginning. Note that any non-eager adversary $\advA'$ can be perfectly simulated  by an eager adversary that makes ${q_\mathsf{gen}}$ queries to $\mathsf{VKGen}$ and answers $\advA'$ queries to $\mathsf{VKGen}$ ``on demand''. This is justified by the fact that the output of $\mathsf{VKGen}$ is independent of all previous outputs.

W.l.o.g.~we assume that $\advA$ is an eager adversary. Computational unforgeability follows from the indistinguishability of the following games
\begin{itemize}
\item[$\sfGame_0$:] This is the real unforgeability experiment. $\sfGame_0$ returns 1 if the adversary $\advA$ produces a valid forgery and 0 if not.
\item[$\sfGame_1$:] This is game exactly as $\sfGame_0$ with the following differences: 
    \begin{itemize}
    \item The Groth-Sahai CRS is sampled together with its discrete logarithms from the perfectly binding distribution. Note that the discrete logarithms of the CRS allow to open the Groth-Sahai commitments.
    \item At the beginning, variables $\mathsf{err}_2$ and $\mathsf{err}_3$ are initialized to $0$ and a random index $i^*$ is chosen from $\{1,\ldots, q_\mathsf{gen}\}$.
    \item On a query to $\mathsf{Corrupt}$ with argument $i$, if $i=i^*$ set $\mathsf{err_3}\gets 1$ and proceed as in $\sfGame_0$.
    \item Let $(m,R,\sigma)$ the purported forgery output by $\advA$. If $[vk]_2$, the opening of commitment $[\vecb{c}]_2$ from $\sigma$, is not equal to $[vk_{i^*}]_2$,  set $\mathsf{err}_3\gets 1$. If $[vk]_2\notin R$, then set $\mathsf{err}_2=1$.
    \end{itemize}
\item[$\sfGame_2$:] This is game exactly as $\sfGame_1$ except that, if $\mathsf{err}_2$ is set to 1, $\sfGame_2$ aborts.
\item[$\sfGame_3$:] This is game exactly as $\sfGame_2$ except that, if $\mathsf{err}_3$ is set to 1, $\sfGame_3$ aborts. 
\end{itemize}
Since in $\sfGame_1$ variables $\err_2$ and $\err_3$ are just dummy variables, the only difference with $\sfGame_0$ comes from the Groth-Sahai CRS distribution. It follows that there is an adversary $\advB_{1}$ against SXDH such that $|\Pr[\sfGame_0=1]-\Pr[\sfGame_1=1]|\leq \adv_{\mathrm{SXDH}}(\advB_{1})$.

\begin{lemma} There exist adversaries $\advB_2$ and $\advB_3$ against the ${q_\mathsf{gen}}$-permutation pairing assumption and against the $\mathcal{Q}^\top_{{q_\mathsf{gen}}}\mbox{-}\kermdh$ assumption, respectively, such that
$$
|\Pr[\sfGame_2=1]-\Pr[\sfGame_1=1]|\leq \adv_{{q_\mathsf{gen}}\mbox{-}\mathrm{PPA}}(\advB_2)+\adv_{\mathcal{Q}^\top_{{q_\mathsf{gen}}}\mbox{-}\skermdh}(\advB_3).
$$
\end{lemma}
\begin{proof}
Note that
\begin{align*}
\Pr[\sfGame_1=1]
 = &\Pr[\sfGame_1=1|\err_2=0]\Pr[\err_2=0]+\\
&\Pr[\sfGame_1=1|\err_2=1]\Pr[\err_2=1]\\
 \leq& \Pr[\sfGame_2=1] + \Pr[\sfGame_1=1|\err_2=0]\\
\Longrightarrow  &|\Pr[\sfGame_2=1]-\Pr[\sfGame_1=1]|\leq \Pr[\sfGame_1=1|\err_2=1].
\end{align*}
We proceed to bound this last probability constructing two adversaries against collision resistance of $g$ and preimage resistance of $h$. Let $1\leq \mu\leq n^{2/3}$ the index defined in $\pi_G$ and $\pi_S$.

Consider an adversary $\advA_h$ that finds a second preimage of $h$ when $\mathcal{M}=Q_{q_\mathsf{gen}}$. $\advA_h$ receives as challenge $B\in Q_{q_\mathsf{gen}}$ and honestly simulates $\sfGame_1$ with the following exception. On the $i$ th query of $\advA$ to $\mathsf{VKGen}$ picks $(sk,[vk])\gets\mathsf{BB}.\KG(1^\lambda)$ and sets $(sk_i,\vecb{vk}_i):=(sk,([vk]_2,[\vecb{b}_i]_1,[\vecb{b}_i]_2,sk[\vecb{b}_i]_2))$, where $([\vecb{b}_i]_1,[\vecb{b}_i]_2)$ is the $i$ th element of $B$. When $\advA$ corrupts the $i$ th party, it returns $sk_i$ but it might also request $\vecb{a}_i$ to its oracle if we are proving securtity under the $(\ell,m)$-PPA assumption. When $\advA$ outputs and $\pi_{Q_m}$, $\advA_h$ extracts $A'=\{([\vecb{a}'_1]_1,[\vecb{a}'_1]_2),\ldots,([\vecb{a}'_m]_1,[\vecb{a}'_m]_2\}$ and returns $A'\cup \bar{A}_\mu$, where $\bar{A}_\mu:= B\setminus A_\mu$.

Consider another adversary $\advA_g$ against the collision resistance of $g$ when $\mathcal{M}=\GG^{q_\mathsf{gen}}$. $\advB$ receives as challenge $[\matr{B}]_1\in\GG^{2\times {q_\mathsf{gen}}}_1$ and $[\matr{B}]_2\in\GG^{2\times {q_\mathsf{gen}}}_2$ and honestly simulates $\sfGame_1$ embedding $[\matr{B}]_1,[\matr{B}]_2$ in the user keys in the same way as $\advA_h$. When $\advA$ outputs $[\vecb{c}]_2,\allowbreak\GS.\Com_{ck_2}([\kappa'_2]_2),\ldots,\allowbreak\GS.\Com_{ck_2}([\kappa'_m]_2)$, $\advA_g$ extracts $[vk],[\kappa'_2],\ldots,[\kappa'_m]$. W.l.o.g.~\allowbreak assume that $\matr{B}=\matr{A}_\mu\cat\bar{\matr{A}}_\mu$, where $\bar{\matr{A}}_\mu$ is some matrix whose rows are the discrete logs of the elements of $\bar{A}_\mu$. $\advA_g$ attempts to extract a permutation matrix $\matr{P}$ such that  $[\matr{A}']_1=[\matr{A}_\mu]_1\matr{P}$. If there is no such permutation matrix, then $\advA_g$ aborts. Else,  $\advA_g$ returns
$\pmatri{
	{[\vecb{\kappa}_\mu]_2}\\
	{[\vecb{0}]_2}},
\pmatri{
	{\matr{P}[\vecb{\kappa}']_2}\\
	{[\vecb{0}]_2}}
\in\GG^{q_\mathsf{gen}}_2$,
where $[\kappa'_1]$ is the opening of $[\vecb{c}]$.

Perfect soundness of proof $\pi_g$  (recall that the Groth-Sahai CRS is perfectly binding)  implies that
\begin{align*}
&g_{[\matr{A}']_1}([\vecb{\kappa}']_2)=[\vecb{y}]_2. %\label{eq-rs-1}
\end{align*}
Perfect soundness of proof $\pi_g$ and $\pi_{Q_m}$ implies that
\begin{align*}
&h(A')=[\vecb{x}]_1\text{ and } A'\in Q_m.%\label{eq-rs-2}
\end{align*}
Given perfect soundness of proofs $\pi_G,\pi_H$, it holds that that
\begin{align*}
&g_{[\matr{A}']_1}([\vecb{\kappa}']_2)=g_{[\matr{A}_\mu]_1}([\vecb{\kappa}_\mu]_2)%\label{eq-rs-3}
\\
&h(A')=h(A_\mu). %\label{eq-rs-4}.
\end{align*}
By Lemma \ref{lemma:hg} we get that either $A'\neq {A}_\mu$ is a second preimage for $h(A_\mu)$, thus $A'\cup\bar{A}_\mu\neq B$ and $\advA_h$ is successful, or there exists a permutation matrix $\matr{P}$, which is the one that $\advA_g$ searches, such that $g_{[\matr{A}_\mu]_1}(\matr{P}[\vecb{\kappa}']_2)=g_{[\matr{A}_\mu]_1}([\vecb{\kappa}_\mu]_2)$. $\mathsf{err}_2=1$ implies that $[vk]_2=[\kappa'_1]_2\neq[\kappa_{\mu,i}]_2$, for all $1\leq i\leq m$, and thus $\matr{P}[\vecb{\kappa}']_2\neq [\vecb{\kappa}_\mu]_2$ and, since $[\matr{B}]_1=[\matr{A}_\mu\cat\bar{\matr{A}}_\mu]_1$,
$$
g_{[\matr{A}_\mu]_1}(\matr{P}[\vecb{\kappa}']_2)=
g_{[\matr{B}]_1}\pmatri{
	{\matr{P}[\vecb{\kappa}']_2}\\
	{[\vecb{0}]_2}}
=
g_{[\matr{A}_\mu]_1}([\vecb{\kappa}_\mu]_2)
=
g_{[\matr{B}]_1}\pmatri{
	{[\vecb{\kappa}_\mu]_2}\\
	{[\vecb{0}]_2}}
$$ 
and $\advA_g$ is successful.

As stated in Section \ref{sec:hash}, from $\advA_h$ we can construct an adversary $\advB_2$ that breaks the $q_\mathsf{gen}$-PPA assumption and from $\advA_g$ we can construct an adversary $\advB_3$ that breaks the $\mathcal{Q}^\top_m\mbox{-}\skermdh$ assumption, with the same advantages. We conclude that 
\begin{align*}
\Pr[\sfGame_1=1|\err_2=1] \leq & \adv_{{q_\mathsf{gen}}\mbox{-}\mathrm{PPA}}(\advB_2)+\adv_{\mathcal{Q}^\top_{q_\mathsf{gen}}\mbox{-}\skermdh}(\advB_3)
\end{align*}
\end{proof}

\begin{lemma}
$$
\Pr[\sfGame_3=1]\geq \frac{1}{{q_\mathsf{gen}}}\Pr[\sfGame_2=1].
$$
\end{lemma}
\begin{proof}
%\footnote{The analysis used in this proof uses similar techniques to Boneh and Franklin's analysis of identity based encryption \cite{C:BonFra01}, which in turn is inspired on Coron's analysis of full domain hash \cite{C:Coron00}.}
It holds that
\begin{align*}
\Pr[\sfGame_3=1] &= \Pr[\sfGame_3=1|\mathsf{err}_3=0]\Pr[\mathsf{err}_3=0]\\
&=\Pr[\sfGame_2=1|\mathsf{err}_3=0]\Pr[\mathsf{err}_3=0]\\
&=\Pr[\mathsf{err}_3=0|\sfGame_2=1]\Pr[\sfGame_2=1].
\end{align*}
The probability that $\mathsf{err}_3=0$ given $\sfGame_2=1$ is the probability that the ${q_\mathsf{cor}}$ calls to $\mathsf{Corrupt}$ do not abort and that $[vk]_2=[vk_{i^*}]_2$. Since $\advA$ is an eager adversary, at the $i$ th call to $\mathsf{Corrupt}$ the index $i^*$ is uniformly distributed over the ${q_\mathsf{gen}}-i+1$ indices of uncorrupted users. Similarly, when $\advA$ outputs its purported forgery, the probability that $[vk]_2=[vk_{i^*}]_2$ is $1/({q_\mathsf{gen}}-{q_\mathsf{cor}})$, since $[vk]_2\in R$ (or otherwise $\sfGame_2$ would have aborted). Therefore
$$
\Pr[\mathsf{err}_2=1|\sfGame_2=1]=\frac{{q_\mathsf{gen}}-1}{{q_\mathsf{gen}}}\frac{{q_\mathsf{gen}}-2}{{q_\mathsf{gen}}-1}\ldots\frac{{q_\mathsf{gen}}-{q_\mathsf{cor}}}{{q_\mathsf{gen}}-{q_\mathsf{cor}}+1}\frac{1}{{q_\mathsf{gen}}-{q_\mathsf{cor}}}=\frac{1}{{q_\mathsf{gen}}}.
$$ 
\end{proof}

\begin{lemma}  There exist adversaries $\advB_4$ and $\advB_5$ against the unforgeability of the one-time signature scheme and the weak unforgeability of the Boneh-Boyen signature scheme such that
$$
\Pr[\sfGame_3=1]\leq q_\mathsf{sig}\adv_{\mathsf{OT}}(\advB_4)+\adv_{\mathsf{BB}}(\advB_5)
$$
\end{lemma}
\begin{proof}
We construct adversaries $\advB_4$ and $\advB_5$ as follows.

$\advB_4$ receives $vk_\mathsf{ot}^\dag$ and simulates $\sfGame_3$ honestly but with the following differences. It chooses a random $j^*\in\{1,\ldots, q_\mathsf{sig}\}$ and answer the $j^*$ th query to $\mathsf{Sign}(i,m^\dag,R^\dag)$ honestly but computing $\sigma_\mathsf{ot}^\dag$ querying on $(m^\dag,R^\dag)$ its oracle and setting $vk_\mathsf{ot}^\dag$ as the corresponding one-time verification key. Finally, when $\advA$ outputs its purported forgery $(m,R,(\sigma_\mathsf{ot},vk_\mathsf{ot},\ldots))$, $\advB_4$ outputs the corresponding one-time signature.

$\advB_5$ receives $[vk]$ and simulates $\sfGame_3$ honestly but with the following differences. Let $i:=0$. $\advB_5$ computes $(sk_\mathsf{ot}^i,vk_\mathsf{ot}^i)\gets\mathsf{OT}.\mathsf{KeyGen}(gk)$, for each $1\leq i\leq q_\mathsf{sig}$ and queries its signing oracle on $(vk_{\mathsf{ot}}^1,\ldots,vk_\mathsf{ot}^{q_{\mathsf{sig}}})$ obtaining $[\sigma_1]_1,\ldots,[\sigma_{q_\mathsf{sig}}]_1$. On the $i^*$ th query of $\advA$ to the key generation algorithm, $\advB_5$ picks $\vecb{a}\gets\mathcal{Q}$ and outputs $\vecb{vk}:=([vk]_2,[\vecb{a}]_1,[\vecb{a}]_2,\vecb{a}[vk]_2)$. When $\advA$ queries the signing oracle on input $(i^*,m,R)$, $\advB_5$ computes an honest signature but replaces $vk_\mathsf{ot}$ with $vk_\mathsf{ot}^i$ and $[\sigma]_1$ with $[\sigma_i]_2$, and then adds 1 to $i$. Finally, when $\advA$ outputs its purported forgery $(m,R,(\sigma_\mathsf{ot},vk_\mathsf{ot},[\vecb{c}]_2,[\vecb{d}]_1,\ldots))$, it extracts $[\sigma]_1$ from $[\vecb{d}]_1$ as its forgery for $vk_\mathsf{ot}$.

Let $E$ be the event where $vk_\mathsf{ot}$, from the purported forgery of $\advA$, has been previously output by $\Sign$. We have that
$$
\Pr[\sfGame_3=1]\leq \Pr[\sfGame_3=1|E]+\Pr[\sfGame_3=1|\neg E].
$$
Since  $(m,R)$ has never been signed by a one-time signature and that, conditioned on $E$, the probability of $vk_\mathsf{ot}=vk_\mathsf{ot}^\dag$ is $1/q_\mathsf{sig}$, then
\begin{align*}
q_{\mathsf{sig}}\adv_\mathsf{OT}(\advB_4)\geq  \Pr[\sfGame_3=1|E]
\end{align*}
Finally, if $\neg E$ holds, then $[\sigma]$ is a forgery for $vk_\mathsf{ot}$ and thus
$$
\adv_{\mathsf{BB}}(\advB_5)\geq \Pr[\sfGame_3=1|\neg E]$$
\end{proof}
\end{proof}
