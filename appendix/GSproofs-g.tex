% !TEX root = ../main-ring-signature.tex

Given $[\vecb{a}]_2 = \beta[\vecb{u}_1]_1 + r[\vecb{u}_2]_1$, we construct a Groth-Sahai proofs that $\beta x=y$ even without knowledge of $\beta$. Let $s,t,\delta\gets\Z_q$, these proofs  consists of two vectors
\begin{align}
[&\vecb{\psi}]_2=-t[\vecb{w}_1]_2-\delta[\vecb{w}_2]_2 &[\vecb{\omega}]_1 =s[\vecb{a}]_1+\delta[\vecb{u}_2]_1
\label{eq:wi-proofs}
\end{align}
and commitments to $x$ and to $y$
\begin{align*}
&[\vecb{c}]_2= x[\vecb{w}_1]_2+s[\vecb{w}_2]_2, &[\vecb{d}]_1 =x[\vecb{a}_1]_2 + t[\vecb{u}_2]_2,
\end{align*}
satisfying the following verification equation
\begin{equation}
[\vecb{a}]_1[\vecb{c}^\top]_2-[\vecb{d}]_1[\vecb{w}_1]_2 =[\vecb{u}_2]_1[\vecb{\psi}^\top]_2+[\vecb{\omega}]_1[\vecb{w}^\top_2]_2.
\label{eq:wi-verif}
\end{equation}
These proofs and commitments can be re-randomized as follows
\begin{align}
&[\vecb{a}']_1 = [\vecb{a}]_1 + \gamma[\vecb{u}_2]_1 \nonumber\\
&[\vecb{c}']_2= [\vecb{c}]_2 + \alpha[\vecb{w}_2]_2, &[\vecb{d}']_1 =[\vecb{d}]_1 + \eta[\vecb{u}_2]_1 \nonumber\\
&[\vecb{\psi}']_2=[\vecb{\psi}]_2+\gamma[\vecb{c}']_2 - \eta[\vecb{w}_1]-\tau[\vecb{w}_2]_2, & [\vecb{\omega}']_1 = [\vecb{\omega}]_1 + \alpha[\vecb{a}]_1+\tau[\vecb{u}_2]_1,
\label{eq:wi-rerand}
\end{align}
for $\gamma,\alpha,\eta,\tau\gets\Z_q$.

We prove the following Lemma.
\begin{lemma} \label{lemma:betax=y}
Consider the the quadratic equation $\beta x = y$, whose variables are $\beta,x,y$. 
The proof system whose crs consists of perfectly binding Groth-Sahai commitment keys $[\matr{U}]_1,[\matr{W}]$, whose prover computes the proofs as in (\ref{eq:wi-proofs}), and the verifier verifies equation (\ref{eq:wi-verif}), is perfectly complete and sound, and computationally zero-knowledge under the SXDH assumption. Further, the proofs from (\ref{eq:wi-rerand}) follows exactly the same distribution as the proofs computed by the prover.
\end{lemma}
\begin{proof}
Completeness follows by inspection. 
Soundness follows from the fact that, whenever $\matr{U},\matr{W}$ come from the perfectly binding distribution, $\vecb{u}_1\vecb{w}_1^\top,\vecb{u}_1\vecb{w}_2^\top,\allowbreak\vecb{u}_2\vecb{w}_1^\top,\allowbreak\vecb{u}_2\vecb{w}_2^\top$ form basis of $\Z_q^{2\times 2}$.
The right side of equation (\ref{eq:wi-verif}) has no component in $\vecb{u}_1\vecb{w}^\top_1$, while the left side component is $\beta x-y$. Hence, we conclude that $\beta x = y$.

Computational zero-knowledge follows from the following argument.
When $[\matr{U}]_1,[\matr{W}]_2$ are sampled from the perfectly hiding distribution, $\vecb{u}_1 = \mu\vecb{u}_2$ and $\vecb{w}_1  = \nu\vecb{w}_2$, for some random $\mu$ and $\nu$. In this setting we can sample $\vecb{a} = r\vecb{u}_2$ without changing $\vecb{a}$'s distribution, and we can simulate the proofs for any $[\vecb{c}]_2,[\vecb{d}]_1$ as 
\begin{equation} \label{eq:sim-proofs}
[\vecb{\psi}]_2 = r[\vecb{c}]_2-\delta[\vecb{w}_2]_2\text{ and }[\vecb{\omega}]_1 = [\vecb{d}]_1\nu + \delta[\vecb{u}_2]_1\text{, for } \delta\gets\Z_q.
\end{equation} 
In particular, our simulator sets $[\vecb{c}]_2=s[\vecb{w}_2]_2,[\vecb{d}]_1=t[\vecb{u}_2]_1$ for $s,t\gets\Z_q$.

Note that both, the honest and the simulated proofs, follows exactly the same distribution. Indeed in both cases, $\vecb{a}$ and $\vecb{c},\vecb{d}$ are uniformly distributed in, respectively, $\Span(\vecb{u}_2)$ and $\Span(\vecb{w}_2)$, and the proofs are uniformly chosen among those that satisfy the respective verification equation.

Finally, is direct that the re-randomized proofs follow the same distribution as the real proofs.
Define $\tilde{r} := r + \gamma,\tilde{s} := s+\alpha,\tilde{t} := t+\eta, \tilde{\delta} := \delta+\tau+r\alpha$, which are uniformly distributed over $\Z_q$.
It follows by inspection that the re-randomized proof can be obtained as a real proof using $\tilde{r},\tilde{s},\tilde{t}$ and $\tilde{\delta}$ as random coins.
%If there is an adversary $\advA$ which tells apart a real proof from a simulated one, we can construct an adversary against the semantic security of ElGamal. Our adversary  runs $\advA$ until it outputs $\beta,[x]_2,[y]_2$ and request a ciphertexts $[\vecb{c}]_1$ which is the encryption of either $[x]_2$ or $0$. It defines $[\vecb{a}]_1=r[\vecb{u}_2]_1$,$ r\gets\Z_q$, computes $[\vecb{d}]_2 = \beta[\vecb{c}]_2 + \rho[\vecb{w}]_2$  and simulates $[\vecb{\psi}]_2,[\vecb{\omega}]_1$ as in (\ref{eq:sim-proofs}), gives $[\vecb{a}]_1,[\vecb{c}]_2,[\vecb{d}]_2,[\vecb{\psi}]_2,[\vecb{\omega}]_1$ to $\advA$, and outputs whatever $\advA$ outputs. Clearly, when $[\vecb{c}]_1$ encrypts $[x]_2$, the proof follows exactly the same distribution of an honestly computed proof, and when encrypts $0$ is distributed as te simulated proof.
\end{proof} 