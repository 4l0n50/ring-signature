We instantiate definition \ref{def:hash1} with the function $g$ and \ref{def:hash2}  with $h$ defined as follows. In the case of $g$, $\mathcal{Y}=\GG_2^2$, $\mathcal{M}=\GG_2^m$ and $\KGen$ picks a group description $gk\gets\ggen_a(1^\lambda)$ together with $[\matr{A}]_1\in\GG^{2\times m}$, where $\matr{A}\gets\mathcal{Q}_m$, and the function is defined as
$$
g_{[\matr{A}]_1}([\vecb{x}]_2):= [\matr{A}\vecb{x}]_2.
$$
Given a collision $[\vecb{x}]_2,[\vecb{x}']_2$ for $g$, then $([\vecb{x}]_2-[\vecb{x}']_2)\neq [\vecb{0}]_2$ is in the kernel of $[\matr{A} ]_1$. Therefore, is trivial to prove that for any adversary $\advA$ there is an adversary $\advB$ such that $\adv^{\mathsf{Col}_g}(\advA)=\adv_{\mathcal{Q}_m^\top\mbox{-}\akermdh}(\advB)$, whenever $\matr{A}\gets\mathcal{Q}_m$.


In the case of $h$, $\mathcal{Y}=\GG_2^1$, $\mathcal{M}=Q_m$, where
$$
Q_m := \left\{
(A,B)\subset\GG_1^{2}\times\GG_2:
\begin{array}{l} 
{|A|=|B|=m\text{ and }}\\
{\forall ([\vecb{a}]_1,[b]_2)\in A,e([a_2]_1,[1])=e([a_1]_1,[b]_2)}
\end{array}
\right\},
$$ and $\KGen=\ggen_s$, and the function is defined as
$$
h(A,B):= \sum_{[\vecb{a}]_1\in A}[\vecb{a}]_1.
$$
Given $(A,B)\gets Q_m$ and a second preimage $(A',B')\in Q_m$ of $h(A,B)$, it is trivial to construct an adversary breaking the $m$-PPA assumption. Indeed, given $[\matr{A}]_1,[\vecb{b}]\in\GG_1^{2\times m}\times \GG_2^{1\times m}$ the challenge of the $m$-PPA assumption, then any matrix $[\matr{A}']$ and any row vector $[\vecb{b}']$ whose columns are the elements of, respectively, $A'$ and $B$, breaks $m$-PPA assumption. Then for any adversary $\advA$ there is an adversary $\advB$ such that $\adv^{\mathsf{aPre}_g}(\advA)=\adv_{m\mbox{-}\mathsf{PPA}}(\advB)$.

We note that given $(A,B)\in Q_m,[\matr{A}]_1\in\GG^{2\times m},[\vecb{b}]_2\in\GG^{1\times m},[\vecb{x}]_2\in\GG^m$ and $[\vecb{y}]_1\in\GG_1^2,[\vecb{z}]_2\in\GG_2^2$ one can express the statements $A\in Q_m$, $g_{[\matr{A}]_1}([\vecb{x}]_2)=[\vecb{z}]_2$, and $h(A)=[\vecb{y}]_1$ as equations (\ref{eq:Q}),(\ref{eq:g}), and (\ref{eq:h}), respectively.
 \begin{align}
&\sum_{i=1}^n e([a_{2,i}]_1,[1]_2)=e([a_{1,i}]_1,[b_{i}]_2) \text{ and }\nonumber\\
&e([a_{i,1}]_1,[1]_2)=e([1]_1,[b_i]_2) \text{ for each } 1\leq i\leq m \label{eq:Q}\\
&\sum_{j=1}^n e([a_{i,j}]_1,[x_i]_2) = e([1]_1,[y_i]_2) \text{ for each } i\in\{1,2\} \label{eq:g}\\
&\sum_{[\vecb{a}]\in A} [a_i]_1 = [y_i]_1 \text{ for each } i\in\{1,2\}.\label{eq:h}
\end{align}
Thus, one can compute Groth-Sahai proofs of size $\Theta(m),\Theta(1)$, and $\Theta(1)$, respectively, for the satisfiability of each statement.
%
%Finally, we prove a simple lemma informally stated in Section \ref{sec:tech-overview}.
%\begin{lemma}\label{lemma:hg}
%Let $A\gets Q_m,A'\in Q_m,[\vecb{x}],[\vecb{x}']\in\GG^m$, and $[\matr{A}],[\matr{A}']$ the matrices whose columns are the elements of $A$ and $A'$, respectively. Then $h(A)=h(A')$ and $g_{[\matr{A}]}([\vecb{x}])=g_{[\matr{A}']}([\vecb{x}'])$ implies that $A'$ is a second preimage of $h(A)$ or there exists a permutation matrix $\matr{P}$ such that $g_{[\matr{A}]}([\vecb{x}])=g_{[\matr{A}]}([\matr{P}\vecb{x}'])$.
%\end{lemma}
%\begin{proof}
%If $A\neq A'$, then $A'$ is a second preimage of $h(A)$. Else, there is a permutation matrix $\matr{P}$ such that $[\matr{A}'] =[\matr{A}\matr{P}]$. Then
%$$
% g_{[\matr{A}]}([\vecb{x}])=g_{[\matr{A}']}([\vecb{x}'])\Longleftrightarrow  g_{[\matr{A}]}([\vecb{x}])=g_{[\matr{A}\matr{P}]}([\vecb{x}'])=g_{[\matr{A}]}([\matr{P}\vecb{x}']).
%$$
%\end{proof}