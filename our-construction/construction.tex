% !TEX root = ../main-ring-signature.tex

In the following let $n:=|R|, m:=\sqrt[3]{n}$, and for $1\leq \alpha\leq n$ define $1\leq \mu \leq n^{2/3}$ and $1\leq \nu\leq m$ such that $\alpha=(\mu-1)m+\nu$. For a sequence $\{s\}_{1\leq i\leq n}$ we define $s_{\mu,\nu}:=s_{(\mu-1)m+\nu}$. Consider $\mathsf{OT}=(\mathsf{OT}.\KG,\mathsf{OT}.\mathsf{Sign},\allowbreak\mathsf{OT}.\mathsf{Ver})$ a one-time signature scheme.

\begin{description}
\item[$\mathsf{CRSGen}(gk)$:] Pick a perfectly hiding CRS for the Groth-Sahai proof system $\crs_\GS$ and define $(ck_1,ck_2):=\crs_\GS$. Note that $\crs_\GS$ can be also used for the $\Theta(\sqrt{n})$ set-membership of Chandran et al. The CRS is $\rho:=(gk,\crs_\GS).$

\item[$\KG(\rho)$:] Pick $\vecb{a}\gets\mathcal{Q}$ and $(sk,[vk]_2)\gets\mathsf{BB}.\KG(gk)$, compute $[\vecb{a}]_1$, $[\vecb{a}]_2$ and then erase $\vecb{a}$ (but if not erased we prove security under the $(\ell,m)$-PPA). The secret key is $sk$ and the verification key is $\vecb{vk}:=([vk]_2,[\vecb{a}]_1,[\vecb{a}]_2,\vecb{a}[vk]_2)$.

\item[$\mathsf{Sign}_{\rho,sk}(m,R)$:] Let $\alpha$ the index of the signer with respect to $R$.
\begin{enumerate}
\item Compute $(sk_\mathsf{ot},vk_\mathsf{ot})\gets\mathsf{OT}.\KG(gk)$ and $\sigma_\mathsf{ot}\gets\allowbreak\mathsf{OT}.\allowbreak\mathsf{Sign}_{sk_\mathsf{ot}}(m,R)$.

\item Compute $[\vecb{c}]_2:=\GS.\Com_{ck_2}([vk_\alpha]_2;\vecb{r})$, $\vecb{r}\gets\Z_q^2$, $[\sigma]_1\gets\mathsf{BB}.\mathsf{Sign}_{sk_\alpha}(vk_\mathsf{ot})$, $[\vecb{d}]_1:=\GS.\Com_{ck_1}([\sigma]_1;\vecb{s})$, $\vecb{s}\gets\Z_q^2$, and a GS proof $\pi_\mathsf{BB}$ that $\mathsf{BB}.\mathsf{Ver}_{[vk]_2}(\allowbreak[\sigma]_1,vk_\mathsf{ot})=1$.

\item For $1\leq i \leq n^{2/3}$, let $[\vecb{\kappa}_i]_2=([vk_{i,1}]_2,\ldots,[vk_{i,m}]_2)^\top$, $A_i=\{([\vecb{a}_{i,1}]_1,[\vecb{a}_{i,1}]_2),\allowbreak\ldots,\allowbreak([\vecb{a}_{i,m}]_1,[\vecb{a}_{i,m}]_2)\}$, and $[\matr{A}_i]_1:=[\vecb{a}_{i,1}\cat\cdots\cat\vecb{a}_{i,m}]_1$ . Define the sets
$H=\{h(A_1),\allowbreak\ldots,\allowbreak h(A_{n^{2/3}})\}$ and
$G=\{
	g_{[\matr{A}_1]_1}([\vecb{\kappa}_1]_2)
	\allowbreak\ldots,\allowbreak
	g_{[\matr{A}_{n^{2/3}}]_1}([\vecb{\kappa}_{n^{2/3}}]_2)\}$.

\item Let $[\vecb{x}]_1:=h(A_\mu)$ and $[\vecb{y}]_2=g_{[\matr{A}_\mu]_1}([\vecb{\kappa}_\mu]_2)$. Compute GS commitments to $[\vecb{x}]_1$ and $[\vecb{y}]_2$ and compute proofs $\pi_G$ and $\pi_H$ that they belong to $G$ and $H$, respectively. It is also proven that they appear in the same positions reusing the commitments to $b_1,\ldots,b_{m}$ and $b'_1,\ldots,b'_{m}$, used in the set-membership proof of Chandran et al., which define $[\vecb{x}]_1$'s and $[\vecb{y}]_2$'s position in $H$ and $G$ respectively.

\item Let
$
[\vecb{\kappa'}]_2:=([vk_\alpha]_2,[vk_{\mu,1}]_2,\ldots,[vk_{\alpha-1}]_2,[vk_{\alpha+1}]_2,\ldots,[vk_{\mu,m}]_2)^\top\in\GG^m_2$, 
$[\matr{A}']_1:=[\vecb{a}_\alpha \cat \vecb{a}_{\mu,1} \cat \cdots \cat \vecb{a}_{\alpha-1}\cat \vecb{a}_{\alpha+1}\cat\cdots\cat\vecb{a}_{\mu,m}]_1\in\GG^{2\times m}_1$ and
$A'=\{([\vecb{a}_{\mu,1}]_1,[\vecb{a}_{\mu,1}]_2),\ldots,([\vecb{a}_{\mu,1}]_1,\allowbreak [\vecb{a}_{\mu,1}]_2)\}$.
Compute GS commitments to all but the first element of $[\vecb{\kappa}']_2$ (note that $[\vecb{c}]_2$ is a commitment to the first element of $[\vecb{\kappa}']_2$). Compute also a GS proof $\pi_g$ that $g_{[\matr{A}']_1}([\vecb{\kappa}']_2)=[\vecb{y}]_2$, a GS proof $\pi_{h}$ that $h(A')=[\vecb{x}]_1$, and a GS proof $\pi_{Q_m}$ that $A'\in Q_m$.

\item Return the signature $\grkb{\sigma}:=(vk_\mathsf{ot},\sigma_\mathsf{ot},[\vecb{c}]_2,[\vecb{d}]_1,\pi_{\mathsf{BB}},\pi_G,\pi_H, \pi_g,\pi_h,\pi_{Q_m})$. (GS proofs include commitments to variables).
\end{enumerate}

\item[$\mathsf{Verify}_{\rho,R}(m,\grkb{\sigma})$:] Verify the validity of the one-time signature and of all the proofs. Return 0 if any of these checks fails and 1 otherwise.
\end{description}

We prove the following theorem which states the security of our construction.

\begin{theorem}\label{theo:security}
The scheme presented in this section is a ring signature scheme
with perfect correctness, perfect anonymity and computational unforgeability under the
$Q_\mathsf{gen}$-permutation pairing assumption, the $\mathcal{Q}_{Q_\mathsf{gen}}^\top\mbox{-}\skermdh$ assumption, the $\mathrm{SXDH}$ assumption, and the assumption
that the one-time signature and the Boneh-Boyen signature are unforgeable.
Concretely, for any PPT adversary $\advA$ against the unforgeability of the scheme, there exist adversaries $\advB_1,\advB_2,\advB_3,\advB_4,\advB_5$ such that
\begin{align*}
\adv(\advA)\leq &\adv_{\mathrm{SXDH}}(\advB_1)+\adv_{Q_\mathsf{gen}\mbox{-}\mathrm{PPA}}(\advB_2)+\adv_{\mathcal{Q}^\top_{Q_\mathsf{gen}}\mbox{-}\skermdh}(\advB_3)+\\
&Q_\mathsf{gen}(Q_\mathsf{sign}\adv_{\mathsf{OT}}(\advB_4)+\adv_{\mathsf{BB}}(\advB_5)),
\end{align*}
where $Q_\mathsf{gen}$ and $Q_\mathsf{sign}$ are, respectively, upper bounds for the number of queries that $\advA$ makes to its $\mathsf{VKGen}$ and $\mathsf{Sign}$ oracles.
\end{theorem}