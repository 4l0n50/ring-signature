% !TEX root = ../main-ring-signature.tex

We instantiate definition \ref{def:hash1} with the function $g$ and \ref{def:hash2}  with $h$ defined as follows. For  $h$, $\mathcal{M}=\mathcal{Q}_m$, $\mathcal{Y}=[m]$, $\KGen$ returns the description of an asymmetric group plus ElGamal encryption keys $[\vecb{u}]_1$  and $[\vecb{v}]_2$ together with the decryption keys, respectively, $\vecb{k},\vecb{k}'$ 
\begin{align*}
&h([\matr{A}]_1,[\matr{B}]_2):= \sum_{i=1}^m[\vecb{a}_i]_1,\\
&\mathcal{Q}_m := \left\{
\begin{array}{c}
([\matr{A}]_1,[\matr{B}]_2) \in \GG_1^{2\times m}\times\GG_2^{2\times m}:
\exists \vecb{b} \in\bits^m,\vecb{r},\vecb{s}\in\Z_q^m\text{ s.t. }\\
{\matr{A} = \pmatri{0\\\vecb{b}^\top}+\vecb{u}\vecb{r}^\top, \matr{B} = \pmatri{0\\\vecb{b}^\top}+\vecb{v}\vecb{s}^\top},
\end{array}\right\}.\\
\end{align*}
Note that, if $A=(\vecb{a}_1\cat\cdots\cat\vecb{a}_m),\matr{B}=(\vecb{b}_1\cat\ldots\cat\vecb{b}_m)$, then $[\vecb{a}_i]$ and $[\vecb{b}_i]$ are, respectively, ElGamal encryptions $\Enc_{[\vecb{u}]_1}(b_i;r_i)$ and $\Enc_{[\vecb{v}]_2}(b_i;s_i)$.

It might seem odd to define $\mathcal{Q}_m$ as a set of pairs of matrices in both groups while $h$ only require elements in one group. However, this will be crucial in the security proof of our ring signature, where we need to compute $[vk \vecb{b}]_2$, for some $vk\in\Z_q$,
from  $[\vecb{b}]_2$ without knowledge of its discrete logarithm. For simplicity, we may just write $h([\matr{A}]_1)$ for $[\matr{A}]_1\in \GG_1^{2\times m}$ (which is still well defined).

%Although is hard to compute $h$, is easy to show that $h(A)=h(A')$ whenever elements of $A'$ are re-randomizations of the elements of $A$. That is, for each $\vecb{a}'\in A'$ there exists a unique  $\vecb{a}\in A$ and $\delta\in\Z_q$ such that $\vecb{a}' = \vecb{a}+\delta\vecb{u}$. Given all the random coins used for rerandomization, the statement $h(A)=h(A')$ can be proved using Groth-Sahai.

Note that the function $h$ outputs an encryption of the hamming weight of $b_1,\ldots,b_m$. Although is possible to find collisions, two matrices are a collision if they encrypt bitstrings with the same hamming weight or equivalently, the decryptions are equal up to a permutation. Consequently, to restore collision resistance, we weaken the equality relation to the following equivalence relation
$$
\matr{A}' \equiv_\vecb{u} \matr{A} \iff \exists \matr{P}\in\mathcal{S}_m, \vecb{\delta}\in\Z_q^m \text{ s.t. } \matr{A}' = \matr{A}\matr{P} + \vecb{u}\vecb{\delta}^\top,
$$
where $\mathcal{S}_m\subset \bits^{m\times m}$ is the set of permutation matrices of size $m\times m$.

Indeed, let $\matr{A}=\smallpmatrix{0\\\vecb{b}^\top}+\vecb{u}\vecb{r}^\top,\matr{A}'=\smallpmatrix{0\\{\vecb{b}'}^\top}+\vecb{u}\vecb{\delta}^\top$ such that $h([\matr{A}]_1)=h([\matr{A}']_1)$. Since $\vecb{b}$ and $\vecb{b}'$ have the same hamming weigh, then there exists a permutation matrix $\matr{P}$ such that $\vecb{b}^\top\matr{P} = {\vecb{b}'}^\top$ and hence
\begin{align*}
\matr{A}' &= \pmatri{0\\\vecb{b}^\top\matr{P}} + \vecb{u}\vecb{\delta} + \vecb{u}\vecb{r}^\top\matr{P} - \vecb{u}\vecb{r}^\top\matr{P}\\
&=
 \left(\pmatri{0\\\vecb{b}^\top} + \vecb{u}\vecb{r}^\top \right)\matr{P} + \vecb{u}(\vecb{\delta}^\top-\vecb{r}^\top\matr{P})\\
 &=
 \matr{A}\matr{P} + \vecb{u}\tilde{\vecb{\delta}}^\top
\end{align*}
%Given a second preimage $h$, it is trivial to construct an adversary breaking the $m$-PPA assumption. Indeed, Let $[\matr{A}]_1,[\matr{A}]_2$ the challenge of the $m$-PPA assumption and let $A$ the set of columns of $[\matr{A}]_1$ and $[\matr{A}]_2$, which is clearly uniformly distributed in $Q_m$. Then given any $A'\in Q_m$ such that $A'\neq A$ and $h(A)=h(A')$, it holds that $[\matr{A}']_1$, the matrix whose columns are the first components of the elements of $A'$, is not a permutation of $[\matr{A}]_1$ and hence breaks $m$-PPA assumption. Then for any adversary $\advA$ there is an adversary $\advB$ such that $\adv^{\mathsf{aPre}_g}(\advA)=\adv_{m\mbox{-}\mathsf{PPA}}(\advB)$. 

In the case of $g$, $\mathcal{M}=\GG^m_2$, $\mathcal{Y}=\GG_2^2$, and $\KGen$ picks a group description $gk\gets\ggen_a(1^\lambda)$ together with $[\matr{A}]_1$ such that
$$
\matr{A}  = \pmatri{0 & \cdots & 0\\ b_1 & \cdots & b_m} +\vecb{u}\vecb{r}^\top \in\Z^{2\times m}_q,
$$
 where $\vecb{r}\gets\Z_q^m$ and $b_1,\ldots,b_m\in\bits$. The function is defined as
$$
g_{[\matr{A}]_1}([\vecb{x}]_2):= \vecb{k}^\top\matr{A}\vecb{x}=(\matr{B}\vecb{x})^\top\vecb{k}=\sum_{i=1}^m b_ix_i.
$$
Although not efficiently computable, one can prove that 
 $g_{[\matr{A}]_1}([\vecb{x}]_2) = g_{[\matr{A}]_1}([\vecb{x}']_2)$
 using the homomorphic properties of Groth-Sahai proofs. If we additionally provide proofs that $b_ix_i=x_ib'_i$, where $b_i'$ is the decryption of $\vecb{b}_i$, from which it migth be computed a proof that $g_\matr{A}(\vecb{x})=\sum_{i=1}^m b_ix_i$.

Note that in asymmetric groups the order of multiplicads in te expressions $b_ix_i$ and $x_ib'_i$ is relevant since it imply that commitments (or encryption) to $b_i$ and $b'_i$ live, respectively, in $\GG_1$ and $\GG_2$. The Groth-Sahai proof that two commitments in different groups open to the same value is obtained equivalently as a proofs $[\vecb{\pi}_i]_1,[\vecb{\theta}_i]_2$ such that
$$ 
[\vecb{a}_ix_i]_1[\vecb{v}^\top]_2-[\vecb{u}]_1[(\vecb{b}_ix_i)^\top]_2 =[\vecb{\pi}_i]_1[\vecb{v}^\top]_2+[\vecb{u}]_1[\vecb{\theta}^\top_i]_2.
$$
A proof that $g_\matr{A}(\vecb{x}) = g_\matr{A}(\vecb{x}')$ can be derived from the addition of the previous proof plus re-randomization for keeping the proof zero-knowledge. That is
$$
[\vecb{\pi}]_1 := \sum_{i=1}^m [\vecb{\pi}_i]_1+\delta[\vecb{u}]_1
\text{ and }
[\vecb{\theta}]_2 := \sum_{i=1}^m [\vecb{\theta}_i]_2-\delta[\vecb{v}]_2.
$$
such that
$$
\left(\sum_{i=1}^m [\vecb{a}_ix_i]_1\right)[\vecb{v}^\top]-[\vecb{u}]_1\left(\sum_{i=1}^m[\vecb{b}_ix_i]_2\right)^\top =
[\vecb{\pi}]_1[\vecb{v}^\top]_2 + [\vecb{u}_1][\vecb{\theta}_i^\top]_2
$$ Consequetly, in our scheme we will publish  $[\vecb{b}_ix_i]_2$ and the proofs $[\vecb{\pi}_i]_1,[\vecb{\theta}_i]_2$ which will render possible to prove $g_\matr{A}(\vecb{x})=g_\matr{A}(\vecb{x}')$.

Given a collision $[\vecb{x}]_2,[\vecb{x}']_2$ for $g$, then $([\vecb{x}]_2-[\vecb{x}]'_2)\neq [\vecb{0}]$ is in the kernel of $[\matr{A}]_1$. Therefore, is trivial to prove that for any adversary $\advA$ there is an adversary $\advB$ such that $\adv^{\mathsf{Col}_g}(\advA)=\adv_{\mathcal{Q}_m^\top\mbox{-}\skermdh}(\advB)$, whenever $\matr{A}\gets\mathcal{Q}_m$.

We note that given $A\in Q_m,[\matr{A}]_1\in\GG^{2\times m}_1,[\vecb{x}]_2\in\GG^m_2$, $[\vecb{y}]_1\in\GG_2^2$ and $[\vecb{y}']_1\in\GG^1_2$ one can express the statements $A\in Q_m$, $g_{[\matr{A}]_1}([\vecb{x}]_2)=[\vecb{y}]_2$, and $h(A)=[\vecb{y}']_1$ as (\ref{eq:Q}),(\ref{eq:g}), and (\ref{eq:h}), respectively.
 \begin{align}
&e([a_{1}]_1,[1]_2) = e([1]_1,[{b}_1]_2)\text{ and }e([a_{2}]_1,[1]_2)=e([a_{1}]_1,[b_1]_2)
\text{ for each }([\vecb{a}]_1,[\vecb{b}]_2)\in A \label{eq:Q}\\
&\sum_{j=1}^m e([a_{i,j}]_1,[x_i]_1) = e([1]_1,[y_i]_2) \text{ for each } i\in\{1,2\} \label{eq:g}\\
&\sum_{([\vecb{a}]_1,[\vecb{a}]_2)\in A} [a_i]_1 = [y'_i]_1 \text{ for each } i\in\{1,2\}.\label{eq:h}
\end{align}
Hence, one can compute Groth-Sahai proofs of size $\Theta(m),\Theta(1)$, and $\Theta(1)$, respectively, for the satisfiability of each statement.

Finally, we prove a simple lemma that relates both functions
\begin{lemma}\label{lemma:hg}
Let $A\gets Q_m,A'\in Q_m,[\vecb{x}]_2,[\vecb{x}']_2\in\GG^m_2$, and $[\matr{A}]_1,[\matr{A}']_1$ the matrices whose columns are the first component of the elements of $A$ and $A'$, respectively. Then $h(A)=h(A')$ and $g_{[\matr{A}]_1}([\vecb{x}]_2)=g_{[\matr{A}']_1}([\vecb{x}']_2)$ implies that $A'$ is a second preimage of $h(A)$ or there exists a permutation matrix $\matr{P}$ such that $g_{[\matr{A}]_1}([\vecb{x}]_2)=g_{[\matr{A}]_1}([\matr{P}\vecb{x}']_2)$.
\end{lemma}
\begin{proof}
If $A\neq A'$, then $A'$ is a second preimage of $h(A)$. Else, there is a permutation matrix $\matr{P}$ such that $[\matr{A}']_1 =[\matr{A}\matr{P}]_1$. Then
$$
 g_{[\matr{A}]_1}([\vecb{x}]_2)=g_{[\matr{A}']_1}([\vecb{x}']_2)\Longleftrightarrow  g_{[\matr{A}]_1}([\vecb{x}]_2)=g_{[\matr{A}\matr{P}]_1}([\vecb{x}']_2)=g_{[\matr{A}]_1}([\matr{P}\vecb{x}']_2).
$$
\end{proof}