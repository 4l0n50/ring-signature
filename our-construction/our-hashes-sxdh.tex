% !TEX root = ../main-ring-signature.tex

We instantiate definitions \ref{def:hash1} and \ref{def:hash2}  with, respectively, the functions $g$ and $h$ defined in this section. Function $h$ is unconditionally collision resistant, under a relaxation of the notion of equality of purported collisions, while the latter function is collision resistant under the SXDH assumption.

%We additionally show how to compute zero-knowledge proofs of membership in the message space of $h$ and of satisfiability of $g_k(x)=g_k(x')$.
%We obtain shorter proofs replacing some commitments by ElGamal encryptions as done in \cite{PKC:EscGro14}. Although the security of this shorter proofs is shown in \cite{PKC:EscGro14}, for completeness we include in Appendix XXX the security proofs of our instantiations of the general framwork described in \cite{PKC:EscGro14}.

Our constructions are inspired on much simpler instantiation of this functions which we can only prove secure under a stronger assumption, the Permutation Pairing assumption introduced by Groth and Lu \cite{AC:GroLu07}. This simpler instantiation can be found in Appendix XXX. 

\subsubsection{The function $h$.} Consider Groth-Sahai commitment keys  $[\matr{U}]_1 = [\vecb{u}_1|\vecb{u}_2]_1\in\GG_1^{2\times 2},[\matr{V}]_2=[\vecb{v}_1|\vecb{v}_2]_2\in\GG_2^{2\times 2}$. The message space of $h$ a set of matrices $\mathcal{Q}_m$ defined as follows
\begin{align*}
&\mathcal{Q}_m^\vecb{\beta} := \left\{
\begin{array}{c}
\matr{A} \in \Z_q^{2\times m}:
\exists \vecb{r}\in\Z_q^m\text{ s.t. }
{\matr{A} = \matr{U}\pmatri{\vecb{\beta}^\top\\\vecb{r}^\top}
%\matr{B} = \pmatri{0\\\vecb{\beta}^\top}+\vecb{v}\vecb{s}^\top
}
\end{array}\right\},
\ \mathcal{Q}_m := \cup_{\vecb{\beta}\in\bits^m} \mathcal{Q}_m^\vecb{\beta},\\
&\text{and } h(\matr{A}):= \sum_{i=1}^m\vecb{a}_i,
\end{align*}
and the space for hashes is $\mathcal{Y}=\Z_q^2$. The function is simply $h(\matr{A})=\sum_{i=1}^m \vecb{a}_i$.

If $A=(\vecb{a}_1\cat\cdots\cat\vecb{a}_m)\in\mathcal{Q}_m$, then $[\vecb{a}_i]_1$ is a Groth-Sahai commitment $\Com_{[\matr{U}]_1}(\beta_i;r_i)$.
Therefore $[h(\matr{A})]_1$ outputs a Groth-Sahai commitment to the hamming weight of $\vecb{\beta}$, that is $[h(\matr{A})]_1 = \Com_{[\matr{U}]_1}(\sum_{i=1}^m \beta_i;\sum_{i=0}^m r_i)$. Although is possible to find collisions, it happens only if the matrices open to bitstrings with the same hamming weight or equivalently, the openings are equal up to a permutation. Consequently, to restore collision resistance, we weaken the equality relation to the following equivalence relation
$$
\matr{A}' \equiv_\vecb{u} \matr{A} \iff \exists \matr{P}\in\mathcal{S}_m, \vecb{\delta}\in\Z_q^m \text{ s.t. } \matr{A}' = \matr{A}\matr{P} + \vecb{u}_2\vecb{\delta}^\top,
$$
where $\mathcal{S}_m\subset \bits^{m\times m}$ is the set of permutation matrices of size $m$.

Whenever $\matr{U}$ is chosen from the perfectly binding distribution, is impossible to find collisions $[\matr{A}]_1,[\matr{A}]_1'\in[\mathcal{Q}_m]_1$ such that $\matr{A}\not\equiv_\vecb{u}\matr{A}'$ and $h(\matr{A})=h(\matr{A}')$. Given $\matr{A},\matr{A}'\in\mathcal{Q}_m$ we may assume that $\matr{A}=\matr{U}\smallpmatrix{\vecb{\beta}^\top\\\vecb{r}^\top},\matr{A}'=\matr{U}\smallpmatrix{{\vecb{\beta}'}^\top\\\vecb{\delta}^\top}$. Since $\vecb{\beta}$ and $\vecb{\beta}'$ must have the same hamming weigh there exists a permutation matrix $\matr{P}$ such that $\vecb{\beta}^\top\matr{P} = {\vecb{\beta}'}^\top$, and hence
\begin{align*}
\matr{A}' &= \matr{U}\pmatri{\vecb{\beta}^\top\matr{P}\\\vecb{\delta}^\top}+ \vecb{u}_2\vecb{r}^\top\matr{P} - \vecb{u}_2\vecb{r}^\top\matr{P}\\
&=
\matr{U} \pmatri{\vecb{\beta}^\top\\\vecb{r}^\top}\matr{P} + \vecb{u}_2(\vecb{\delta}^\top-\vecb{r}^\top\matr{P})\\
 &=
 \matr{A}\matr{P} + \vecb{u}_2\tilde{\vecb{\delta}}^\top.
\end{align*}
When $\matr{U}$ is chosen from the perfectly hiding distribution, it turns out that $\mathcal{Q}_m$ becomes $\mathcal{Q}_m^{0^m}$, which is in fact $\Z_q^{2\times m}$, and there doesn't exist $\matr{A},\matr{A}'$ such that $\matr{A}\not\equiv_{\matr{U}}\matr{A}'$. 

Further, is possible to efficiently prove membership in $\mathcal{Q}_m$. It suffices to give Groth-Sahai proofs for
\begin{equation}
\beta(1-\beta')=0\text{ and }\beta=\beta' \label{eq:Qm-memb},
\end{equation}
for $\beta=\beta_1,\ldots,\beta_m,\beta'=\beta_1,\ldots,\beta_m$.

To do so we compute a commitment to $\beta'$, $[\vecb{b}]_2 = \beta'[\vecb{v}_1]_2+\rho[\vecb{v}_2]_2$ and proofs
\begin{align}
&[\vecb{\theta}]_2 = r([\vecb{v}_1]_2-[\vecb{b}]_2)+\delta[\vecb{v}_2]_2
&[\vecb{\pi}]_1 = \beta\rho[\vecb{u}_1]_1-\delta[\vecb{u}_2]_1 \nonumber\\
&[\vecb{\xi}]_2 = r[\vecb{v}_1]_2+\delta'[\vecb{v}_2]_2
&[\vecb{\phi}]_2 = \rho[\vecb{u}_1]_1 -\delta'[\vecb{u}_2]_1.
\label{eq:Qm-memb-proofs}
\end{align}
Such proofs satisfy the following verification equations
\begin{align}
&[\vecb{a}]_1([\vecb{v}_1]_2-[\vecb{b}]_2)^\top = [\vecb{u}_2]_1[\vecb{\theta}^\top]_2+[\vecb{\pi}]_1[\vecb{v}^\top_2]_2\text{ and } \label{eq:Qm-memb-verif1} \\
&[\vecb{a}]_1[\vecb{v}_1^\top]_2-[\vecb{u}_1]_1[\vecb{b}^\top]_2 = [\vecb{u}_2]_1[\vecb{\xi}^\top]_2 + [\vecb{\phi}]_1[\vecb{v}^\top_2]_2 \label{eq:Qm-memb-verif2}
\end{align}

 The following Lemma formally states the security of the previous proofs.
\begin{lemma} \label{lemma:Qm-memb}
Consider the quadratic equation $\beta(1-\beta') = 0$ and the linear equation $\beta = \beta'$, whose variables are $\beta$ and $\beta'$. 
The proof system whose crs consists Groth-Sahai perfectly binding commitment keys $[\matr{U}]_1,[\matr{V}]_2$, whose prover computes the proofs as in (\ref{eq:Qm-memb-proofs}), and the verifier verifies equations (\ref{eq:Qm-memb-verif1}) and (\ref{eq:Qm-memb-verif2}), is perfectly complete and sound, and computationally zero-knowledge under the SXDH assumption.
\end{lemma}
\begin{proof}
Completeness follows by inspection. Soundness follows from the fact that, whenever $\matr{U},\matr{V}$ come from the perfectly binding distribution, $\vecb{u}_1\vecb{v}_1^\top,\vecb{u}_1\vecb{v}_2^\top,\allowbreak\vecb{u}_2\vecb{v}_1^\top,\allowbreak\vecb{u}_2\vecb{v}_2^\top$ form basis of $\Z_q^{2\times 2}$. Since in equations (\ref{eq:Qm-memb-verif1}) and (\ref{eq:Qm-memb-verif2}) the right sides have no components in $\vecb{u}_1\vecb{v}_1$ and left sides components are, respectively, $\beta(1-\beta')$ and $\beta-\beta'$, we conclude that $\beta(1-\beta') = 0$ and $\beta=\beta'$.

Finally, computational zero-knowledge follows from the following argument.
When $[\matr{U}]_1,[\matr{V}]_2$ are sampled from the perfectly hiding distribution, $\vecb{u}_1 = \mu\vecb{u}_2$ and $\vecb{v}_1 = \nu\vecb{v}_2$, for some random $\mu$ and $\nu$. In this setting we can sample $\vecb{a} = r\vecb{u}_2$ without changing $\vecb{a}$'s distribution, and we can simulate the proofs for any $[\vecb{b}]_2$ as 
\begin{align} \label{eq:Qm-sim-proofs}
&[\vecb{\theta}]_2 = r([\vecb{v}_1]_1-[\vecb{b}]_2)+\delta[\vecb{v}_2]_2
&[\vecb{\pi}]_1 = -\delta[\vecb{u}_2]_1 \nonumber\\
&[\vecb{\xi}]_2 = r[\vecb{v}_1]_2-\mu[\vecb{b}]_2+\delta'[\vecb{v}_2]_2
&[\vecb{\phi}]_2 = -\delta'[\vecb{u}_2]_1.
\end{align} 
In particular, our simulator considers $[\vecb{b}]_2=\rho[\vecb{v}_2]_2$, for $\rho\gets\Z_q$, which is follows exactly the same distribution as in the honest proof.

Note that both, the honest and the simulated proofs, follows exactly the same distribution. Indeed in both cases, $\vecb{a}$ and $\vecb{b}$ are uniformly distributed in, respectively, $\Span(\vecb{u}_2)$ and $\Span(\vecb{v}_2)$, and the proofs are uniformly chosen among those that satisfy the respective verification equation.
%If there is an adversary $\advA$ which tells apart a real proof from a simulated one, we can construct an adversary against the semantic security of ElGamal. Our adversary  runs $\advA$ until it outputs solution $\beta$ and request a ciphertext $[\vecb{b}]_2$ to its oracle, which returns the encryption of either $\beta$ or $0$. It defines $[\vecb{a}]_1=r[\vecb{u}_2]_1$,$ r\gets\Z_q$,   and simulates $[\vecb{\theta}]_2,[\vecb{\pi}]_1,[\vecb{\xi}]_2,[\vecb{\phi}]_1$ as in (\ref{eq:Qm-sim-proofs}), gives $[\vecb{a}]_1,[\vecb{b}]_2$ together with the simulated proofs to $\advA$, and outputs whatever $\advA$ outputs. Clearly, when $[\vecb{b}]_1$ encrypts $\beta$, the proof follows exactly the same distribution of an honestly computed proof, and when encrypts $0$ is distributed as the simulated proofs.
\end{proof} 

\subsubsection{The function $g$.} Consider perfectly binding Groth-Sahai commitment keys $[\matr{U}]_1=[\vecb{u}_1\cat\vecb{u}_2]_1,[\matr{V}]_2 = [\vecb{v}_1\cat\vecb{v}_2]_2$, together with a trapdoor for the commitment scheme $\vecb{k}\in\Z_q^2$ such that $\vecb{k}^\top\vecb{u}_2=0$ and $\vecb{k}^\top\vecb{u}_1=1$, consider also $[\matr{A}]_1$ such that $\matr{A}\gets\mathcal{Q}_m^\vecb{\beta}$. The function $g$ is defined as
$$
g_{\matr{A}}(\vecb{x}):= \vecb{k}^\top\matr{A}\vecb{x}=\vecb{\beta}^\top\vecb{x}.
$$
Although not efficiently computable, we add to the function key additional Groth-Sahai proofs wich enable to prove that
 $g_{\matr{A}}(\vecb{x}) = g_{\matr{A}}(\vecb{x}')$, for some fixed $\vecb{x}$, while keeping computationally hard to find collisions --- in our ring signature $[\vecb{x}]_2$ contains $m$ verification keys of a signature scheme while $\vecb{x}$ contains $m$ secret keys.
We provide Groth-Sahai proofs that $\beta x=y$, for $\beta=\beta_i,x=x_i,y=\beta_ix_i$, and $i\in[m]$, which consists of two vectors
\begin{equation}
[\vecb{\psi}]_2=r[\vecb{c}]_2-\delta[\vecb{v}_2]_2\in\GG_2^2,\ [\vecb{\omega}]_1 =(\beta s-t)[\vecb{u}_1]_1+\delta[\vecb{u}_2]_1\in\GG_1^2
\label{eq:wi-proofs}
\end{equation}
and commitments to $x$ and to $y$
$$
[\vecb{c}]_2= x[\vecb{v}_1]_2+s[\vecb{v}_2]_2,\ [\vecb{d}]_2 =y[\vecb{v}_1]_2 + t[\vecb{v}_2]_2,\ s,t\gets\Z_q,
$$
satisfying the following verification equation
\begin{equation}
[\vecb{a}]_1[\vecb{c}^\top]_2-[\vecb{u}_1]_1[\vecb{d}^\top]_2 =[\vecb{u}_2]_1[\vecb{\psi}^\top]_2+[\vecb{\omega}]_1[\vecb{v}^\top_2]_2.
\label{eq:wi-verif}
\end{equation}

We prove the following Lemma.
\begin{lemma} \label{lemma:betax=y}
Consider the the quadratic equation $\beta x = y$, whose variables are $\beta,x,y$. 
The proof system whose crs consists of perfectly binding Groth-Sahai commitment keys $[\matr{U}]_1,[\matr{V}]$, whose prover computes the proofs as in (\ref{eq:wi-proofs}), and the verifier verifies equation (\ref{eq:wi-verif}), is perfectly complete and sound, and computationally zero-knowledge under the SXDH assumption.
\end{lemma}
\begin{proof}
Completeness follows by inspection. 
Soundness follows from the fact that, whenever $\matr{U},\matr{V}$ come from the perfectly binding distribution, $\vecb{u}_1\vecb{v}_1^\top,\vecb{u}_1\vecb{v}_2^\top,\allowbreak\vecb{u}_2\vecb{v}_1^\top,\allowbreak\vecb{u}_2\vecb{v}_2^\top$ form basis of $\Z_q^{2\times 2}$.
The right side of equation (\ref{eq:wi-verif}) has no component in $\vecb{u}_1\vecb{v}^\top_1$, while the left side component is $\beta x-y$. Hence, we conclude that $\beta x = y$.

Computational zero-knowledge follows from the following argument.
When $[\matr{U}]_1,[\matr{V}]_2$ are sampled from the perfectly hiding distribution, $\vecb{u}_1 = \mu\vecb{u}_2$ and $\vecb{v}_1  = \nu\vecb{v}_2$, for some random $\mu$ and $\nu$. In this setting we can sample $\vecb{a} = r\vecb{u}_2$ without changing $\vecb{a}$'s distribution, and we can simulate the proofs for any $[\vecb{c}]_2,[\vecb{d}]_2$ as 
\begin{equation} \label{eq:sim-proofs}
[\vecb{\psi}]_2 = r\mu[\vecb{c}]_2-\mu[\vecb{d}]_2-\delta[\vecb{v}_2]_2\text{ and }[\vecb{\omega}]_1 = \delta[\vecb{u}_2]_1\text{, for } \delta\gets\Z_q.
\end{equation} 
In particular, our simulator considers $[\vecb{c}]=s[\vecb{v}_2]_2,[\vecb{d}]_2=t[\vecb{v}_2]_2$ for $s,t\gets\Z_q$.

Note that both, the honest and the simulated proofs, follows exactly the same distribution. Indeed in both cases, $\vecb{a}$ and $\vecb{c},\vecb{d}$ are uniformly distributed in, respectively, $\Span(\vecb{u}_2)$ and $\Span(\vecb{v}_2)$, and the proofs are uniformly chosen among those that satisfy the respective verification equation.

%If there is an adversary $\advA$ which tells apart a real proof from a simulated one, we can construct an adversary against the semantic security of ElGamal. Our adversary  runs $\advA$ until it outputs $\beta,[x]_2,[y]_2$ and request a ciphertexts $[\vecb{c}]_1$ which is the encryption of either $[x]_2$ or $0$. It defines $[\vecb{a}]_1=r[\vecb{u}_2]_1$,$ r\gets\Z_q$, computes $[\vecb{d}]_2 = \beta[\vecb{c}]_2 + \rho[\vecb{v}]_2$  and simulates $[\vecb{\psi}]_2,[\vecb{\omega}]_1$ as in (\ref{eq:sim-proofs}), gives $[\vecb{a}]_1,[\vecb{c}]_2,[\vecb{d}]_2,[\vecb{\psi}]_2,[\vecb{\omega}]_1$ to $\advA$, and outputs whatever $\advA$ outputs. Clearly, when $[\vecb{c}]_1$ encrypts $[x]_2$, the proof follows exactly the same distribution of an honestly computed proof, and when encrypts $0$ is distributed as te simulated proof.
\end{proof} 

We will additionally add to $g$'s key proofs showing that $\matr{A}\in\mathcal{Q}_m$, as shown in Lemma \ref{lemma:Qm-memb}, and proofs that $\beta_ix_i = y_i$. Formally, we get that $\G$ on input a group key, field elements $x_1,\ldots x_m$, and $\vecb{\beta}\in\bits^m$, samples perfectly binding Groth-Sahai commitment keys $[\matr{U}]_1,[\matr{V}]_2$ and $\matr{A}\gets\mathcal{Q}_m^\vecb{\beta}$. Additionally, for $i\in[m]$, samples commitments $[\vecb{b}_i]_2 = \beta_i[\vecb{v}_1]_1+\rho_i[\vecb{v}_2]_2$, $[\vecb{c}_i]_2  = x_i[\vecb{v}_1]_2+s_i[\vecb{v}_2]_2$ and $[\vecb{d}_i]_2  = \beta_ix_i[\vecb{v}_1]_2+t_i[\vecb{v}_2]_2$ and proofs $[\vecb{\theta}_i]_2,[\vecb{\pi}_i]_1,[\vecb{\xi}]_2,[\vecb{\phi}]_1$ and $[\vecb{\psi}]_2,[\vecb{\omega}]_1$ computed as in (\ref{eq:Qm-memb-proofs}) and (\ref{eq:wi-proofs}), respectively. The function key is
$$
k = ([\matr{U}]_1,[\vecb{v}]_2,[\matr{A}]_1,[\matr{B}]_2,[\matr{C}]_2,[\matr{D}]_2,[\matr{\Theta}]_2,[\matr{\Pi}]_1,
[\matr{\Xi}]_2,[\matr{\Phi}]_1,[\matr{\Psi}]_2,[\matr{\Omega}]_1),
$$
where the matrices columns are denoted by the corresponding lower case letter.

Next, we show that given $k$  it is hard to find $\vecb{x}\neq 0$ such that $g_{\matr{A}}(\vecb{x})=0$. Intuitively, since the opening of $[\matr{A}]_1$ is computationally hidden, it  may open to some $\vecb{\beta}\in\bits^m$ that contains a single $1$ at a random position $i\gets[m]$. Then, by the hiding property Groth-Sahai commitments and the fact that at least one coordinate of $\vecb{x}$ is not zero, $g_\matr{A}(\vecb{x}) = x_i\neq 0$ with probability $1/m$. 

Formally, we prove the Lemma below. 
\begin{lemma}
For any adversary $\advA$ there exists an adversary $\advB$ against SXDH such that for any $\tilde{x}\in\Z_q^m$ and any $\vecb{\beta}\in\bits^m$  the probability that $\advA(k)$, $k\gets\KGen(gk,\tilde{\vecb{x}},\vecb{\beta})$, outputs a collision $[\vecb{x}]_2,[\vecb{x}']_2$ for $h_\matr{A}$  is less than $2m \adv_{\mathrm{SXDH}}(\advB)$. key.
\end{lemma}
\begin{proof}
The proof follows from the indistinguishability of the following games.
\begin{description}
\item[$\sfGame_0(\advA)$:] This game honestly runs the colission resitance experiment for $h_{\matr{A}}$ and outputs 1 if $\vecb{x}\neq\vecb{x}'$ and $h_\matr{A}(\vecb{x})=h_{\matr{A}}(\vecb{x}')$.
\item[$\sfGame_1(\advA)$:] This games picks a random $i\gets[m]$ and aborts if $[x_i]_2=[x'_i].$
\item[$\sfGame_2(\advA)$:] This game is exactly as $\sfGame_1$ but $[\matr{U}]_1$ and $[\matr{V}]_2$ are sampled from the perfectly hiding distribution.
\item[$\sfGame_3(\advA)$:] This game is exactly as $\sfGame_2$ but $\matr{A}$ is sampled from $\matr{Q}_m^{\vecb{\beta}_i}$, where $\vecb{\beta}_i\in\bits^m$ contains a single 1 at postion $i$. Consequently, $[\matr{B}]_2,[\matr{C}]_2,[\matr{D}]_2$ and corresponding proofs are computed using $\vecb{\beta}_i$.
\item[$\sfGame_4(\advA)$:] This game is exactly as $\sfGame_3$ but $[\matr{U}]_1$ and $[\matr{V}]_2$ are sampled from the perfectly binding distribution.
\end{description}
The probability we want to bound is $\Pr[\sfGame_0(\advA)=1]$ and it holds that $\Pr[\sfGame_0(\advA)=1]\leq m\Pr[\sfGame_1(\advA)=1]$ since $i$ is information theoretically hidden to $\advA$, while there is at least one index $i^*\in[m]$ such that $x_{i^*} \neq x_{i^*}'$. It also holds that $\Pr[\sfGame_1(\advA)=1]-\Pr[\sfGame_2(\advA)=1]\leq\adv_\mathrm{SXDH}(\advB)$, for adversary $\advB$, since the only change in the games is the Groth-Sahai commitment key wich is changed from perfectly binding to perfectly hiding.

It also holds that $\Pr[\sfGame_2(\advA)=1]-\Pr[\sfGame_3(\advA)=1]=0$, since commitment keys $[\matr{U}]_1,[\matr{V}]_2$ are perfectly hiding in both games, and hence, matrices $[\matr{A}]_1,[\matr{B}]_2,[\matr{C}]_2,[\matr{D}]_2$ follow exactly the same distribution in both games.
%Note that, by the self-reducibility of DDH, we can change many ciphertexts at once without increasing  the security loss.\footnote{For completeness, assume that you want to switch from encryptions of $m_1,\ldots,m_\ell\in\Z_q$ to encryptions of $m'_1,\ldots,m'_\ell\in\Z_q$. Construct an adversary that asks to its left or right oracle for encryptions of $0$ or $1$, receives $[\vecb{c}]_2$ as challenge, and returns $[\vecb{c}_i] := m_i[\vecb{c}_i] + m'_i[\smallpmatrix{0\\1}-\vecb{c}_i]_2+\delta_i[\vecb{v}]_2$, $\delta_i\gets\Z_q$. Clearly, when $[\vecb{c}]_2$ encrypts $1$, then the adversary returns encryptions to $m_1,\ldots,m_\ell$, and when it encrypts $0$ returns encryptions to $m'_1,\ldots,m'_\ell$.} Note that, when changing the cyperthexts, we need to simulate proofs as in (\ref{eq:Qm-sim-proofs}) and (\ref{eq:sim-proofs}).
Similarly as before, $\Pr[\sfGame_4(\advA)=1]-\Pr[\sfGame_3(\advA)=1]\leq\adv_\mathrm{SXDH}(\advB)$.

Finaly, $\Pr[\sfGame_4(\advA)=1]=0$ since in this case $x_i \neq x'_i$ and $g_\matr{A}(\vecb{x}) = x_i$.
\end{proof}

%Next, we show that the function key $k$ allows to derive a proof that $g_\matr{A}(\vecb{x}) = g_\matr{A}(\vecb{x}')$.
%To do so we only need to sum many of the proofs from Lemma \ref{lemma:betax=y} and add some re-randomization for keeping the proof zero-knowledge. That is
%\begin{equation}
%[\vecb{\psi}]_2 := \sum_{i=1}^m [\vecb{\psi}_i]_2-\delta[\vecb{v}_2]_2
%\text{ and }
%[\vecb{\omega}]_1 := \sum_{i=1}^m [\vecb{\omega}_i]_1+\delta[\vecb{u}_2]_1.
%\label{eq:derived-proof}
%\end{equation}
%such that
%\begin{equation}
%\sum_{i=1}^m [\vecb{a}_i]_1[\vecb{c}_i^\top]-[\vecb{u}_1]_1\sum_{i=1}^m[\vecb{d}^\top]_2 =
%[\vecb{\psi}]_1[\vecb{v}^\top_2]_2 + [\vecb{u}_2]_1[\vecb{\omega}_i^\top]_2.
%\label{eq:derived-verififcation}
%\end{equation}
%
%Consequetly, in our scheme we will publish  $[\vecb{\beta}_ix_i]_2$ and the proofs $[\vecb{\pi}_i]_1,[\vecb{\theta}_i]_2$ which will render possible to prove $g_\matr{A}(\vecb{\beta})=g_\matr{A}(\vecb{\beta}')$.
%
% Using the homomorphic properties of Groth-Sahai proofs it migth be derived a proof that
%$$
%[\matr{A}]_1([0]_2, [\vecb{x}']_2) - \pmatri{{[0]_1}\\{[1]_1}}[\vecb{x}^\top\matr{B}^\top]_2=[\vecb{\theta}]_1[\vecb{v}^\top_2]-[\vecb{u}]_1[\vecb{\pi}^\top]_2.
%$$
%If we multiply the previous equation by $\vecb{k}^\top$ on the left and by $\vecb{k}'$ on the right, we get that
%\begin{align*}
%\vecb{k}^\top[\matr{A}]_1([0]_2, [\vecb{x}']_2)\vecb{k}' - \vecb{k}^\top\pmatri{{[0]_1}\\{[1]_1}}[\vecb{x}^\top\matr{B}^\top]_2\vecb{k}'
%&=
%\vecb{k}^\top[\vecb{\theta}]_1[\vecb{v}^\top_2]\vecb{k}'-\vecb{k}^\top[\vecb{u}]_1[\vecb{\pi}^\top]_2\vecb{k}'\\
%[\vecb{k}^\top\matr{A}\vecb{x}']_T(0, 1)\vecb{k}' - \vecb{k}^\top\pmatri{0\\1}[\vecb{x}^\top\matr{B}^\top\vecb{k}']_T
%&=0\\
%[g_\vecb{\matr{A}}(\vecb{x}')]_T - [g_\matr{A}(\vecb{x})]_T &= 0.
%\end{align*}
%Note that the previous idendity follows from the fact that $g_{\matr{B}}(\vecb{x}) = {\vecb{k}'}^\top\matr{B}\vecb{x}=\vecb{\beta}^\top\vecb{x}=g_\matr{A}(\vecb{x})$ and that $\vecb{k},\vecb{k}'$ are decryption keys and thus $\vecb{k}^\top\vecb{u}=\vecb{v}^\top\vecb{k}'=0$ and $\vecb{k}^\top\smallpmatrix{0\\1}=(0,1)\vecb{k}'=1$.
%
%Given a collision $[\vecb{\beta}]_2,[\vecb{\beta}']_2$ for $g$, then $([\vecb{\beta}]_2-[\vecb{\beta}]'_2)\neq [\vecb{0}]$ is in the kernel of $[\matr{A}]_1$. Therefore, is trivial to prove that for any adversary $\advA$ there is an adversary $\advB$ such that $\adv^{\mathsf{Col}_g}(\advA)=\adv_{\mathcal{Q}_m^\top\mbox{-}\skermdh}(\advB)$, whenever $\matr{A}\gets\mathcal{Q}_m$.
%
%We note that given $A\in Q_m,[\matr{A}]_1\in\GG^{2\times m}_1,[\vecb{\beta}]_2\in\GG^m_2$, $[\vecb{y}]_1\in\GG_2^2$ and $[\vecb{y}']_1\in\GG^1_2$ one can express the statements $A\in Q_m$, $g_{[\matr{A}]_1}([\vecb{\beta}]_2)=[\vecb{y}]_2$, and $h(A)=[\vecb{y}']_1$ as (\ref{eq:Q}),(\ref{eq:g}), and (\ref{eq:h}), respectively.
% \begin{align}
%&e([a_{1}]_1,[1]_2) = e([1]_1,[{b}_1]_2)\text{ and }e([a_{2}]_1,[1]_2)=e([a_{1}]_1,[b_1]_2)
%\text{ for each }([\vecb{a}]_1,[\vecb{\beta}]_2)\in A \label{eq:Q}\\
%&\sum_{j=1}^m e([a_{i,j}]_1,[x_i]_1) = e([1]_1,[y_i]_2) \text{ for each } i\in\{1,2\} \label{eq:g}\\
%&\sum_{([\vecb{a}]_1,[\vecb{a}]_2)\in A} [a_i]_1 = [y'_i]_1 \text{ for each } i\in\{1,2\}.\label{eq:h}
%\end{align}
%Hence, one can compute Groth-Sahai proofs of size $\Theta(m),\Theta(1)$, and $\Theta(1)$, respectively, for the satisfiability of each statement.

Finally, we prove a simple lemma that relates functions $h$ and $g$
\begin{lemma}\label{lemma:hg}
Let $A\gets Q_m,A'\in Q_m,[\vecb{x}]_2,[\vecb{x}']_2\in\GG^m_2$. Then $h(\matr{A})=h(\matr{A}')$ and $g_{\matr{A}}(\vecb{x})=g_{\matr{A}'}(\vecb{x}')$ implies that $\matr{A}'$ is a second preimage of $h(\matr{A})$ or there exists a permutation matrix $\matr{P}$ such that $g_{\matr{A}}(\vecb{x})=g_{\matr{A}}(\matr{P}\vecb{x}')$.
\end{lemma}
\begin{proof}
If $\matr{A}\not\equiv_\matr{U} \matr{A}'$, then $\matr{A}'$ is a second preimage of $h(\matr{A})$. Else, there is a permutation matrix $\matr{P}$ and $\vecb{\delta}\in\Z_q^m$ such that $\matr{A}' =\matr{A}\matr{P}+\vecb{u}_2\vecb{\delta}^\top$. Then
$$
 g_{\matr{A}}(\vecb{x})=g_{\matr{A}'}(\vecb{x}')\Longleftrightarrow  g_{\matr{A}'}(\vecb{x})=g_{\matr{A}\matr{P}+\vecb{u}_2\vecb{\delta}^\top}(\vecb{x})=g_{\matr{A}\matr{P}}(\vecb{x}')=g_\matr{A}(\matr{P}\vecb{x}').
$$
\end{proof}