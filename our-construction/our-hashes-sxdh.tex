% !TEX root = ../main-ring-signature.tex

\subsubsection{The function $h$.} Consider perfectly binding Groth-Sahai commitment key $[\matr{U}]_1 = [\vecb{u}_1|\vecb{u}_2]_1\in\GG_1^{2\times 2}$ together with its trapdoor $\vecb{k}\in\Z_q^2$ such that $\vecb{k}^\top\vecb{u}_1 = 1$ and $\vecb{k}^\top\vecb{u}_2 = 0$. The message space of $h$ the set of matrices $\mathcal{Q}_m\in\Z_q^{2\times m}$ defined as follows
\begin{align*}
&\mathcal{Q}_m^\vecb{\beta} := \left\{
\begin{array}{c}
\matr{A} \in \Z_q^{2\times m}:
\exists \vecb{r}\in\Z_q^m\text{ s.t. }
{\matr{A} = \matr{U}\pmatri{\vecb{\beta}^\top\\\vecb{r}^\top}
%\matr{B} = \pmatri{0\\\vecb{\beta}^\top}+\vecb{v}\vecb{s}^\top
}
\end{array}\right\},
\ \mathcal{Q}_m := \cup_{\vecb{\beta}\in\bits^m} \mathcal{Q}_m^\vecb{\beta},\\
&\text{and } h(\matr{A}):= \vecb{k}^\top\sum_{i=1}^m\vecb{a}_i = \sum_{i=1}^m \beta_i \text{ is the hamming weight of }\vecb{\beta}.
\end{align*}

Note that, if $\vecb{a}_1,\ldots \vecb{a}_m\in\mathcal{Q}_{1}$, then $\matr{A} = (\vec{a}_1\cat\cdots\cat\vecb{a}_m)\in\mathcal{Q}_{m}$ and $[\vecb{a}_i]_1$ is a Groth-Sahai commitment to $\beta_i$.
Therefore $\sum_{i=0}^m[\vecb{a}_i]_1$ is a Groth-Sahai commitment to the hamming weight of $\vecb{\beta}$, that is $\Com_{[\matr{U}]_1}(\sum_{i=1}^m \beta_i;\sum_{i=0}^m r_i)$.

Sincet $h$ is not efficiently computable, we craft a key $k$ which encodes a matrix $\matr{A}$ and makes possible the construction proofs that $\matr{A}'\in\mathcal{Q}_m$ and $h(\matr{A}')=h(\matr{A})$ only if $\matr{A}'\equiv_{\matr{U}}\matr{A}$. Cocretely, the global key is a Groth-Sahai commitment key $[\matr{U}]_1,[\matr{V}]_2$ while each local key is $[\vecb{a}]_1$ and proofs that $\vec{a}\in\mathcal{Q}_1$.

Although is possible to find collisions, it happens only if the matrices open to bitstrings with the same hamming weight or equivalently, the openings are equal up to a permutation. Consequently, to restore collision resistance, we weaken the equality relation to the following equivalence relation
$$
\matr{A}' \equiv_\vecb{u} \matr{A} \iff \exists \matr{P}\in\mathcal{S}_m, \vecb{\delta}\in\Z_q^m \text{ s.t. } \matr{A}' = \matr{A}\matr{P} + \vecb{u}_2\vecb{\delta}^\top,
$$
where $\mathcal{S}_m\subset \bits^{m\times m}$ is the set of permutation matrices of size $m$.

We prove the following lemma.
\begin{lemma}
If $\matr{U}$ is chosen from the perfectly binding distribution, is (unconditionally) impossible to find collisions $\matr{A},\matr{A}'\in\mathcal{Q}_m$ such that $\matr{A}\not\equiv_\vecb{u}\matr{A}'$ and $h(\matr{A})=h(\matr{A}')$.
\end{lemma}
\begin{proof}
Given $\matr{A},\matr{A}'\in\mathcal{Q}_m$ we may assume that $\matr{A}=\matr{U}\smallpmatrix{\vecb{\beta}^\top\\\vecb{r}^\top},\matr{A}'=\matr{U}\smallpmatrix{{\vecb{\beta}'}^\top\\\vecb{\delta}^\top}$. Since $\vecb{\beta}$ and $\vecb{\beta}'$ must have the same hamming weigh there exists a permutation matrix $\matr{P}$ such that $\vecb{\beta}^\top\matr{P} = {\vecb{\beta}'}^\top$, and hence
\begin{align*}
\matr{A}' &= \matr{U}\pmatri{\vecb{\beta}^\top\matr{P}\\\vecb{\delta}^\top}+ \vecb{u}_2\vecb{r}^\top\matr{P} - \vecb{u}_2\vecb{r}^\top\matr{P}\\
&=
\matr{U} \pmatri{\vecb{\beta}^\top\\\vecb{r}^\top}\matr{P} + \vecb{u}_2(\vecb{\delta}^\top-\vecb{r}^\top\matr{P})\\
 &=
 \matr{A}\matr{P} + \vecb{u}_2\tilde{\vecb{\delta}}^\top.
\end{align*}
\end{proof}

Even 
when $\matr{U}$ is chosen from the perfectly hiding distribution, it turns out that $\mathcal{Q}_m$ becomes $\mathcal{Q}_m^{0^m}$, which is in fact $\Z_q^{2\times m}$, and there doesn't exist $\matr{A},\matr{A}'$ such that $\matr{A}\not\equiv_{\matr{U}}\matr{A}'$. 

To prove membership of $\vecb{a}$ in $\mathcal{Q}_1$ we construct Groth-Sahai proofs for
\begin{equation}
\beta(1-\beta)=0 \label{eq:Qm-memb},
\end{equation}
for $\beta=\beta_1,\ldots,\beta_m$. For more detail on how this proof is constructed and how to re-randomize them see Appendix \ref{sec:GSproofs-h}.

We will need to construct proofs that $h(\matr{A}') = h(\matr{A})$ from $[\vecb{a}'_1]_1,\ldots,[\vecb{a}'_m]_1$ and some $[\vecb{h}]_1 = \Com_{[\matr{U}]_1}(h(\matr{A}))$. To do so, it suffices to show that
\begin{equation}
\sum_{i=1}^{m}[\vecb{a}'_i]_1 - [\vecb{g}]_1 = \gamma[\vecb{u}_2]_1
\label{eq:coll-h}
\end{equation}
which can be proved with a Groth-Sahai proof. Further, such proof corresponds to a proof of membership in the span of $\vecb{u}_2$ which can be proved using more efficient QA-NIZK proofs \cite{C:JutRoy14,EC:KilWee15}. In this case the proof consists of only 1 element of $\GG_1$, instead of the $2$ required by Groth-Sahai proofs.

\subsubsection{The function $g$.} Consider perfectly binding Groth-Sahai commitment keys $[\matr{U}]_1=[\vecb{u}_1\cat\vecb{u}_2]_1,[\matr{W}]_2 = [\vecb{w}_1\cat\vecb{w}_2]_2$, together with a trapdoor for the commitment scheme $\vecb{k}\in\Z_q^2$ such that $\vecb{k}^\top\vecb{u}_2=0$ and $\vecb{k}^\top\vecb{u}_1=1$, consider also $[\matr{A}]_1$ such that $\matr{A}\gets\mathcal{Q}_m^\vecb{\beta}$.

We start defining $g$ as $g_\matr{A}(\vecb{x}) = \vecb{k}^\top\matr{A}\vecb{x} = \vecb{\beta}^\top\vecb{x}$. Although not efficiently computable, similarly as with $h$, it is possible to give a Groth-Sahai proof showing that some $\vecb{x},\vecb{x}'$ are indeed a collision (provided some extra information is given in the key).
In our ring signature $\vecb{\beta}=0$ but, since it remains hidden to the adversary, we could change to a game where $\vecb{\beta}$ is a random bit-vector of hamming weight 1. Since the unique coordinate $i$ such that $\beta_i=1$ remains hidden, $g_\matr{A}(\vecb{x}) = x_i \neq x'_i =g_\matr{A}(\vecb{x}')$ with probability $\approx 1/m$, whenever $\vecb{x}\neq\vecb{x}'$. But this reasoning is flawed, since $\vecb{\beta}=0$ in the actual instantiation and thus $g_\matr{A}(\vecb{x}) = 0$ for all $\vecb{x}\in\Z_q^m$ and is trivial to find collisions.
We will show that what is indeed hard to compute is the Groth-Sahai proof that $\vecb{\beta}^\top\vecb{x} = \vecb{\beta}^\top\vecb{x}'$, even when $g_\matr{A}(\vecb{x})\equiv 0$.

The problem is now, how to encode the proof in the hash function? Lets see how the Groth-Sahai proof for the statement $\vecb{\beta}^\top\vecb{x} = y$ looks like. We would like $\vecb{\beta}$ and $y$ to remain hidden, so we commit to these values with $[\matr{A}]_1$ and $[\vecb{d}]_1 = \vecb{\beta}^\top\vecb{x}[\vecb{u}_1]_1+t[\vecb{u}_2]_1$, respectively. The verification equation is
\begin{align*}
&[\matr{A}]\vecb{x}[\vecb{w}_1^\top]_2-[\vecb{d}]_1[\vecb{w}_1^\top]_2 = [\vecb{u}_1]_1[\vecb{\theta}^\top]_2+[\vecb{\pi}]_1[\vecb{w}_2^\top]_2\\
\iff & [\matr{A}]\vecb{x}[\vecb{w}_1^\top]_2 - [\vecb{u}_1]_1[\vecb{\theta}^\top]_2 - [\vecb{\pi}]_1[\vecb{w}_2^\top]_2 = [\vecb{d}]_1[\vecb{w}_1^\top]_2.
\end{align*}
We define $g$ as the left side of the previous equation
$$
g_{\matr{A}}([\vecb{x}]_2,[\vecb{\theta}]_2,[\vecb{\pi}]):= [\matr{A}]_1[\vecb{x}\vecb{w}_1^\top]_2-[\vecb{u}_2][\vecb{\theta}^\top]_2-[\vecb{\pi}]_1[\vecb{w}_2^\top]_2,
$$
which is efficiently given $\vecb{w}_1$.

However, is still not hard to find collisions. Indeed, any $[\vecb{x}]_2=[\vecb{x}']_2,[\vecb{\theta}]_2=\epsilon + \delta[\vecb{v_2}]_2,[\vecb{\theta}']_2 = \epsilon + \delta'[\vecb{w}_2]_2,[\vecb{\pi}]_1 = -\delta[\vec{u}_2]_2$  and $[\vecb{\pi}']_1 = -\delta'[\vecb{u}_2]_1$, $\delta\neq \delta'$ form a collision. It turns out that this are the only type of collisions so we we will say that $([\vecb{x}]_2,[\vecb{\theta}]_2,[\vecb{\pi}]_1)$ and $([\vecb{x}']_2,[\vecb{\theta}']_2,[\vecb{\pi}']_1)$ are a collision if  $g_{[\matr{A}]_1}([\vecb{x}]_2,[\vecb{\theta}]_2,[\vecb{\pi}]_1) = g_{[\matr{A}]_1}([\vecb{x}']_2,[\vecb{\theta}']_2,[\vecb{\pi}']_1)$ but we additionally require that $[\vecb{x}]_2\neq[\vecb{x}']_2$. For simplicity, we may simply write $g_\matr{A}(\vecb{x})$ instead of $g_{[\matr{A}]_1}([\vecb{x}]_2,[\vecb{\theta}]_2,[\vecb{\pi}]_1)$.

%The function $g$ outputs the evaluation of the left side of (a rearrangement of) the Groth-Sahai verification equation for the equation
%$
%\vecb{\beta}^\top\vecb{x} = y.
%$
%Intuitively, $g_{[\matr{A}]_1}([\vecb{x}]_2,[\vecb{\theta}]_2,[\vecb{\pi}]_1) = g_{[\matr{A}]_1}([\vecb{x}']_2,[\vecb{\theta}']_2,[\vecb{\pi}']_1)$ implies that $\vecb{\beta}^\top(\vecb{x}-\vecb{x}')=0$. Since $\vecb{\beta}$ is remains computationally hidden, it must be the case that $\vecb{\beta}$ is the all zero vector except at the position where $\vecb{x}$ and $\vecb{x}'$ differ.

We consider a reference string $(k_0,k_1,\ldots,k_m)$ formed of Groth-Sahai proofs wich enable to prove that
 $g_{\matr{A}}(\vecb{x}) = g_{\matr{A}}(\vecb{x}')$, for some fixed $\vecb{x}$, while keeping computationally hard to find collisions --- in our ring signature $[\vecb{x}]_2$ contains $m$ verification keys of a signature scheme while $\vecb{x}$ contains $m$ secret keys.
 %The global key $k_0$ contains the Groth-Sahai commitment keys and each local key $k_i$ contains Groth-Sahai proofs that $\beta x=y$, for $\beta=\beta_i,x=x_i,y=\beta_ix_i$, and $i\in[m]$. We note that such proofs can be re-randomized as shown in Appendix \ref{sec:GSproofs-g}.
We prove $g$'s collision resistance in an scenario where $k$ gives additional information about $[\matr{A}]_1$ to the adversary.
%Specifically, $k_0$ contains the Groth-Sahai commitment keys $[\matr{U}]_1,\allowbreak[\matr{V}]_2,\allowbreak[\matr{W}]_2$. We use two commitment keys in $\GG_2$ in order to open commitments computed with $[\matr{W}]_2$ even when the distribution of $[\matr{V}]_2$ is changed. Hence, we prove collision resistance even when the adversary knows $\matr{W}$. Each $k_i$ contains a proof showing that $\matr{a}_i\in\mathcal{Q}_1$, as shown in App.~\ref{sec:GSproofs-g}, and a proof that $\beta_ix_i = y_i$, as shown in App.~\ref{sec:GSproofs-g}. We prove collision resistance with respect to adaptive adversaries only when $\vecb{\beta} = \vecb{0}$. In the general case, when $\vecb{\beta}\in\bits^m$ the security degrades exponentially with the hamming weight of $\vecb{\beta}$, although we don't prove it. Anyway, for our ring signatures it will suffice collision resistance with respect to $\vecb{\beta}=\vecb{0}$.

Formally, $\KGen$ on input a group key $gk$ and vectors $\vecb{\beta}\in\bits^m$ and $\vecb{x}\in\Z_q^m$,  outputs $(k_0,k_1,\ldots,k_m)$ where $k_0 \gets \KGen_\mathsf{global}(gk)$ and  $k_i\gets \KGen_{\mathsf{local}}(gk,k_0,\beta_i,\allowbreak x_i)$. The global key generator $\KGen_\mathsf{global}$ on input a group key outputs Groth-Sahai commitment keys $[\matr{U}]_1,[\matr{V}]_2,[\matr{W}]_2$. We use two commitment keys in $\GG_2$ in order to open commitments computed with $[\matr{W}]_2$ even when the distribution of $[\matr{V}]_2$ is changed. Hence, we prove collision resistance even when the adversary knows $\matr{W}$. The local key generator on input a group key, a global key, a bit $\beta\in\bits$ and $x\in\Z_q$, samples $\matr{a}\gets\mathcal{Q}_1$ and outputs $[\vecb{a}]_1$, $[\vecb{b}]_2 = \Com_{[\matr{V}]_2}(\beta)$,  $[\vecb{c}]_2 = x[\vecb{w}_1]_2$, $[\vecb{d}]_1 = \Com_{[\matr{U}]_1}(\beta x)$ and proofs proofs $[\vecb{\theta}]_2,[\vecb{\pi}]_1,[\vecb{\xi}]_2,[\vecb{\phi}]_1$ and $[\vecb{\psi}]_2,[\vecb{\omega}]_1$ computed as in (\ref{eq:Qm-memb-proofs}) and (\ref{eq:wi-proofs}), respectively (see App.~\ref{sec:GSproofs-h} and \ref{sec:GSproofs-g}). The function key is
$$
k = ([\matr{U}]_1,[\matr{V}]_2,[\matr{W}]_2,[\matr{A}]_1,[\matr{B}]_2,[\matr{C}]_2,[\matr{D}]_2,[\matr{\Theta}]_2,[\matr{\Pi}]_1,
[\matr{\Xi}]_2,[\matr{\Phi}]_1,[\matr{\Psi}]_2,[\matr{\Omega}]_1),
$$
where the matrices columns are denoted by the corresponding lower case letter.

Next, we show that given $k$  it is hard to find $\vecb{x}\neq \vecb{x}'$ such that $g_{\matr{A}}(\vecb{x})=g_\matr{A}(\vecb{x}')$. %Intuitively, since the opening of $[\matr{A}]_1$ is computationally hidden, it  may open to some $\vecb{\beta}\in\bits^m$ that contains a single $1$ at a random position $i\gets[m]$. Then, by the hiding property Groth-Sahai commitments and the fact that $\vecb{x}$ and $\vecb{x}'$ differ in at least one coordinate, $g_\matr{A}(\vecb{x}) = x_i\neq x'_i  = g_{\matr{A}}(\vecb{x}')$ with probability $1/m$. 
%
%Further, for our ring signature, we require a slightly different reduction to the SXDH assumption. We allow the adversary against $g$ to give only commitments  to the purported collision, computed with commitment key $[\matr{W}]_2$, such that even the reduction ignores its opening. To achieve such strong guarantee, our reduction requires an additional advice indicating an index $i$ such that $x'_i\neq x'_j$ for all $j\in[m]$. To get a uniform treatment of both types of reduction, we consider reduction (an adversary against SXDH) $\advB$ which always receives an advice $i\in[0,m]$. If $i=0$ then the adversary against $g$ must give collisions ``in the clear'', and if $i>0$, then the adversary against $g$ is allowed to give committed collisions such that $x_i\neq x_i'$. Anyway, the proof for $i=0$ and $i>0$ are almost identical. 
%The  next lemma shows adaptive collision resistance. 
\begin{lemma} \label{lemma:h-cr-sxdh}
For any adversary $\advA$ against adaptive collision resistance of $g$, making at most $m$  queries to the key-generator oracle and $t$ queries to the corruption oracle and $t<m$, there exists an adversary $\advB$ against SXDH such that the probability that $\advA$ outputs a collision $([\vecb{x}]_1,[\vecb{\theta}]_2,[\vecb{\pi}]_1),(\vecb{x}',[\vecb{\theta}']_2,[\vecb{\pi}']_1)$ for $g_\matr{A}$  is less than $2m \adv_{\mathrm{SXDH}}(\advB)$. This statement holds even when $\matr{W}$, the discrete logs of $[\matr{W}]_2$, are given to the adversary 
\end{lemma}
\begin{proof}
We may assume that the adversary is ``eager'', that is, it makes all its queries to the key-generator oracle at the beginning. Note that any non-eager adversary $\advA'$ can be perfectly simulated  by an eager adversary that makes $m$ queries to its oracle and answers $\advA'$ queries ``on demand''. This is justified by the fact that the output of the key-generator oracle is independent of all previous outputs.
The proof follows from the indistinguishability of the following games.
\begin{description}
\item[$\sfGame_0(\advA)$:] This game honestly runs the collision resistance experiment for $g_{\matr{A}}$ and outputs 1 if $[\vecb{x}]_2 \neq [\vecb{x}']_2$ and $g_\matr{A}([\vecb{x}]_2,[\vecb{\theta}]_2,[\vecb{\pi}]_1)=g_{\matr{A}}([\vecb{x}']_2,[\vecb{\theta}']_2,[\vecb{\pi}']_2)$.
\item[$\sfGame_1(\advA)$:] This game is exactly as $\sfGame_0$ but it picks $i\gets[m]$ and aborts if the adversary requests the random coins for generating $k_i$ or outputs $[x_i]_2=[x'_i]_2$.
\item[$\sfGame_2(\advA)$:] This game is exactly as $\sfGame_1$ but $[\matr{U}]_1$ and $[\matr{V}]_2$ are sampled from the perfectly hiding distribution.
\item[$\sfGame_3(\advA)$:] This game is exactly as $\sfGame_2$ but $\matr{A}$ is sampled from $\mathcal{Q}_m^{\vecb{\beta}_i}$, where $\vecb{\beta}_i\in\bits^m$ contains a single 1 at postion $i$. Consequently, $[\matr{B}]_2,[\matr{C}]_2,[\matr{D}]_2$ and corresponding proofs are computed using $\vecb{\beta}_i$.
\item[$\sfGame_4(\advA)$:] This game is exactly as $\sfGame_3$ but $[\matr{U}]_1$ and $[\matr{V}]_2$ are sampled from the perfectly binding distribution.
\end{description}
$\Pr[\sfGame_0(\advA)=1]\leq \frac{m}{m-t}\Pr[\sfGame_1(\advA)=1]$  follows from the fact that the probability that $\sfGame_1$ doesn't abort is: a) the probability that each of the corruption calls doesn't abort and b) the probability that $[x_i]_2=[x'_i]_2$. The probability of a) is $\frac{m-i}{m}$ at the $i$-th call, and the probability of b) is $\frac{1}{m-t}$. It follows that the desired probaility is $\frac{m-1}{m}\frac{m-2}{m-1}\cdots\frac{m-t}{m}\frac{1}{m-t}=\frac{1}{m}$

It holds that $\Pr[\sfGame_1(\advA)=1]-\Pr[\sfGame_2(\advA)=1]\leq\adv_\mathrm{SXDH}(\advB)$, for some adversary $\advB$, since the only change in the games is the Groth-Sahai commitment key wich is changed from perfectly binding to perfectly hiding. %However, here the argument is slightly more subtle. An adversary attempting to tell apart perfectly binding from perfectly hiding Groth-Sahai commitments keys (or equivalently, an adversary against SXDH) can't efficiently simulate $\sfGame_1(\advA)$ and $\sfGame_2(\advA)$ as they require the computation of $g$ while the discrete logarithm of the commitment keys is unknown. Let $\vecb{x}_1,\vecb{x}'_1$ and $\vecb{x}_2,\vecb{x}'_2$ the purported collisions output by $\advA$ in, respectively, $\sfGame_1$ and $\sfGame_2$. We note that $\Pr[\sfGame_1(\advA)=1] \leq \Pr[x_{1,i}\neq x'_{1,i}]$ and $\Pr[\sfGame_2(\advA)=1] = \Pr[x_{2,i}\neq x'_{2,i}]$, since in $\sfGame_2$ it holds that $g_{\matr{A}}(\vecb{x}) = \vecb{k}^\top\matr{A}\vecb{x} = 0 = g_\matr{A}(\vecb{x}')$ for any $\vecb{x},\vecb{x}'$. Therefore, $\Pr[\sfGame_1(\advA)=1]-\Pr[\sfGame_2(\advA)=1]\leq \Pr[x_{1,i}\neq x'_{1,i}]-\Pr[x_{2,i}\neq x'_{2,i}]$. Instead of simulating the games, the adversary $\advB$, which receives as challenge Groth-Sahai commitment keys, runs $\sfGame_1$ replacing the commitment keys by it challenges and outputs $1$ if $x_i\neq x'_i$ and $0$ otherwise regardless of whether $g_\matr{A}(\vecb{x})=g_{\matr{A}}(\vecb{x}')$ or not.

It also holds that $\Pr[\sfGame_2(\advA)=1]-\Pr[\sfGame_3(\advA)=1]=0$, since commitment keys $[\matr{U}]_1,[\matr{V}]_2$ are perfectly hiding in both games, and hence, matrices $[\matr{A}]_1,[\matr{B}]_2,[\matr{C}]_2,[\matr{D}]_2$ follow exactly the same distribution in both games.
%Note that, by the self-reducibility of DDH, we can change many ciphertexts at once without increasing  the security loss.\footnote{For completeness, assume that you want to switch from encryptions of $m_1,\ldots,m_\ell\in\Z_q$ to encryptions of $m'_1,\ldots,m'_\ell\in\Z_q$. Construct an adversary that asks to its left or right oracle for encryptions of $0$ or $1$, receives $[\vecb{c}]_2$ as challenge, and returns $[\vecb{c}_i] := m_i[\vecb{c}_i] + m'_i[\smallpmatrix{0\\1}-\vecb{c}_i]_2+\delta_i[\vecb{w}]_2$, $\delta_i\gets\Z_q$. Clearly, when $[\vecb{c}]_2$ encrypts $1$, then the adversary returns encryptions to $m_1,\ldots,m_\ell$, and when it encrypts $0$ returns encryptions to $m'_1,\ldots,m'_\ell$.} Note that, when changing the cyperthexts, we need to simulate proofs as in (\ref{eq:Qm-sim-proofs}) and (\ref{eq:sim-proofs}).

Similarly as before, $\Pr[\sfGame_4(\advA)=1]-\Pr[\sfGame_3(\advA)=1]\leq\adv_\mathrm{SXDH}(\advB)$. We point out that the adversary, after corrupting the $j$-th key, could detect that $[\vecb{a}_j]_1$'s  can no longer be opened to 0 since now the commitment keys are prfectly binding. However, since only $\beta_i\neq 0$, this can not happened as the game would have aborted when corrupting $k_i$.

Finaly, $\Pr[\sfGame_4(\advA)=1]=0$ since in this case $x_i \neq x'_i$ and $g_\matr{A}([\vecb{x}]_2,[\vecb{\theta}]_2,\allowbreak[\vecb{\pi}]_1)=g_{\matr{A}}([\vecb{x}']_2,[\vecb{\theta}']_2,[\vecb{\pi}']_2)$  only if $x_i = x'_i$. Note that in this case $\vecb{u}_1\vecb{w}_1^\top,\vecb{u}_1\vecb{w}_2^\top,\allowbreak\vecb{u}_2\vecb{w}_1^\top,\vecb{u}_1\vecb{w}_2^\top$ is a basis of $\Z_q^{2\times 2}$. Then, there is a collision only if
\begin{align*}
&\matr{A}(\vecb{x}-\vecb{x}')\vecb{w}_1^\top - (\vec{\theta}-\vecb{\theta}')\vecb{w}_2 - \vecb{u}_2(\vecb{\pi}-\vecb{\pi}')^\top = 0\\
&\vecb{u}_1\vecb{\beta}^\top\vecb{x}\vecb{w}_1^\top - (\vec{\theta}-\vecb{\theta}')\vecb{w}_2 - \vecb{u}_2(\vecb{\pi}-\vecb{\pi}'-\vecb{w}_1\vecb{r}^\top(\vecb{x}-\vecb{x}'))^\top = 0\\
\Longrightarrow & \vecb{\beta}^\top(\vecb{x}-\vecb{x}') = 0\\
\Longrightarrow & x_i  = x'_i\\
\end{align*}
\end{proof}

We will also need to show that $g_{\matr{A}'}(\vecb{x}')= g_{\matr{A}}(\vecb{x})$ from $[\vecb{d}_1]_2,\ldots,[\vecb{d}_m]_2$ and some $[\vecb{g}]_2 = \Com_{[\matr{W}]_2}(\vecb{x})$. To do so, it suffices to give to prove the satisfiability of
\begin{equation}
\sum_{i=1}^{m}[\vecb{d}_i]_2 - [\vecb{g}]_2 = \gamma[\vecb{w}_2]_2.
\label{eq:coll-g}
\end{equation}
A Groth-Sahai proof requires $2$ group elements, while a QA-NIZK proof requires only one \cite{C:JutRoy14,EC:KilWee15}.


Finally, lemma \ref{lemma:hg-sxdh} relates functions $h$ and $g$ and is analogous to lemma \ref{lemma:hg} with some differences. When $g$ is instantiated in the ring signature it won't be possible to extract the permutation $\matr{P}$ which relates $\matr{A}$ and $\matr{A}'$ as with the PPA instantiation.\footnote{It might be also possible to extract $\matr{P}$ if we compute a Groth-Sahai commitment to the randomness used to re-randomize $\matr{A}'$. This solution is far less efficient though.} To alleviate this we prove that is hard to find $\vecb{x},\vecb{x}'$ such that $g_\matr{A}(\vecb{x})=g_\matr{A'}(\vecb{x}')$, $\vecb{x}$ is uniform over $\Z_q^m$, and $\vecb{x}$ is not a permutation of $\vecb{x}'$.

The proof almost mimics lemma \ref{lemma:h-cr-sxdh} but the abort condition of $\sfGame_1$ is different. 
Consider this new abort condition: ``pick random $i\in[m]$ and abort if $x_i=x_j$ for some $j\in[m]$''. The following lemma implies that, with high probability, the abort condition will not be triggered.
\begin{lemma}
If $\vecb{x}\gets\Z_q^m$ is not a permutation of $\vecb{x}'\in\Z_q^m$, then
$$
\Pr[\exists i\in[m] \forall j\in[m], x_i\neq x'_j] > 1-\frac{m^2}{q}
$$
\end{lemma}
\begin{proof}
Conditioned on the event $E:=x_i\neq x_j$ for all $i\neq j$, the probability is 1. Indeed, to reach a contradiction assume that there is a non-zero probability that for all $i\in[m]$ there is some $j\in[m]$ such that $x_i=x'_j$. Even more, the assumption that $x_i\neq x_j$ for all $i\neq j$ implies that such $j$ is unique. But then we might define the permutation $\pi$ which assigns $j$ to $i$ and $x_i = x'_{\pi(i)}$ for all $i$ which contradicts the assumption that $\vecb{x}$ is not a permutation of $\vecb{x}'$.

On the other hand, there are $q^m$ different ways of picking $\vecb{x}$ of which $q!/(q-m)!$ satisfy $E$. Then
\begin{align*}
\Pr[E] & = \frac{q!/(q-m)!}{q^m} > \frac{(q-m)^m}{q^m} = \left(1-\frac{m}{q}\right)^m \geq 1-\frac{m^2}{q},
\end{align*}
where the last step follows from Bernoulli's inequality.

We conclude that
\begin{align*}
\Pr[\exists i\in[m] \forall j\in[m], x_i\neq x'_j] &\geq  \Pr[\exists i\in[m] \forall j\in[m], x_i\neq x'_j|E]\Pr[E]\\
& \geq 1-\frac{m^2}{q}.
\end{align*}
\end{proof}
 While the abort condition of lemma \ref{lemma:h-cr-sxdh} doesn't guarantee that $\vecb{\beta}^\top \vecb{x} \neq \vecb{\beta}^\top \vecb{x}'$ if $\vecb{x}'$ is permuted, the new abort condition does. Indeed, $\vecb{\beta}^\top\matr{P}\vecb{x}' = x_j\neq x_i = \vecb{\beta}^\top\vecb{x}$, for some $j\in[m]$.
 
 \begin{lemma} \label{lemma:h-cr-sxdh}
 Consider the following variant of the adaptive collision-resistance game. The adversary wins if produces $\matr[A]'_1,[\vecb{x}]_2,[\vecb{x}],[\vecb{\pi}]_2,[\vecb{\pi}']_2,[\vecb{\theta}]_1,[\vecb{\theta}']_1$ such that $h(\matr{A})=h(\matr{A}')$ and $g_\matr{A}([\vecb{x}]_2,[\vecb{\theta}]_2,[\vecb{\pi}]_1)=g_{\matr{A}}([\vecb{x}']_2,[\vecb{\theta}']_2,[\vecb{\pi}']_2)$.
For any adversary $\advA$ against this variant of adaptive collision resistance of $g$, making at most $m$  queries to the key-generator oracle and $t$ queries to the corruption oracle and $t<m$, there exists an adversary $\advB$ against SXDH with the following property such that he probability that $\advA$ outputs wins is less than $2m\frac{q}{1-m^2/q} \adv_{\mathrm{SXDH}}(\advB)$. This statement holds even when $\matr{W}$, the discrete logs of $[\matr{W}]_2$, are given to the adversary 
\end{lemma}
\begin{proof}
As in lemma \ref{lemma:g-cr-sxdh} we may assume, without loss of generality, that the adversary is ``eager'', that is, it makes all its queries to the key-generator oracle at the beginning.

The proof follows from the indistinguishability of the following games.
\begin{description}
\item[$\sfGame_0(\advA)$:] This game honestly runs the modified collision resistance experiment for $g_{\matr{A}}$ and outputs 1 if $[\vecb{x}]_2 $ is not a permutation of $[\vecb{x}']_2$ and $g_\matr{A}([\vecb{x}]_2,[\vecb{\theta}]_2,[\vecb{\pi}]_1)=g_{\matr{A}'}([\vecb{x}']_2,[\vecb{\theta}']_2,[\vecb{\pi}']_2)$ and $h(\matr{A})=h(\matr{A}')$.
\item[$\sfGame_1(\advA)$:] This game is exactly as $\sfGame_0$ but it picks $i\gets[m]$ and aborts if the adversary requests the random coins for generating $k_i$ or $[x_i]_2=[x'_j]_2$ for some $j\in[m]$.
\item[$\sfGame_2(\advA)$:] This game is exactly as $\sfGame_1$ but $[\matr{U}]_1$ and $[\matr{V}]_2$ are sampled from the perfectly hiding distribution.
\item[$\sfGame_3(\advA)$:] This game is exactly as $\sfGame_2$ but $\matr{A}$ is sampled from $\mathcal{Q}_m^{\vecb{\beta}_i}$, where $\vecb{\beta}_i\in\bits^m$ contains a single 1 at postion $i$. Consequently, $[\matr{B}]_2,[\matr{C}]_2,[\matr{D}]_2$ and corresponding proofs are computed using $\vecb{\beta}_i$.
\item[$\sfGame_4(\advA)$:] This game is exactly as $\sfGame_3$ but $[\matr{U}]_1$ and $[\matr{V}]_2$ are sampled from the perfectly binding distribution.
\end{description}
$\Pr[\sfGame_0(\advA)=1]\leq \frac{m}{m-t}\Pr[\sfGame_1(\advA)=1]$  follows from the fact that the probability that $\sfGame_1$ doesn't abort is: a) the probability that each of the corruption calls doesn't abort and b) the probability that $[x_i]_2=[x'_i]_2$. The probability of a) is $\frac{m-i}{m}$ at the $i$-th call, and the probability of b) is $\frac{1}{m-t}$. It follows that the desired probaility is $\frac{m-1}{m}\frac{m-2}{m-1}\cdots\frac{m-t}{m}\frac{1}{m-t}=\frac{1}{m}$

It holds that $\Pr[\sfGame_1(\advA)=1]-\Pr[\sfGame_2(\advA)=1]\leq\adv_\mathrm{SXDH}(\advB)$, for some adversary $\advB$, since the only change in the games is the Groth-Sahai commitment key wich is changed from perfectly binding to perfectly hiding. %However, here the argument is slightly more subtle. An adversary attempting to tell apart perfectly binding from perfectly hiding Groth-Sahai commitments keys (or equivalently, an adversary against SXDH) can't efficiently simulate $\sfGame_1(\advA)$ and $\sfGame_2(\advA)$ as they require the computation of $g$ while the discrete logarithm of the commitment keys is unknown. Let $\vecb{x}_1,\vecb{x}'_1$ and $\vecb{x}_2,\vecb{x}'_2$ the purported collisions output by $\advA$ in, respectively, $\sfGame_1$ and $\sfGame_2$. We note that $\Pr[\sfGame_1(\advA)=1] \leq \Pr[x_{1,i}\neq x'_{1,i}]$ and $\Pr[\sfGame_2(\advA)=1] = \Pr[x_{2,i}\neq x'_{2,i}]$, since in $\sfGame_2$ it holds that $g_{\matr{A}}(\vecb{x}) = \vecb{k}^\top\matr{A}\vecb{x} = 0 = g_\matr{A}(\vecb{x}')$ for any $\vecb{x},\vecb{x}'$. Therefore, $\Pr[\sfGame_1(\advA)=1]-\Pr[\sfGame_2(\advA)=1]\leq \Pr[x_{1,i}\neq x'_{1,i}]-\Pr[x_{2,i}\neq x'_{2,i}]$. Instead of simulating the games, the adversary $\advB$, which receives as challenge Groth-Sahai commitment keys, runs $\sfGame_1$ replacing the commitment keys by it challenges and outputs $1$ if $x_i\neq x'_i$ and $0$ otherwise regardless of whether $g_\matr{A}(\vecb{x})=g_{\matr{A}}(\vecb{x}')$ or not.

It also holds that $\Pr[\sfGame_2(\advA)=1]-\Pr[\sfGame_3(\advA)=1]=0$, since commitment keys $[\matr{U}]_1,[\matr{V}]_2$ are perfectly hiding in both games, and hence, matrices $[\matr{A}]_1,[\matr{B}]_2,[\matr{C}]_2,[\matr{D}]_2$ follow exactly the same distribution in both games.
%Note that, by the self-reducibility of DDH, we can change many ciphertexts at once without increasing  the security loss.\footnote{For completeness, assume that you want to switch from encryptions of $m_1,\ldots,m_\ell\in\Z_q$ to encryptions of $m'_1,\ldots,m'_\ell\in\Z_q$. Construct an adversary that asks to its left or right oracle for encryptions of $0$ or $1$, receives $[\vecb{c}]_2$ as challenge, and returns $[\vecb{c}_i] := m_i[\vecb{c}_i] + m'_i[\smallpmatrix{0\\1}-\vecb{c}_i]_2+\delta_i[\vecb{w}]_2$, $\delta_i\gets\Z_q$. Clearly, when $[\vecb{c}]_2$ encrypts $1$, then the adversary returns encryptions to $m_1,\ldots,m_\ell$, and when it encrypts $0$ returns encryptions to $m'_1,\ldots,m'_\ell$.} Note that, when changing the cyperthexts, we need to simulate proofs as in (\ref{eq:Qm-sim-proofs}) and (\ref{eq:sim-proofs}).

Similarly as before, $\Pr[\sfGame_4(\advA)=1]-\Pr[\sfGame_3(\advA)=1]\leq\adv_\mathrm{SXDH}(\advB)$. We point out that the adversary, after corrupting the $j$-th key, could detect that $[\vecb{a}_j]_1$'s  can no longer be opened to 0 since now the commitment keys are prfectly binding. However, since only $\beta_i\neq 0$, this can not happened as the game would have aborted when corrupting $k_i$.

Finaly, $\Pr[\sfGame_4(\advA)=1]=0$ since in this case $x_i \neq x'_i$ and $g_\matr{A}([\vecb{x}]_2,[\vecb{\theta}]_2,\allowbreak[\vecb{\pi}]_1)=g_{\matr{A}}([\vecb{x}']_2,[\vecb{\theta}']_2,[\vecb{\pi}']_2)$  only if $x_i = x'_i$. Note that in this $\vecb{u}_1\vecb{w}_1^\top,\vecb{u}_1\vecb{w}_2^\top,\vecb{u}_2\vecb{w}_1^\top,\vecb{u}_1\vecb{w}_2^\top$ is a basis of $\Z_q^{2\times 2}$. There is a collision only if
\begin{align*}
&\matr{A}(\vecb{x}-\vecb{x}')\vecb{w}_1^\top - (\vec{\theta}-\vecb{\theta}')\vecb{w}_2 - \vecb{u}_2(\vecb{\pi}-\vecb{\pi}')^\top = 0\\
&\vecb{u}_1\vecb{\beta}^\top\vecb{x}\vecb{w}_1^\top - (\vec{\theta}-\vecb{\theta}')\vecb{w}_2 - \vecb{u}_2(\vecb{\pi}-\vecb{\pi}'-\vecb{w}_1\vecb{r}^\top(\vecb{x}-\vecb{x}'))^\top = 0\\
\Longrightarrow & \vecb{\beta}^\top(\vecb{x}-\vecb{x}') = 0\\
\Longrightarrow & x_i  = x'_i\\
\end{align*}
\end{proof}

\begin{lemma}\label{lemma:hg-sxdh}
Let $\matr{A}\gets \mathcal{Q}_m,\matr{A}'\in \mathcal{Q}_m,\vecb{x},\vecb{x}'\in\Z_q^m$. Then $h(\matr{A})=h(\matr{A}')$ and $g_{\matr{A}}(\vecb{x})=g_{\matr{A}'}(\vecb{x}')$ implies that $\matr{A}'$ is a second preimage of $h(\matr{A})$ or there exists a permutation matrix $\matr{P}$ such that $g_{\matr{A}}(\vecb{x})=g_{\matr{A}}(\matr{P}\vecb{x}')$.
\end{lemma}
\begin{proof}
If $\matr{A}\not\equiv_\matr{U} \matr{A}'$, then $\matr{A}'$ is a second preimage of $h(\matr{A})$. Else, there is a permutation matrix $\matr{P}$ and $\vecb{\delta}\in\Z_q^m$ such that $\matr{A}' =\matr{A}\matr{P}+\vecb{u}_2\vecb{\delta}^\top$. Then
$$
 g_{\matr{A}}(\vecb{x})=g_{\matr{A}'}(\vecb{x}')\Longleftrightarrow  g_{\matr{A}'}(\vecb{x})=g_{\matr{A}\matr{P}+\vecb{u}_2\vecb{\delta}^\top}(\vecb{x})=g_{\matr{A}\matr{P}}(\vecb{x}')=g_\matr{A}(\matr{P}\vecb{x}').
$$
\end{proof}