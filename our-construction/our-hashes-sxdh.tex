% !TEX root = ../main-ring-signature.tex

We instantiate definitions \ref{def:hash1} and \ref{def:hash2}  with, respectively, the functions $g$ and $h$ as defined as follows.

\paragraph{The function $h$.} The algorithm $\KGen$ returns the description of an asymmetric group plus ElGamal encryption keys $[\vecb{u}]_1$  and $[\vecb{v}]_2$. Let $\mathcal{M}=\mathcal{Q}_m$, $\mathcal{Y}=[0,m]$, and $\vecb{\beta}\in\bits^m$, and define
\begin{align*}
&\mathcal{Q}_m^\vecb{\beta} := \left\{
\begin{array}{c}
\matr{A} \in \Z_q^{2\times m}:
\exists \vecb{r}\in\Z_q^m\text{ s.t. }
{\matr{A} = \pmatri{0\\\vecb{\beta}^\top}+\vecb{u}\vecb{r}^\top
%\matr{B} = \pmatri{0\\\vecb{\beta}^\top}+\vecb{v}\vecb{s}^\top
}
\end{array}\right\},
\ \mathcal{Q}_m := \cup_{\vecb{\beta}\in\bits^m} \mathcal{Q}_m^\vecb{\beta},\\
&\text{and } h(\matr{A}):= \sum_{i=1}^m\vecb{a}_i,
\end{align*}

Note that, if $A=(\vecb{a}_1\cat\cdots\cat\vecb{a}_m)\in\mathcal{Q}_m$, then $[\vecb{a}_i]_1$ is an ElGamal encryption $\Enc_{[\vecb{u}]_1}(\beta_i;r_i)$.

%It might seem odd to define $\mathcal{Q}_m$ as a set of pairs of matrices in both groups while $h$ only require elements in one group. However, this will be crucial in the security proof of our ring signature, where we need to compute $[vk \vecb{\beta}]_2$, for some $vk\in\Z_q$,
%from  $[\vecb{\beta}]_2$ without knowledge of its discrete logarithm. For simplicity, we may just write $h([\matr{A}]_1)$ for $[\matr{A}]_1\in \GG_1^{2\times m}$ (which is still well defined).

%Although is hard to compute $h$, is easy to show that $h(A)=h(A')$ whenever elements of $A'$ are re-randomizations of the elements of $A$. That is, for each $\vecb{a}'\in A'$ there exists a unique  $\vecb{a}\in A$ and $\delta\in\Z_q$ such that $\vecb{a}' = \vecb{a}+\delta\vecb{u}$. Given all the random coins used for rerandomization, the statement $h(A)=h(A')$ can be proved using Groth-Sahai.

Note that $[h(\matr{A})]_1$ outputs an encryption of the hamming weight of $\vecb{\beta}$. Although is possible to find collisions, two matrices form a collision if they encrypt bitstrings with the same hamming weight or equivalently, the decryptions are equal up to a permutation. Consequently, to restore collision resistance, we weaken the equality relation to the following equivalence relation
$$
\matr{A}' \equiv_\vecb{u} \matr{A} \iff \exists \matr{P}\in\mathcal{S}_m, \vecb{\delta}\in\Z_q^m \text{ s.t. } \matr{A}' = \matr{A}\matr{P} + \vecb{u}\vecb{\delta}^\top,
$$
where $\mathcal{S}_m\subset \bits^{m\times m}$ is the set of permutation matrices of size $m$.

Lets see that is in fact impossible to find collisions $\matr{A},\matr{A}'\in\mathcal{Q}_m$ such that $\matr{A}\not\equiv_\vecb{u}\matr{A}'$ and $h(\matr{A})=h(\matr{A}')$. Assume that $\matr{A}=\smallpmatrix{0\\\vecb{\beta}^\top}+\vecb{u}\vecb{r}^\top,\matr{A}'=\smallpmatrix{0\\{\vecb{\beta}'}^\top}+\vecb{u}\vecb{\delta}^\top$. Since $\vecb{\beta}$ and $\vecb{\beta}'$ have the same hamming weigh there exists a permutation matrix $\matr{P}$ such that $\vecb{\beta}^\top\matr{P} = {\vecb{\beta}'}^\top$, and hence
\begin{align*}
\matr{A}' &= \pmatri{0\\\vecb{\beta}^\top\matr{P}} + \vecb{u}\vecb{\delta} + \vecb{u}\vecb{r}^\top\matr{P} - \vecb{u}\vecb{r}^\top\matr{P}\\
&=
 \left(\pmatri{0\\\vecb{\beta}^\top} + \vecb{u}\vecb{r}^\top \right)\matr{P} + \vecb{u}(\vecb{\delta}^\top-\vecb{r}^\top\matr{P})\\
 &=
 \matr{A}\matr{P} + \vecb{u}\tilde{\vecb{\delta}}^\top.
\end{align*}

\paragraph{The function $g$.} Consider $\mathcal{M}=\GG^m_2$, $\mathcal{Y}=\GG_2^2$, and $\KGen$ receives a bitstring $\vecb{\beta}\in\bits^m$ together with a and picks a group key $gk$ and samples ElGamal encryption key $\vecb{u}$ together with its decryption key $\vecb{k}\in\Z_q^2$, samples $\matr{A}\gets\mathcal{Q}_m^\vecb{\beta}$, and outputs $[\matr{A}]_1$. The function $g$ is defined as
$$
g_{\matr{A}}(\vecb{x}):= \vecb{k}^\top\matr{A}\vecb{x}=\vecb{\beta}^\top\vecb{x}.
$$
Although not efficiently computable, in our ring signature we provide additional elements wich enable to prove that 
 $g_{\matr{A}}(\vecb{x}) = g_{\matr{A}}(\vecb{x}')$, for an honestly sampled $[\vecb{x}]_2$ --- which in our ring signature correspond to honestly sampled verification key and we can't use its discrete logarithm ---  and any $[\vecb{x}']_2$ sampled by the adversary.
First, we sample another matrix $\matr{B}\gets\mathcal{Q}_m^\vecb{\beta}$, i.e.~encrypts the same $\vecb{\beta}$ as $\matr{A}$, an provide proofs that $\beta_i[x_i]_2=[x_i\beta'_i]_2$, where $\beta_i'$ is the decryption of $\vecb{b}_i$.
 The Groth-Sahai proof for the previous equation consists of two vectors
$$[\vecb{\pi}_i]_1 = \delta_i[\vecb{u}]_1\in\GG_1^2,\ [\vecb{\theta}_i]_2=\in\GG_2^2$$ satisfying the following verification equation
$$ 
[\vecb{a}_i]_1([0]_2, {[x_i]_2})-\pmatri{[0]_1\\{[1]_1}}[(\vecb{b}_ix_i)^\top]_2 =[\vecb{\pi}_i]_1[\vecb{v}^\top]_2+[\vecb{u}]_1[\vecb{\theta}^\top_i]_2.
$$
A proof that $g_\matr{A}(\vecb{\beta}) = g_\matr{A}(\vecb{\beta}')$ can be derived from the addition of the previous proof plus re-randomization for keeping the proof zero-knowledge. That is
$$
[\vecb{\pi}]_1 := \sum_{i=1}^m [\vecb{\pi}_i]_1+\delta[\vecb{u}]_1
\text{ and }
[\vecb{\theta}]_2 := \sum_{i=1}^m [\vecb{\theta}_i]_2-\delta[\vecb{v}]_2.
$$
such that
$$
\left(\sum_{i=1}^m [\vecb{a}_i]([0]_1,[x_i]_1)\right)-\pmatri{{[0]_1}\\{[1]_1}}\left(\sum_{i=1}^m[\vecb{\beta}_ix_i]_2\right)^\top =
[\vecb{\pi}]_1[\vecb{v}^\top]_2 + [\vecb{u}_1][\vecb{\theta}_i^\top]_2
$$ Consequetly, in our scheme we will publish  $[\vecb{\beta}_ix_i]_2$ and the proofs $[\vecb{\pi}_i]_1,[\vecb{\theta}_i]_2$ which will render possible to prove $g_\matr{A}(\vecb{\beta})=g_\matr{A}(\vecb{\beta}')$.

 Using the homomorphic properties of Groth-Sahai proofs it migth be derived a proof that
$$
[\matr{A}]_1([0]_2, [\vecb{x}']_2) - \pmatri{{[0]_1}\\{[1]_1}}[\vecb{x}^\top\matr{B}^\top]_2=[\vecb{\theta}]_1[\vecb{v}^\top_2]-[\vecb{u}]_1[\vecb{\pi}^\top]_2.
$$
If we multiply the previous equation by $\vecb{k}^\top$ on the left and by $\vecb{k}'$ on the right, we get that
\begin{align*}
\vecb{k}^\top[\matr{A}]_1([0]_2, [\vecb{x}']_2)\vecb{k}' - \vecb{k}^\top\pmatri{{[0]_1}\\{[1]_1}}[\vecb{x}^\top\matr{B}^\top]_2\vecb{k}'
&=
\vecb{k}^\top[\vecb{\theta}]_1[\vecb{v}^\top_2]\vecb{k}'-\vecb{k}^\top[\vecb{u}]_1[\vecb{\pi}^\top]_2\vecb{k}'\\
[\vecb{k}^\top\matr{A}\vecb{x}']_T(0, 1)\vecb{k}' - \vecb{k}^\top\pmatri{0\\1}[\vecb{x}^\top\matr{B}^\top\vecb{k}']_T
&=0\\
[g_\vecb{\matr{A}}(\vecb{x}')]_T - [g_\matr{A}(\vecb{x})]_T &= 0.
\end{align*}
Note that the previous idendity follows from the fact that $g_{\matr{B}}(\vecb{x}) = {\vecb{k}'}^\top\matr{B}\vecb{x}=\vecb{\beta}^\top\vecb{x}=g_\matr{A}(\vecb{x})$ and that $\vecb{k},\vecb{k}'$ are decryption keys and thus $\vecb{k}^\top\vecb{u}=\vecb{v}^\top\vecb{k}'=0$ and $\vecb{k}^\top\smallpmatrix{0\\1}=(0,1)\vecb{k}'=1$.

Given a collision $[\vecb{\beta}]_2,[\vecb{\beta}']_2$ for $g$, then $([\vecb{\beta}]_2-[\vecb{\beta}]'_2)\neq [\vecb{0}]$ is in the kernel of $[\matr{A}]_1$. Therefore, is trivial to prove that for any adversary $\advA$ there is an adversary $\advB$ such that $\adv^{\mathsf{Col}_g}(\advA)=\adv_{\mathcal{Q}_m^\top\mbox{-}\skermdh}(\advB)$, whenever $\matr{A}\gets\mathcal{Q}_m$.

We note that given $A\in Q_m,[\matr{A}]_1\in\GG^{2\times m}_1,[\vecb{\beta}]_2\in\GG^m_2$, $[\vecb{y}]_1\in\GG_2^2$ and $[\vecb{y}']_1\in\GG^1_2$ one can express the statements $A\in Q_m$, $g_{[\matr{A}]_1}([\vecb{\beta}]_2)=[\vecb{y}]_2$, and $h(A)=[\vecb{y}']_1$ as (\ref{eq:Q}),(\ref{eq:g}), and (\ref{eq:h}), respectively.
 \begin{align}
&e([a_{1}]_1,[1]_2) = e([1]_1,[{b}_1]_2)\text{ and }e([a_{2}]_1,[1]_2)=e([a_{1}]_1,[b_1]_2)
\text{ for each }([\vecb{a}]_1,[\vecb{\beta}]_2)\in A \label{eq:Q}\\
&\sum_{j=1}^m e([a_{i,j}]_1,[x_i]_1) = e([1]_1,[y_i]_2) \text{ for each } i\in\{1,2\} \label{eq:g}\\
&\sum_{([\vecb{a}]_1,[\vecb{a}]_2)\in A} [a_i]_1 = [y'_i]_1 \text{ for each } i\in\{1,2\}.\label{eq:h}
\end{align}
Hence, one can compute Groth-Sahai proofs of size $\Theta(m),\Theta(1)$, and $\Theta(1)$, respectively, for the satisfiability of each statement.

Finally, we prove a simple lemma that relates both functions
\begin{lemma}\label{lemma:hg}
Let $A\gets Q_m,A'\in Q_m,[\vecb{\beta}]_2,[\vecb{\beta}']_2\in\GG^m_2$, and $[\matr{A}]_1,[\matr{A}']_1$ the matrices whose columns are the first component of the elements of $A$ and $A'$, respectively. Then $h(A)=h(A')$ and $g_{[\matr{A}]_1}([\vecb{\beta}]_2)=g_{[\matr{A}']_1}([\vecb{\beta}']_2)$ implies that $A'$ is a second preimage of $h(A)$ or there exists a permutation matrix $\matr{P}$ such that $g_{[\matr{A}]_1}([\vecb{\beta}]_2)=g_{[\matr{A}]_1}([\matr{P}\vecb{\beta}']_2)$.
\end{lemma}
\begin{proof}
If $A\neq A'$, then $A'$ is a second preimage of $h(A)$. Else, there is a permutation matrix $\matr{P}$ such that $[\matr{A}']_1 =[\matr{A}\matr{P}]_1$. Then
$$
 g_{[\matr{A}]_1}([\vecb{\beta}]_2)=g_{[\matr{A}']_1}([\vecb{\beta}']_2)\Longleftrightarrow  g_{[\matr{A}]_1}([\vecb{\beta}]_2)=g_{[\matr{A}\matr{P}]_1}([\vecb{\beta}']_2)=g_{[\matr{A}]_1}([\matr{P}\vecb{\beta}']_2).
$$
\end{proof}