% !TEX root = ../main-ring-signature.tex

The Groth Sahai (GS) proof system is a non-interactive witness indistinguishable proof system (and in some cases also zero-knowledge) for the language of quadratic equations over a bilinear group. The admissible equation types must be in the following form:
\begin{equation}\label{gseq}
\sum_{j=1}^{m_y} f(\alpha_j, \vary_j)+\sum_{i=1}^{m_x} f(\varx_i, \beta_i)+\sum_{i=1}^{m_x} \sum_{j=1}^{m_y}  f(\varx_i,\escQE_{i,j} \vary_j)=t,
\end{equation}
 where $\boldsymbol \alpha  \in \Am_1^{m_y}$, $\boldsymbol \beta  \in \Am_2^{m_x}$, $\matr{\EscQE}=(\escQE_{i,j}) \in \Z_q^{m_x\times m_y}$, $t \in \Am_T$, and $\Am_1,\Am_2,\Am_T\in\{\Z_q,\GG_1,\GG_2, \GG_T\}$ 
are equipped with some bilinear map $f:\Am_1\times \Am_2 \rightarrow \Am_T$.

The GS proof system is a \emph{commit-and-prove} proof system, that is, the prover first commits to solutions
of equation (\ref{gseq}) using the GS commitments, and then computes a proof that the committed values satisfies equation (\ref{gseq}).

GS proofs are perfectly sound when the CRS is sampled from the perfectly binding distribution, and perfectly witness-indistinguishable when sampled from the perfectly hiding distribution. Computational indistinguishability of  both distributions implies either perfect soundness and computational witness indistinguishability or computational soundness and perfect witness-indistinguishability.

Further, Belenky et al.~noted that Groth-Sahai proofs can be \emph{re-randomized} \cite{C:BCCKLS09}. This means that, given commitments and proofs showing the satisfiability of some equation, on can compute new proofs which looks exactly as fresh proofs (i.e. computed with fresh randomness) for the same equation, even without knowing the commitment openings nor the randomness. In this work compute such proofs for integer equations $\beta(\beta-1)=0$ and $\beta x = y$. For completeness, in App.~\ref{sec:GSproofs-hg} we show how to construct and re-randomize such proofs.

\subsection{Groth-Sahai Commitments.}
Following Groth and Sahai's work \cite{EC:GroSah08}, in symmetric groups and using the SXDH assumption, GS commitments are vectors in $\GG^2_\gamma$, $\gamma\in\{1,2\}$, the form
\begin{small}\begin{align*}
&\GS.\Com_{ck_\gamma}([x]_\gamma;\vecb{r}):=\pmatri{{[0]_\gamma}\\{[x]_\gamma}}+r_\gamma\left[\vecb{u}_1-\pmatri{0\\1}\right]_\gamma+{r}_2[\vecb{u}_2]_\gamma\\
&\GS.\Com_{ck_\gamma}(x;\vecb{r}):=x[\vecb{u}_1]_\gamma+{r}[\vecb{u}_2]_\gamma
%& \GS.\Com_{ck_2}([x]_2;\vecb{r}):=\pmatri{{[0]_2}\\{[x]_2}}+r_1\left[\vecb{v}_1-\pmatri{0\\1}\right]_2+{r}_2[\vecb{v}_2]_2\\
%&\GS.\Com_{ck_2}(x;\vecb{r}):=x[\vecb{v}_1]_2+{r}[\vecb{v}_2]_2
\end{align*}\end{small}
where $ck_\gamma:=[\vecb{u}_1\cat\vecb{u}_2]_\gamma$, and $\vecb{u}_2$ are sampled from the same distribution as $\matr{A}$, the matrix from definition \ref{def:dlin}. The GS reference string is formed by the commitment keys $ck_1,ck_2$  and $\vecb{u}_1:=w\vecb{u}_2+\vecb{e}_2$ in the perfectly binding setting, and $\vecb{u}_1:=w\vecb{u}_2$ in the perfectly hiding setting, for $w\gets\Z_q$.

We define commitments to row vectors as the horizontal concatenation of commitments to each of the coordinates. That is, for $\vecb{x}\in\Z_q^m$ and $\vecb{r}\in\Z_q^m$
\begin{align*}
\GS.\Com_{ck_\gamma}(\vecb{x}^\top;\vecb{r}^\top) := [\vecb{u}_1]_\gamma\vecb{x}^\top+[\vecb{u}_2]_\gamma\vecb{r}^\top\in\GG_\gamma^{2\times m}.
\end{align*}

Given a Groth-Sahai commitment $[\vecb{c}]_\gamma$, we will say that $[\vecb{c}']_\gamma$ is a re-randomization of $[\vecb{c}]_\gamma$ if $[\vecb{c}']_\gamma = [\vecb{c}]_\gamma + \GS.\Com_{ck_s}(0;\delta)$, for $\delta\gets\Z_q$.
