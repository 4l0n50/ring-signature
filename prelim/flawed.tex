Bose et al.~claim to construct a constant-size ring signature in the standard model \cite{ACISP:BosDasRan15}. However, they construct a weak ring signature where: a) the public keys are generated all at once in a correlated way; b) the set of parties which are able to participate in a ring is fixed as well as the maximum ring size; and c) the key size is linear in the maximum ring size. In the work of Chandran et al.~and also in our setting: a) the key generation is independently run by the user using only the CRS as input; b) any party can be member of the ring as long as she has a verification key, and the maximum ring size is unbounded; and c) the key size is constant. These stronger requirements are in line with the original spirit of {non-coordination} of  Rivest et al.~\cite{AC:RivShaTau01}.

Gritti et al.~claim to construct a logarithmic ring signature in the standard model \cite{IET:GriSusPla16}. However, their construction is flawed as explained below.\footnote{We use multiplicative notation for the group operations to keep the expressions as they appear in the original work.}
In page 12, Gritti et al.~define $v_{b_i} := v_{b_1\cdots b_i *}$, where $b_1\cdots b_i *$ is the set of all bit-strings of size $d:=\log n$ whose prefix is $b_1\cdots b_i$. From this, one has to conclude that $v_{b_i}$ is a set (or vector) of group elements of size $2^{d-i}$.
In the same page they define the commitment $D_{b_i} := v_{b_i}h^{s_{b_i}}$, for random $s_{b_i}\in\Z_q$, which, according to the previous observation, is the multiplication of a set (or vector) of group elements with a group element. Given that length reducing group to group commitments are known to not exist \cite{EC:AbeHarOhk12}, its representation requires at least $2^{d-i}$ group elements.\footnote{In fact, there exists length reducing group to group commitments \cite{EC:AKOT15} with a weaker binding property, but is far from clear how to use these commitments in the Gritti et al.'s work} Since commitments $D_{b_0},\ldots,D_{b_d}$ are part of the signature, the actual signature size is $\Theta(2^d)=\Theta(n)$, rather than  $\Theta(d)=\Theta(\log n)$ as claimed by Gritti et al.

