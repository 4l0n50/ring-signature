Boneh and Boyen introduced a short signature -- each signature consists of only one group element -- which is secure against existential forgery under weak chosen message attacks without random oracles \cite{EC:BonBoy04a}.
The verification of the validity of any signature-message pair can be written as a set of pairing product equations. Thereby, using Groth-Sahai proofs one can show the possession of a valid signature without revealing the actual signature.

We construct our ring signature using Boneh-Boyen signatures, but we could replace the Boneh-Boyen signature scheme with a structure preserving signature scheme secure under milder assumptions (e.g.~\cite{EPRINT:JutRoy17}). We rather keep it simple and stick to Boneh-Boyen signature which, since the verification key is just one group element, simplifies the notation and reduces the size of the final signature.
 
\begin{definition}[weak Existential Unforgeability (wUF-CMA)] We say that a signature scheme $\Sigma = (\mathsf{KGen},\mathsf{Sign},\mathsf{Ver})$ is wUF-CMA if for any PPT adversary $\advA$
	$$
	\Pr\left[\begin{array}{l}
	gk \gets \ggen_s(1^\lambda), (m_1,\ldots,m_{q_\mathsf{sig}})\gets\advA(gk), (sk,vk)\gets\KGen(1^\lambda), \\
	(m,\sigma)\gets\advA(\Sign_{sk}(m_1),\ldots,\Sign_{sk}(m_{q_\mathsf{sig}})):\\
	\Ver_{vk}(m,\sigma)=1 \text{ and } m\notin \{m_1,\ldots,m_{q_\mathsf{sig}}\}
	\end{array}\right]
	$$
is negligible in $\lambda$.
\end{definition}

The Boneh-Boyen signature is proven wUF-CMA secure under the $m$-\emph{strong Diffie-Hellman} assumption, which is described below.

\begin{definition}[$m\mbox{-}SDH$ assumption]
For any PPT adversary $\advA$
$$
\Pr\left[gk\gets\G_s(1^\lambda),x\gets\Z_q:\advA(gk,[x],[x^2],\ldots,[x^m])=(c,\left[\frac{1}{x+c}\right])\right]
$$
is negligible in $\lambda$.
\end{definition}

The Boneh-Boyen signature scheme is described below.

\begin{description}
\item[$\mathsf{BB}.\KG$:] Given a group key $gk$, pick $vk\gets\Z_q$. The secret/public key pair is defined as $(sk,[vk]):=(vk,[vk])$.
\item[$\mathsf{BB}.\Sign$:] Given a secret key $sk\in\Z_q$ and a message $m\in\Z_q$, output the signature $[\sigma]:=\left[\frac{1}{sk+m}\right]$. In the unlikely case that $sk+m=0$ we let $[\sigma]:=[0]$.
\item[$\mathsf{BB}.\Ver$:] On input the verification key $[vk]$, a message $m\in\Z_q$, and a signature $[\sigma]$, verify that $[m+vk][\sigma]=[1]_T$.
\end{description} 

