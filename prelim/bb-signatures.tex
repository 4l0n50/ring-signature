% !TEX root = ../main-ring-signature.tex


Boneh and Boyen introduced a short signature --- each signature consists of only one group element --- which is secure against existential forgery under weak chosen message attacks without random oracles \cite{EC:BonBoy04a}.
The verification of the validity of any signature-message pair can be written as a set of pairing product equations. Thereby, using Groth-Sahai proofs one can show the possession of a valid signature without revealing the actual signature.

We construct our ring signature using Boneh-Boyen signatures, but we could replace the Boneh-Boyen signature scheme with a structure preserving signature scheme secure under milder assumptions (e.g.~\cite{EPRINT:JutRoy17}). We rather keep it simple and stick to Boneh-Boyen signature which, since the verification key is just one group element, simplifies the notation and reduces the size of the final signature.
 
\begin{definition}[weak Existential Unforgeability (wUF-CMA)] We say that a signature scheme $\Sigma = (\mathsf{KGen},\mathsf{Sign},\mathsf{Ver})$ is wUF-CMA if for any PPT adversary $\advA$
	$$
	\Pr\left[\begin{array}{l}
	gk \gets \ggen_a(1^\lambda), (m_1,\ldots,m_{q_\mathsf{sig}})\gets\advA(gk), (sk,vk)\gets\KGen(1^\lambda), \\
	(m,\sigma)\gets\advA(\Sign_{sk}(m_1),\ldots,\Sign_{sk}(m_{q_\mathsf{sig}})):\\
	\Ver_{vk}(m,\sigma)=1 \text{ and } m\notin \{m_1,\ldots,m_{q_\mathsf{sig}}\}
	\end{array}\right]
	$$
is negligible in $\lambda$.
\end{definition}

The Boneh-Boyen signature is proven wUF-CMA secure under the $m$-\emph{strong Diffie-Hellman} assumption, which is described below.

\begin{definition}[$m\mbox{-}SDH$ assumption]
For any PPT adversary $\advA$
$$
\Pr\left[gk\gets\G_a(1^\lambda),x\gets\Z_q:\advA(gk,[x]_{3-s},[x]_s,[x^2]_s,\ldots,[x^m]_s)=(c,\left[\frac{1}{x+c}\right]_s)\right]
$$
is negligible in $\lambda$.
\end{definition}

Given $s\in\{1,2\}$, the Boneh-Boyen signature scheme is described below.

\begin{description}
\item[$\mathsf{BB}.\KG$:] Given a group key $gk$, pick $vk\gets\Z_q$. The secret/public key pair is defined as $(sk,[vk]):=(vk,[vk]_{3-s})$.
\item[$\mathsf{BB}.\Sign$:] Given a secret key $sk\in\Z_q$ and a message $m\in\Z_q$, output the signature $[\sigma]_{s}:=\left[\frac{1}{sk+m}\right]_{s}$. In the unlikely case that $sk+m=0$ we let $[\sigma]_{s}:=[0]_{s}$.
\item[$\mathsf{BB}.\Ver$:] On input the verification key $[vk]_{3-s}$, a message $m\in\Z_q$, and a signature $[\sigma]_{s}$, verify that $[m+vk]_{3-s}[\sigma]_{s}=[1]_T$.
\end{description} 

It is direct to prove knowledge of a Boneh-Boyen signature for some message $m$ under some committed verification key with a Groth-Sahai proof for the verification equation. In our SXDH based ring signature we need to prove a slightingly different statement. Since we have a commitment to the secret key $[\vecb{c}]_{2} = \Com_{ck_2}(sk;s) = sk[\vecb{v}_1]_2+s[\vecb{v}_2]_2$ we need to show that
\begin{equation}
e([{\sigma}]_1, m[\vecb{v}_1]_2 + [\vecb{c}]_2) - [\vecb{v}_1]_T= e([r]_1,[v_2]_2),
\label{eq:bbs-verification}
\end{equation}
for some $s\in\Z_q$.
