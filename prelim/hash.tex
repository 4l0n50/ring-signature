We recall the definition of a hash function and also a weaker notion from the work of Rogaway and Shrimpton \cite{FSE:RogShr04}. For both defintions we consider functions $g,h:\mathcal{K}\times\mathcal{M}\to\mathcal{Y}$ and an algorithm $\KGen$ which on input $1^\lambda$ randomly samples an element from $\mathcal{K}$.

\begin{definition}[Collision Resistance]\label{def:hash1}
 We say that $g$ is a hash-function family with collision resistance if for all PPT adversary $\advA$
$$
\adv_g^{\mathsf{Col}}(\advA) := \Pr[k\gets\KGen(1^\lambda), (x,x')\gets A(k):g_k(x)=g_k(x')]
$$ 
is negligible in $\lambda$.
\end{definition}

\begin{definition}[Always Second-Preimage Resistance \cite{FSE:RogShr04}]\label{def:hash2}
 We say that $h$ is a hash-function family with always second-preimage resistance if for all PPT adversary $\advA$
$$
\adv_h^{\mathsf{aPre}}(\advA) := \Pr[k\gets\KGen(1^\lambda), x\gets\mathcal{M}, x'\gets A(x):h_k(x)=h_k(x')]
$$ 
is negligible in $\lambda$.
\end{definition}

We instantiate defintion \ref{def:hash1} with the function $g$ and \ref{def:hash2}  with $h$ defined as follows. For both functions $\mathcal{Y}=\GG^2$. In the case of $g$, $\mathcal{M}=\GG^m$ and $\KGen$ picks a group description $gk\gets\ggen_s(1^\lambda)$ together with $[\matr{A}]\in\GG^{2\times m}$, where $\matr{A}\gets\mathcal{Q}_m$, and the function is defined as
$$
g_{[\matr{A}]}([\vecb{x}]):= [\matr{A}\vecb{x}].
$$
Given a collision $[\vecb{x}],[\vecb{x}']$ for $g$, then $([\vecb{x}]-[\vecb{x}]')\neq [\vecb{0}]$ is in the kernel of $[\matr{A} ]$. Therefore, is trivial to prove that for any adversary $\advA$ there is an adversary $\advB$ such that $\adv^{\mathsf{Col}_g}(\advA)=\adv_{\mathcal{Q}_m^\top\mbox{-}\kermdh}(\advB)$, whenever $\matr{A}\gets\mathcal{Q}_m$.

 In the case of $h$, $\mathcal{M}=Q_m$ and $\KGen=\ggen_s$, and the function is defined as
$$
h(A):= \sum_{[\vecb{a}]\in A}[\vecb{a}].
$$
Given $A\gets Q_m$ and a second preimage $A'\in Q_m$ of $h(A)$, it is trivial to construct an adversary breaking the $m$-PPA assumption. Indeed, given $[\matr{A}]$ the challenge of the $m$-PPA assumption, then any matrix $[\matr{A}']$ whose columns are the elements of $A'$ is not a permutation of $[\matr{A}']$ and breaks $m$-PPA assumption. Then for any adversary $\advA$ there is an adversary $\advB$ such that $\adv^{\mathsf{aPre}_g}(\advA)=\adv_{m\mbox{-}\mathsf{PPA}}(\advB)$.

We note that given $A\in Q_m,[\matr{A}]\in\GG^{2\times m},[\vecb{x}]\in\GG^m$ and $[\vecb{y}]\in\GG^2$ one can express the staments $A\in Q_m$, $g_{[\matr{A}]}([\vecb{x}])=[\vecb{y}]$, and $h(A)=[\vecb{x}]$ as equations (\ref{eq:Q}),(\ref{eq:g}), and (\ref{eq:h}), respectively.
 \begin{align}
&e([a_{2,i}],[1])=e([a_{1,i}],[a_{1,i}]) \text{ for each } 1\leq i\leq m \label{eq:Q}\\
&\sum_{j=1}^m e([a_{i,j}],[x_i]) = e([y_i],[1]) \text{ for each } i\in\{1,2\} \label{eq:g}\\
&\sum_{[\vecb{a}]\in A} [a_i] = [y_i] \text{ for each } i\in\{1,2\}.\label{eq:h}
\end{align}
Thus, one can compute Groth-Sahai proofs of size $\Theta(m),\Theta(1)$, and $\Theta(1)$, respectively, for the satisfiability of each statement.

Finally, we prove a simple lemma informally stated in Section \ref{sec:tech-overview}.
\begin{lemma}\label{lemma:hg}
Let $A\gets Q_m,A'\in Q_m,[\vecb{x}],[\vecb{x}']\in\GG^m$, and $[\matr{A}],[\matr{A}']$ the matrices whose columns are the elements of $A$ and $A'$, respectively. Then $h(A)=h(A')$ and $g_{[\matr{A}]}([\vecb{x}])=g_{[\matr{A}']}([\vecb{x}'])$ implies that $A'$ is a second preimage of $h(A)$ or there exists a permutation matrix $\matr{P}$ such that $g_{[\matr{A}]}([\vecb{x}])=g_{[\matr{A}]}([\matr{P}\vecb{x}'])$.
\end{lemma}
\begin{proof}
If $A\neq A'$, then $A'$ is a second preimage of $h(A)$. Else, there is a permutation matrix $\matr{P}$ such that $[\matr{A}'] =[\matr{A}\matr{P}]$. Then
$$
 g_{[\matr{A}]}([\vecb{x}])=g_{[\matr{A}']}([\vecb{x}'])\Longleftrightarrow  g_{[\matr{A}]}([\vecb{x}])=g_{[\matr{A}\matr{P}]}([\vecb{x}'])=g_{[\matr{A}]}([\matr{P}\vecb{x}']).
$$
\end{proof}