% !TEX root = ../main-ring-signature.tex

We recall the definition of a hash function and also a weaker notion from the work of Rogaway and Shrimpton \cite{FSE:RogShr04}. For both definitions we consider functions $g,h:\mathcal{K}\times\mathcal{M}\to\mathcal{Y}$ and an algorithm $\KGen$ which on input $1^\lambda$ randomly samples an element from $\mathcal{K}$.

\begin{definition}[Collision Resistance]\label{def:hash1}
 We say that $g$ is a hash-function family with collision resistance if for all PPT adversary $\advA$
$$
\adv_g^{\mathsf{Col}}(\advA) := \Pr[k\gets\KGen(1^\lambda), (x,x')\gets A(k):g_k(x)=g_k(x')]
$$ 
is negligible in $\lambda$.
\end{definition}

\begin{definition}[Always Second-Preimage Resistance \cite{FSE:RogShr04}]\label{def:hash2}
 We say that $h$ is a hash-function family with always second-preimage resistance if for all PPT adversary $\advA$
$$
\adv_h^{\mathsf{aPre}}(\advA) := \Pr[k\gets\KGen(1^\lambda), x\gets\mathcal{M}, x'\gets A(x):h_k(x)=h_k(x')]
$$ 
is negligible in $\lambda$.
\end{definition}

We instantiate definition \ref{def:hash1} with the function $g$ and \ref{def:hash2}  with $h$ defined as follows. For both $g$, $\mathcal{Y}=\GG^2_2$ and for $h$, $\mathcal{Y}=\GG_1^2$. In the case of $g$, $\mathcal{M}=\GG^m_2$ and $\KGen$ picks a group description $gk\gets\ggen_a(1^\lambda)$ together with $[\matr{A}]_1\in\GG^{2\times m}_1$, where $\matr{A}\gets\mathcal{Q}_m$, and the function is defined as
$$
g_{[\matr{A}]_1}([\vecb{x}]_2):= [\matr{A}\vecb{x}]_2.
$$
Given a collision $[\vecb{x}]_2,[\vecb{x}']_2$ for $g$, then $([\vecb{x}]_2-[\vecb{x}]'_2)\neq [\vecb{0}]$ is in the kernel of $[\matr{A}]_1$. Therefore, is trivial to prove that for any adversary $\advA$ there is an adversary $\advB$ such that $\adv^{\mathsf{Col}_g}(\advA)=\adv_{\mathcal{Q}_m^\top\mbox{-}\kermdh}(\advB)$, whenever $\matr{A}\gets\mathcal{Q}_m$.

 In the case of $h$, $\mathcal{M}=Q_m$ and $\KGen=\ggen_a$, and the function is defined as
$$
h(A):= \sum_{([\vecb{a}]_1,[\vecb{a}]_2)\in A}[\vecb{a}]_1.
$$
Given a second preimage $h$, it is trivial to construct an adversary breaking the $m$-PPA assumption. Indeed, Let $[\matr{A}]_1,[\matr{A}]_2$ the challenge of the $m$-PPA assumption and let $A$ the set of columns of $[\matr{A}]_1$ and $[\matr{A}]_2$, which is clearly uniformly distributed in $Q_m$. Then given any $A'\in Q_m$ such that $A'\neq A$ and $h(A)=h(A')$, it holds that $[\matr{A}']_1$, the matrix whose columns are the first components of the elements of $A'$, is not a permutation of $[\matr{A}]_1$ and hence breaks $m$-PPA assumption. Then for any adversary $\advA$ there is an adversary $\advB$ such that $\adv^{\mathsf{aPre}_g}(\advA)=\adv_{m\mbox{-}\mathsf{PPA}}(\advB)$.

We note that given $A\in Q_m,[\matr{A}]_1\in\GG^{2\times m}_1,[\vecb{x}]_2\in\GG^m_2$, $[\vecb{y}]_1\in\GG_2^2$ and $[\vecb{y}']_1\in\GG^1_2$ one can express the statements $A\in Q_m$, $g_{[\matr{A}]_1}([\vecb{x}]_2)=[\vecb{y}]_2$, and $h(A)=[\vecb{y}']_1$ as (\ref{eq:Q}),(\ref{eq:g}), and (\ref{eq:h}), respectively.
 \begin{align}
&e([a_{1}]_1,[1]_2) = e([1]_1,[{b}_1]_2)\text{ and }e([a_{2}]_1,[1]_2)=e([a_{1}]_1,[b_1]_2)
\text{ for each }([\vecb{a}]_1,[\vecb{b}]_2)\in A \label{eq:Q}\\
&\sum_{j=1}^m e([a_{i,j}]_1,[x_i]_1) = e([1]_1,[y_i]_2) \text{ for each } i\in\{1,2\} \label{eq:g}\\
&\sum_{([\vecb{a}]_1,[\vecb{a}]_2)\in A} [a_i]_1 = [y'_i]_1 \text{ for each } i\in\{1,2\}.\label{eq:h}
\end{align}
Hence, one can compute Groth-Sahai proofs of size $\Theta(m),\Theta(1)$, and $\Theta(1)$, respectively, for the satisfiability of each statement.

Finally, we prove a simple lemma that relates both functions
\begin{lemma}\label{lemma:hg}
Let $A\gets Q_m,A'\in Q_m,[\vecb{x}]_2,[\vecb{x}']_2\in\GG^m_2$, and $[\matr{A}]_1,[\matr{A}']_1$ the matrices whose columns are the first component of the elements of $A$ and $A'$, respectively. Then $h(A)=h(A')$ and $g_{[\matr{A}]_1}([\vecb{x}]_2)=g_{[\matr{A}']_1}([\vecb{x}']_2)$ implies that $A'$ is a second preimage of $h(A)$ or there exists a permutation matrix $\matr{P}$ such that $g_{[\matr{A}]_1}([\vecb{x}]_2)=g_{[\matr{A}]_1}([\matr{P}\vecb{x}']_2)$.
\end{lemma}
\begin{proof}
If $A\neq A'$, then $A'$ is a second preimage of $h(A)$. Else, there is a permutation matrix $\matr{P}$ such that $[\matr{A}']_1 =[\matr{A}\matr{P}]_1$. Then
$$
 g_{[\matr{A}]_1}([\vecb{x}]_2)=g_{[\matr{A}']_1}([\vecb{x}']_2)\Longleftrightarrow  g_{[\matr{A}]_1}([\vecb{x}]_2)=g_{[\matr{A}\matr{P}]_1}([\vecb{x}']_2)=g_{[\matr{A}]_1}([\matr{P}\vecb{x}']_2).
$$
\end{proof}