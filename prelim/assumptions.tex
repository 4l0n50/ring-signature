% !TEX root = ../main-ring-signature.tex

We will use the natural translation to asymmetric groups of the permutation pairing assumption introduced by Groth and Lu. We formalize a much stronger assumption where the adversary may interactively ask for the discrete logs of $m-\ell$ elements from the challenge, and we call this assumption $(\ell,m)$-PPA.
Our natural non-interactive translation of the permutation pairing assumption becomes the non-interactive $(m,m)$-PPA assumption, which we simply call $m$-PPA, where the adversary hast no access to the discrete logs of the challenge.
\begin{definition}[Permutation Pairing Assumption \cite{AC:GroLu07}]\label{def:ppa}
Let $\ell,m$ integers such that $2\leq \ell \leq m$, let $\mathcal{Q}_{m}=\underbrace{\mathcal{Q}\cat\ldots\cat\mathcal{Q}}_{m\text{ times}}$, where concatenation of  distributions is defined in the natural way and 
$$\mathcal{Q}: \vecb{a}=\pmatri{x\\x^2},\quad x\gets\Z_q.$$
We say that the $(\ell,m)$-permutation pairing assumption holds relative to $\G_a$ if for any adversary $\advA$
\begin{small}$$
\Pr\left[
\begin{array}{l}
	gk\gets\G_a(1^\lambda);\matr{A}\gets\mathcal{Q}_{m};\\
	(i_1,\mathsf{st}_1) \gets \advA(gk,[\matr{A}]_1,[\matr{A}]_2),
	(i_2,\mathsf{st}_2) \gets \advA(\vecb{a}_{i_1},\mathsf{st}_2),
	\ldots,
	(i_{m-\ell},\mathsf{st}_{m-\ell}) \gets \advA(\vecb{a}_{i_{m-\ell-1}},\mathsf{st}_{m-\ell-1})\\
	%\begin{array}{c}\vdots\end{array}\\
	%\\
	([\matr{Z}]_1,[\underline{\vecb{z}}]_2)\gets\advA(gk,[\matr{A}]_1,[\matr{A}]_2, \vecb{a}_{i_1},\ldots, \vecb{a}_{i_{m-\ell}},\mathsf{st}_{m-\ell}):\\
	\mathrm{(i)} \sum_{i\in I}[\vecb{z}_i]_1 = \sum_{i\in I}[\vecb{a}_i]_1,\\
	\mathrm{(ii)}\ \forall i\in I \ [z_{1,i}]_1[1]_2=[1]_1[\underline{z}_{i}]_2 \text{ and } [z_{2,i}]_1[1]_2=[z_{1,i}]_1[\underline{z}_{i}]_2,\\
	\text{ and }\matr{Z}\text{ is not a permutation of the columns of }\matr{A}_I
\end{array}
\right],
$$\end{small}
where $I:=\{1,\ldots, m\}\setminus \{i_1,\ldots, i_{m-\ell} \}$, $[\matr{Z}]=[\vecb{z}_1\cat\cdots\cat\vecb{z}_\ell]_1\in\GG_1^{2\times \ell}, [\matr{A}]_1=[\vecb{a}_1\cat\cdots\cat\vecb{a}_m]_1\in\GG_1^{2\times m}$, $[\underline{\vecb{z}}]_2=\allowbreak[(\underline{z}_1,\ldots,\allowbreak \underline{z}_\ell)]_2\in\GG_2^{1\times \ell}$,
$\matr{A}_I\in \Z_q^{2\times \ell}$ is the matrix whose columns are $\vecb{a}_i$, $i\in I$ (in some fixed order),
 is negligible in $\lambda$.
\end{definition}
Groth and Lu proved the hardness of the PPA in generic symmetric bilinear groups \cite{AC:GroLu07}. In Appendix \ref{sec:aPPA} we show that the $(\ell,m)$-PPA in generic asymmetric groups is as hard as the PPA in generic symmetric groups. It follows that the $m$-PPA is also generically hard.

We recall also the definition of the Decisional Diffie-Hellman assumption (in matrix notation) and the kernel matrix Diffie-Hellman assumption.

\begin{definition}[Decisional Diffie-Hellman (DDH) in $\GG_s$]\label{def:dlin}
 Let  $\gk 
\gets \ggen_a(1^\lambda)$ and let $\matr{A} := (a,1)^\top$, $a\gets\mathbb{Z}_q$.
We say that the DDH assumption holds relative to $\ggen_a$ if for all PPT adversaries $\advD$
$$
\adv_{\mathrm{DDH},\ggen_s}(\advD) := |
	\Pr[
		\advD(
			gk,
			[\matr{A}]_s,
			[\matr{A}{w}]_s)=1]
	-\Pr[
		\advD(
		gk,
		[\matr{A}]_s,
		[\vecb{z}]_s)=1]|
$$
is negligible in $\lambda$, where the probability is taken over $gk\gets\ggen_a(1^\lambda)$, $a\gets\ZZ_q$, ${w}\gets\ZZ_q$, $[\vecb{z}]_2\gets\GG^2_s$, and the coin tosses of the adversary.
We say that the Symmetric eXternal Diffie-Hellman (SXDH) assumption holds if the DDH assumption holds in both $\GG_1$ and $\GG_2$.
\end{definition}

We recall a family of assumptions which are weaker than the so called \emph{Matrix Diffie-Hellman} assumptions (a generalization of the DDH and DLin assumptions introduced by Escala et al.~\cite{C:EHKRV13}).
\begin{definition}[Kernel Diffie-Hellman Assumption in $\GG_s$ \cite{AC:MorRafVil16}] Let  $\gk 
\gets\ggen_a(1^\lambda)$ and $\dist_{\ell,k}$ a distribution over $\Z_q^{\ell\times k}$.
The Kernel Diffie-Hellman assumption in $\GG_s$ ($\dist_{\ell,k}\mbox{-}\kermdh_{\GG_s}$) says that every PPT Algorithm has negligible advantage in the following  game: given $[\matr{A}]_s$, where $\matrA \gets \dist_{\ell,k}$, find $[\vecb{x}]_{3-s} \in \GG^{\ell}_{3-s}$, $\vecb{x} \neq \vecb{0}$, such that 
$[\vecb{x}]_{3-s}^{\top}[\matr{A}]_s=[\vecb{0}]_T$. 
\end{definition}
Although using a different notation, the $Q_m^\top\mbox{-}\kermdh$ in symmetric groups was introduced by Groth and Lu \cite{AC:GroLu07}.
We will be using a natural translation of the $Q_m^\top\mbox{-}\kermdh$ assumption to asymmetric groups, where  $[\matr{A}]_s$ is also given in $\GG_{3-s}$.  Such assumption is a weaker variant of a \emph{split} $\kermdh$ assumption, introduced in \cite{AC:GonHevRaf15}, where the adversary might find an element in $\mathrm{Ker}(\matr{A})$ which is splitted between $\GG_1$ and $\GG_2$.

\begin{definition}[Split Kernel Diffie-Hellman Assumption \cite{AC:GonHevRaf15}] Let  $\gk 
\gets\ggen_a(1^\lambda)$ and $\dist_{\ell,k}$ a distribution over $\Z_q^{\ell\times k}$.
The Split Kernel Diffie-Hellman assumption ($\dist_{\ell,k}\mbox{-}\skermdh$) says that every PPT Algorithm has negligible advantage in the following  game: given $[\matr{A}]_1,[\matr{A}]_2$, where $\matrA \gets \dist_{\ell,k}$, find $[\vecb{x}]_1\in\GG_1^\ell,[\vecb{y}]_2 \in \GG^{\ell}_{2}$, $\vecb{x}\neq\vecb{y}$, such that 
$[\vecb{x}]_{1}^{\top}[\matr{A}]_1=[\vecb{y}]_2^\top[\matr{A}]_2$. 
\end{definition}
 Our weaker variant restrict the adversary to give solutions only in $\GG_1$ (i.e.~$[\vecb{y}]_2=0$), while we simply refer to it as the $Q_m^\top\mbox{-}\skermdh$.

Groth and Lu proved the hardness of the $Q_m^\top\mbox{-}\kermdh$ assumption in generic symmetric groups, and Gonz\'alez et al.~proved that, in generic asymmetric groups, the $\dist_{\ell,k}\mbox{-}\skermdh$ is as hard as the $\dist_{\ell,k}\mbox{-}\kermdh$ assumption in symmetric groups, for any distribution $\dist_{\ell,k}$. We conclude that the $Q_m^\top\mbox{-}\skermdh$ is hard in generic asymmetric groups (and of course, the weaker variant that we will be using).