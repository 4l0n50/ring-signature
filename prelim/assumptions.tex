We will use the permutation pairing assumption, introduced by Groth and Lu for constructing NIZK proofs of correcness of a shuffle.
\begin{definition}[Permutation Pairing Assumption \cite{AC:GroLu07}]\label{def:ppa}
Let $\mathcal{Q}_{m}=\underbrace{\mathcal{Q}\cat\ldots\cat\mathcal{Q}}_{m\text{ times}}$, where concatenation of  distributions is defined in the natural way and 
$$\mathcal{Q}: \vecb{a}=\pmatri{x\\x^2},\quad x\gets\Z_q.$$
We say that the $m$-permutation pairing assumption holds relative to $\G_s$ if for any adversary $\advA$
$$
\Pr\left[
\begin{array}{l}
gk\gets\G_s(1^\lambda);\matr{A}\gets\mathcal{Q}_{m};[\matr{Z}]\gets\advA(gk,[\matr{A}]):\\
\mathrm{(i)} \sum_{i=1}^{m}[\vecb{z}_i]=\sum_{i=1}^{m}[\vecb{a}_i], \mathrm{(ii)}\ \forall 1\leq i\leq m\ [z_{2,i}][1]=[z_{1,i}][z_{1,i}],\\
\text{ and }\matr{Z}\text{ is not a permutation of the columns of }\matr{A}
\end{array}
\right],
$$
where $[\matr{Z}]=[(\vecb{z}_1,\ldots,\vecb{z}_m)], [\matr{A}]=[(\vecb{a}_1,\ldots,\vecb{a}_m)]\in\GG^{2\times m}$,
is negligible in $\lambda$.
\end{definition}
Groth and Lu proved the hardness of the permutation pairing assumption is generic bilinear groups. 

We recall the definition of the decisional linear assumption (in matrix notation) and the kernel matrix Diffie-Hellman assumption.

\begin{definition}[Decisional Diffie-Hellman Assumption (DLin)]\label{def:dlin}
 Let  $\gk 
\gets \ggen_s(1^\lambda)$ and let
$$
\matr{A} :=
\begin{pmatrix} 
a_1 & 0     \\
0     & a_2 \\
1     &  1
\end{pmatrix},
\quad
a_1,a_2\gets\mathbb{Z}_q.
$$
We say that the DLin assumption holds relative to $\ggen_s$ if for all PPT adversaries $\advD$
$$
\adv_{\mathrm{DLin},\ggen_s}(\advD) := |
	\Pr[
		\advD(
			gk,
			[\matr{A}],
			[\matr{A}\vecb{w}])=1]
	-\Pr[
		\advD(
		gk,
		[\matr{A}],
		[\vecb{z}])=1]|
$$
is negligible in $\lambda$, where the probability is taken over $gk\gets\ggen_s(1^\lambda)$, $a_1,a_2\gets\ZZ_q$, $\vecb{w}\gets\ZZ_q^2$, $[\vecb{z}]\gets\GG^3$, and the coin tosses of the adversary. 
\end{definition}

\begin{definition}[Kernel Diffie-Hellman Assumption in $\GG$ \cite{EPRINT:MorRafVil15}] Let  $\gk 
\gets\ggen_s(1^\lambda)$ and $\dist_{\ell,k}$ a distribution over $\Z_q^{\ell\times k}$.
The Kernel Diffie-Hellman assumption in $\GG$ ($\dist_{\ell,k}\mbox{-}\kermdh_{\GG_\gamma}$) says that every PPT Algorithm has negligible advantage in the following  game: given $[\matr{A}]$, where $\matrA \gets \dist_{\ell,k}$, find $[\vecb{x}] \in \GG^{\ell}$, $\vecb{x} \neq \vecb{0}$, such that 
$[\vecb{x}]^{\top}[\matr{A}]=[\vecb{0}]_T$. 
\end{definition}

We will be using the $Q_m^\top\mbox{-}\kermdh$ assumption, which was proven secure in the generic bilinear group model by Groth and Lu \cite{AC:GroLu07}.