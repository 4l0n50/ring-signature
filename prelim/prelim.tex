We write PPT as a shortcut for probabilistic polynomial time Turing machine.

Let $\ggen_s$ be some probabilistic polynomial time algorithm which on input $1^{\lambda}$, where $\lambda$ is the security parameter, returns the \emph{group key} which is the description of a symmetric bilinear group $gk:=(q,\GG,\GG_T,e,\mathcal{P})$, where $\GG$
and $\GG_T$ are groups of prime order $q$, the element $\mathcal{P}$ is a generator of 
$\GG$, and $e:\GG\times\GG\to\GG_T$ is an efficiently computable and non-degenerated bilinear map. We will use additive notation for the group operation of both $\GG$ and $\GG_T$.

Elements in $\GG$ are denoted implicitly as $[a]:=a \Pt$, where $a\in\Z_q$, and elements in $\GG_T$ are denoted as $[a]_T:=a\cdot e(\Pt,\Pt)$. 
The pairing operation is written as a product $\cdot$, that is $[a] \cdot [b]=[a] [b]=e([a],[b])=[ab]_T$. Vectors and matrices are denoted in boldface. Given a matrix $\matr{T}=(t_{i,j})$, $[\matr{T}]$ is
the natural embedding of $\matr{T}$ in $\GG$, that is, the matrix whose $(i,j)$th entry is $t_{i,j}\mathcal{P}$. Given a matrix $\matr{S}$ with the same number of rows as $\matr{T}$, we define $\matr{S}\cat\matr{T}$ as the concatenation of $\matr{S}$ and $\matr{T}$.