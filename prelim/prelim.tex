We write PPT as a shortcut for probabilistic polynomial time Turing machine.

Let $\ggen_s$ be some probabilistic polynomial time algorithm which on input $1^{\lambda}$, where $\lambda$ is the security parameter, returns the \emph{group key} which is the description of a symmetric bilinear group $gk:=(q,\GG,\GG_T,e,\mathcal{P})$, where $\GG$
and $\GG_T$ are groups of prime order $q$, the element $\mathcal{P}$ is a generator of 
$\GG$, and $e:\GG\times\GG\to\GG_T$ is an efficiently computable and non-degenerated bilinear map.

Elements in $\GG$ are denoted implicitly as $[a]:=a \Pt$, where $a\in\Z_q$, and elements in $\GG_T$ are denoted as $[a]_T:=a\Pt$, where $\Pt_T:=e(\Pt,\Pt)$. 
The pairing operation iso written as a product $\cdot$, that is $[a] \cdot [b]=[a] [b]=e([a],[b])=[ab]_T$. Vectors and matrices are denoted in boldface. Given a matrix $\matr{T}=(t_{i,j})$, $[\matr{T}]$ is
the natural embedding of $\matr{T}$ in $\GG$, that is, the matrix whose $(i,j)$th entry is $t_{i,j}\mathcal{P}$. Given a matrix $\matr{S}$ with the same number of rows as $\matr{T}$, we define $\matr{S}\cat\matr{T}$ as the concatenation of $\matr{S}$ and $\matr{T}$.

We recall the definition of the decisional linear assumption (in matricial notation) and the kernel matrix Diffie-Hellman assumption.

\begin{definition}[Decisional Diffie-Hellman Assumption (DLin)]\label{def:dlin}
 Let  $\gk 
\gets \ggen_s(1^\lambda)$ and let
$$
\matr{A} :=
\begin{pmatrix} 
a_1 & 0     \\
0     & a_2 \\
1     &  1
\end{pmatrix},
\quad
a_1,a_2\gets\mathbb{Z}_q.
$$
We say that the DLin assumption holds relative to $\ggen_s$ if for all PPT adversaries $\advD$
$$
\adv_{\mathrm{DLin},\ggen_s}(\advD) := |
	\Pr[
		\advD(
			gk,
			[\matr{A}],
			[\matr{A}\vecb{w}])=1]
	-\Pr[
		\advD(
		gk,
		[\matr{A}],
		[\vecb{z}])=1]|
$$
is negligible in $\lambda$, where the probability is taken over $gk\gets\ggen_s(1^\lambda)$, $a_1,a_2\gets\ZZ_q$, $\vecb{w}\gets\ZZ_q^2$, $[\vecb{z}]\gets\GG^3$, and the coin tosses of the adversary. 
\end{definition}

\begin{definition}[Kernel Diffie-Hellman Assumption in $\GG$ \cite{EPRINT:MorRafVil15}] Let  $\gk 
\gets\ggen_s(1^\lambda)$ and $\dist_{\ell,k}$ a distribution over $\Z_q^{\ell\times k}$.
The Kernel Diffie-Hellman assumption in $\GG$ ($\dist_{\ell,k}\mbox{-}\kermdh_{\GG_\gamma}$) says that every PPT Algorithm has negligible advantage in the following  game: given $[\matr{A}]$, where $\matrA \gets \dist_{\ell,k}$, find $[\vecb{x}] \in \GG^{\ell}$, $\vecb{x} \neq \vecb{0}$, such that 
$[\vecb{x}]^{\top}[\matr{A}]=[\vecb{0}]_T$. 
\end{definition}

We will be using the $Q_m^\top\mbox{-}\kermdh$ assumption, which was proven secure in the generic bilinear group model by Groth and Lu \cite{AC:GroLu07}.

%$\matr{I}_{n\times n}$ refers to the identity matrix in $\Z_q^{n\times n}$, $\matr{0}_{m\times n}$ and $\matr{1}_{m\times n}$ the all-zero and all-one matrices in $\Z_q^{m\times n}$, respectively, and $\vecb{e}^{n}_i$ the $i$th element 
%of the canonical basis of $\Z_q^{n}$ (simply $\matr{I}$, $\matr{0}$, $\matr{1}$, and $\vecb{e}_i$, respectively, if $m$ and $n$ are clear from the context). 
%Given some matrices $\matr{A}\in\Z_q^{m\times t},\matr{A}_1\in\Z_q^{m_1\times t},\ldots,\matr{A}_n\in\Z_q^{m_n\times n}$, we define the operations
% $$\vecb{A}_1 \oplus \ldots \oplus \vecb{A}_n:=\smallpmatrix{ \vecb{A}_1 \\ \vdots \\  \vecb{A}_n} \qquad 
%\matr{A}^n:=\smallpmatrix{ \matr{A} &  & \matr{0} \\   & \ddots &   \\ \matr{0} &  & \matr{A}
%}.$$
%$\vecb{A}_1 \oplus \ldots \oplus \vecb{A}_n:=\smallpmatrix{ \vecb{A}_1 \\ \vdots \\  \vecb{A}_n}$ and $\matr{A}^n:=\smallpmatrix{ \matr{A} &  & \matr{0} \\   & \ddots &   \\ \matr{0} &  & \matr{A}}.$
%
%We make extensive use of the set $[m+k]\times[m+k]\setminus\{(i,i):i\in[n]\}$ and for brevity we denote it by $\indexSet{m}{k}$.
%
%We write $\uvecb{T}\in\Z_q^{mn}$ for the \emph{vectorization} of $\matr{T}\in\Z_q^{m\times n}$, that is
%$
%\uvecb{T}:= (t_{1,1},\ldots,t_{m,1},$ $t_{1,2},\ldots,t_{m,2},\ldots,t_{1,n},\ldots,t_{m,n})^\top
%$. We will use the following fact about vectorizations
%\begin{fact}
%For any $\matr{A}\in\Z_q^{\ell\times n},\matr{B}\in\Z_q^{\ell\times m},\matr{C}\in\Z_q^{m\times n}$
%$$
%\matr{A}=\matr{B}\matr{C}
%\Longleftrightarrow
%\uvecb{A} = (\matr{I}_{n\times n}\otimes\matr{B})\uvecb{C} = 
%\begin{pmatrix}
%\matr{B}                & \matr{0}_{\ell\times m} & \cdots & \matr{0}_{\ell\times m}\\
%\matr{0}_{\ell\times m} & \matr{B}                & \cdots & \matr{0}_{\ell\times m}\\
%\vdots                  & \vdots                  & \ddots & \vdots                 \\
%\matr{0}_{\ell\times m} & \matr{0}_{\ell\times m} & \cdots & \matr{B}
%\end{pmatrix}
%\uvecb{C},
%$$
%where $\otimes$ denotes the Kronecker product.
%\end{fact}
%
%
%
