% !TEX root = main-ring-signature.tex


Kalai et al.~construct the first publicly verifiable non-interactive delegation scheme from a standard assumption \cite{EPRINT:KalPanYan18}. Although we do not know if their techniques can be extended to NIZK proofs, it seems quite plausible. While they consider the most general case of delegating computation for any bounded depth circuit (uniformly generated by a log-space turing machine) through a delegating scheme for the universal circuit, we concentrate on their results for delegating computation of a single circuit encoded in the reference string. %Further, while is true that the depth of the circuit is constant, in the quasi-adaptive setting \cite{AC:JutRoy13} the language and the crs the size of the circuit inputs $n$ is dynamically chosen when generating the crs and in this case it makes sense to say that the depth of the circuit might be polynomially related with $n$. So their scheme is also a quasi-adaptive public verifiable delegation scheme for NP.

For a circuit $C$ of size $s$ and depth $d$, the prover's runtime is $\mathsf{poly}(s)$, the verifier's runtime is $(d+n)\mathsf{polylog}(s)$, and the communication complexity is $d\mathsf{polylog(s)}$. They proof system is constructed from the celebrated (sic) interactive sum-check protocol of Lund, Fortnow, Karloff, and Nisan \cite{FOCS:LFKN90}. Consider a field $\mathbb{F}$ and $\mathbb{H}\subset\mathbb{F}$ of size $\mathsf{poly(\lambda)}$, and consider also a polynomial $f:\mathbb{F}^\ell\to\mathbb{F}$ with individual degree $d=\mathsf{poly(\lambda)}$ and a fixed $A\in\mathbb{F}$. The sum-check protocol construct a proof that
$$
\sum_{x_1,\ldots,x_\ell\in\mathbb{H}} f(x_1,\ldots,x_\ell) = A
$$

While a naive computation of the previous statement requires time exponential in $\ell$, in the sum-check verifier's runtime and the communication complexity is only $O(\ell d|\mathbb{H}|)$ plus the evaluation of $f$ at a random point $t\in\mathbb{F}$. 

Kalai et al.~give a non-interactive variant of the sum-check protocol based on the assumption that, in symmetric bilinear groups, given $[1],[t],[t^2],\ldots,[t^d]$ and $[s],[st],[st^2],\ldots,[st^d]$, is hard to find $[\vecb{x}]\in\GG^{d+1}$ and $[s\vecb{x}]\in\GG^{d+1}$ such that $\sum_{i=1}^{d+1} x_it^{i-1} = 0$. One might think of $[t]$ as the random point chosen by the verifier in the interactive protocol that is now published in the crs together with its powers, so that polynomials in $t$ of degree $d$ are efficiently computable.

Kalai et al.~follow the work of Goldwasser et al.~where they use the sum-check protocol for constructing interactive proofs circuit satisfiability \cite{STOC:GolKalRot08}, encoding satisfiability of a circuit as recursive sum-check protocols. Further, it suffice to use sum-check protocols for polynomials of individual degree 2 and thus, the size of the assumption is just 2. 

%The idea is to compute a polynomial $f_i$ for each level of the circuit and then use the sum-check protocol to prove the satisfiability of the circuit. The firs level correspond to the inputs $(x_1,\ldots,x_n)$ and we consider the interpolation polynomial $v_0(X)=x_j$ iff $X=j$ or equivalently $v_0(X) = \sum_{z\in[n]} \lambda(z,X) x_i$ and $\lambda(z,X) = \frac{\prod_{j\neq z} (X-j)}{\prod_{j\neq z}(z-j)}$. Consider now a low-degree extension of this polynomial, in $\mathbb{H}$ such that $n=|\mathbb{H}|^\ell$,
%$$v_0(X_1,\ldots,X_\ell) = \sum_{z_1,\ldots,z_\ell\in\mathbb{H}} \lambda(z_1,\ldots,z_\ell,X_1,\ldots,X_\ell) x_{z_1,\ldots,z_\ell},$$ where $\lambda_{z_1,\ldots,z_\ell}(X_1,\ldots,X_\ell)=1$ in $z_1,\ldots,z_\ell$ and 0 in any other point of $\mathbb{H}^\ell$.
%Simlarly, we assign a polynomial $v_i$ at level $i$ as follows. For simplicity assume that the gates at level $i$ are only  multiplication gates and that gate $z_1,\ldots,z_\ell$ has inputs wires $w_1,\ldots,w_\ell$ and $y_1,\ldots,y_\ell$